\chapter{$SS\to HH$}

Author: Isabel Andrade\\

Description of the processes to be calculated: we consider the lagrangian that describes the singlet scalar model of dark matter and we choose the direct annihilation of two scalar fields of two bosons Higgs.\\

\begin{align}
  SS \to HH
\end{align}

\begin{figure}
  \centering
  %\includegraphics[scale=0.5]{interaccion}
  \caption{interaction}
\end{figure}

\section{Lagrangian}

Of the lagrangian that describes the singlet scalar model of dark matter:

\begin{equation}
  \mathcal{L}=\mathcal{L}_{sm} + \frac{1}{2} \partial_\mu S\partial^\mu S -\frac{m_0 ^2}{2} S^2 - \frac{\lambda_s}{4} S^4 - \lambda S^2H^\dagger H
\end{equation}

where H is the higgs doublet, and S is the singlet scalar field.

take the lagrangian of interaction

\begin{equation}
  \mathcal{L}^{int} = - \lambda S^2 H^\dagger H
\end{equation}

where $H$ is scalar doublet

\begin{equation}
  H=\begin{bmatrix}{0}\\{\frac{h+{v}}{\sqrt{2}}}\end{bmatrix}
\end{equation}

\begin{align}
    \mathcal{L}^{int} & ={ - \lambda S^2 \begin{bmatrix}{0}&{\frac{h+{v}}{\sqrt{2}}}\end{bmatrix} \begin{bmatrix}{0}\\{\frac{h+{v}}{\sqrt{2}}}\end{bmatrix}}\\  
    & =  {- \frac{\lambda}{2} S^2 (h+{v})(h+{v})}\\
    & = {-\frac{\lambda}{2} S^2 (hh+2 {v} h + {v^2})}\\
    & = {-\frac{\lambda}{2} S^2hh - \lambda {v} S^2 h -\frac{\lambda}{2} {v^2} S^2}
\end{align}

we take the part of the direct interaction Lagrangian

\begin{equation}
  \mathcal{L}=-\frac{\lambda}{2}S^2 hh
\end{equation}

\section{S-matrix}

for direct interaction to first order we obtain $S$

\begin{equation}
  \begin{split}
    S^{(1)} & {=- i \int {dx}^4 \mathcal{H}}\\
    & =  {- i \frac{\lambda}{2}\int {dx}^4 :hhS^2:}\\
    &  =  {- i \frac{\lambda}{2}\int {dx}^4 :{(h_++h_-)}^2{(S_++S_-)}^2 :}
  \end{split}
\end{equation}

where

\begin{equation}
  \begin{split}
    {(h_++h_-)}^2{(S_++S_-)}^2 & = (h_+h_++h_-h_++h_+h_-+h_-h_-)(S_+S_++S_-S_++S_+S_-+S_-S_-)\\
    & = {h_+h_+S_+S_++h_-h_+S_+S_++h_+h_-S_+S_++h_-h_-S_+S_+}\\
    &{h_+h_+S_-S_++h_-h_+S_-S_++h_+h_-S_-S_++h_-h_-S_-S_+}\\
    &{h_+h_+S_+S_-+h_-h_+S_+S_-+h_+h_-S_+S_-+h_-h_-S_+S_-}\\
    &{h_+h_+S_-S_-+h_-h_+S_-S_-+h_+h_-S_-S_-+h_-h_-S_-S_-}
  \end{split}
\end{equation}

To check that only the ordered terms are different from zero we can analyse the full terms for initial and final states defined as:
$\left| S(\vec{p_1})S(\vec{p_2}) \right\rangle $ initial, 
$\left\langle  h(\vec{p'_1}) h(\vec{p'_2})\right|$ final\\

with $s_+, h_+$ associated $ \hat {a}$, $S_+ \left |n\right \rangle$ and $h_+ \left|n\right\rangle$ annihilate

$s_-, h_-$ associated $ \hat {a^{\dagger}}$, $S_+ \left |n\right \rangle$ and $h_+ \left|n\right\rangle$ creation\\

matrix elements\\

\begin{equation}
  \begin{split}
    &\left \langle 1_h 1_h 0_S 0_S | h_+ h_+ S_+ S_+|0_h 0_h 1_S 1_s\right\rangle \propto \left \langle 2_h 2_h 0_S 0_S|0_h 0_h 0_S 0_S\right \rangle=0\\
    &\left \langle 1_h 1_h 0_S 0_S | h_- h_+ S_+ S_+|0_h 0_h 1_S 1_s\right\rangle \propto \left \langle 0_h 2_h 0_S 0_S|0_h 0_h 0_S 0_S\right \rangle=0\\
    &\left \langle 1_h 1_h 0_S 0_S | h_+ h_- S_+ S_+|0_h 0_h 1_S 1_s\right\rangle \propto \left \langle 2_h 0_h 0_S 0_S|0_h 0_h 0_S 0_S\right \rangle=0\\
    &\left \langle 1_h 1_h 0_S 0_S | h_- h_- S_+ S_+|0_h 0_h 1_S 1_s\right\rangle \propto \left \langle 0_h 0_h 0_S 0_S|0_h 0_h 0_S 0_S\right \rangle\neq 0\\
    &\left \langle 1_h 1_h 0_S 0_S | h_+ h_+ S_- S_+|0_h 0_h 1_S 1_s\right\rangle \propto \left \langle 2_h 2_h 0_S 0_S|0_h 0_h 1_S 0_S\right \rangle=0\\
    &\left \langle 1_h 1_h 0_S 0_S | h_- h_+ S_- S_+|0_h 0_h 1_S 1_s\right\rangle \propto \left \langle 0_h 2_h 0_S 0_S|0_h 0_h 1_S 0_S\right \rangle=0\\
    &\left \langle 1_h 1_h 0_S 0_S | h_+ h_- S_- S_+|0_h 0_h 1_S 1_s\right\rangle \propto \left \langle 2_h 0_h 0_S 0_S|0_h 0_h 1_S 0_S\right \rangle=0\\
    &\left \langle 1_h 1_h 0_S 0_S | h_- h_- S_- S_+|0_h 0_h 1_S 1_s\right\rangle \propto \left \langle 0_h 0_h 0_S 0_S|0_h 0_h 1_S 0_S\right \rangle=0\\
    &\left \langle 1_h 1_h 0_S 0_S | h_+ h_+ S_+ S_-|0_h 0_h 1_S 1_s\right\rangle \propto \left \langle 2_h 2_h 0_S 0_S|0_h 0_h 0_S 1_S\right \rangle=0\\
    &\left \langle 1_h 1_h 0_S 0_S | h_- h_+ S_+ S_-|0_h 0_h 1_S 1_s\right\rangle \propto \left \langle 0_h 2_h 0_S 0_S|0_h 0_h 0_S 1_S\right \rangle=0\\
    &\left \langle 1_h 1_h 0_S 0_S | h_+ h_- S_+ S_-|0_h 0_h 1_S 1_s\right\rangle \propto \left \langle 2_h 0_h 0_S 0_S|0_h 0_h 0_S 1_S\right \rangle=0\\
    &\left \langle 1_h 1_h 0_S 0_S | h_- h_- S_+ S_-|0_h 0_h 1_S 1_s\right\rangle \propto \left \langle 0_h 0_h 0_S 0_S|0_h 0_h 0_S 1_S\right \rangle=0\\
    &\left \langle 1_h 1_h 0_S 0_S | h_+ h_+ S_- S_-|0_h 0_h 1_S 1_s\right\rangle \propto \left \langle 2_h 2_h 0_S 0_S|0_h 0_h 2_S 2_S\right \rangle=0\\
    &\left \langle 1_h 1_h 0_S 0_S | h_- h_+ S_- S_-|0_h 0_h 1_S 1_s\right\rangle \propto \left \langle 0_h 2_h 0_S 0_S|0_h 0_h 2_S 2_S\right \rangle=0\\
    &\left \langle 1_h 1_h 0_S 0_S | h_+ h_- S_- S_-|0_h 0_h 1_S 1_s\right\rangle \propto \left \langle 2_h 0_h 0_S 0_S|0_h 0_h 2_S 2_S\right \rangle=0\\
    &\left \langle 1_h 1_h 0_S 0_S | h_- h_- S_- S_-|0_h 0_h 1_S 1_s\right\rangle \propto \left \langle 0_h 0_h 0_S 0_S|0_h 0_h 2_S 2_S\right \rangle=0
  \end{split}
\end{equation}

The only term that contributes to the matrix element of the process is

\begin{equation}
  S^{(1)}=-i\frac{\lambda}{2} \int{{dx}^4 \left\langle 1_h 1_h 0_S 0_S|h_- h_- S_+ S_+|0_h 0_h 1_S 1_S \right \rangle}
\end{equation}
 
Let us define the one–particle states

\begin{equation}
  \begin{split}
    &\left| h(p'_1)\right \rangle \equiv \frac{1}{\sqrt{V}} \hat{a}_{p'_1}^{\dagger} \left| 0 \right \rangle\\ 
    &\left| h(p'_2)\right \rangle \equiv \frac{1}{\sqrt{V}} \hat{a}_{p'_2}^{\dagger} \left | 0 \right \rangle
  \end{split}
\end{equation}

\begin{equation}
  \begin{split}
    &\left| S(p_1)\right \rangle \equiv \frac{1}{\sqrt{V}} \hat{a}_{p_1}^{\dagger} \left| 0 \right \rangle\\ 
    &\left| S(p_2)\right \rangle \equiv \frac{1}{\sqrt{V}} \hat{a}_{p_2}^{\dagger} \left | 0 \right \rangle
  \end{split}
\end{equation}

our states are then normalizad as

\begin{equation}
  \begin{split}
    &\left\langle| h(p_1) | h(p'_1)\right \rangle = \frac{2\pi^3}{V} \delta (p_1-p'_1)\\
    &\left\langle | h(p_2) | h(p'_2)\right \rangle = \frac{2\pi^3}{V} \delta(p_2-p'_2)
  \end{split}
\end{equation}

\begin{equation}
  \begin{split}
    &\left\langle| S(p_1) | S(p'_1)\right \rangle = \frac{2\pi^3}{V} \delta (p_1-p'_1)\\
    &\left\langle | S(p_2) | S(p'_2)\right \rangle = \frac{2\pi^3}{V} \delta(p_2-p'_2)
  \end{split}
\end{equation}

Now we can write down the action of various field operators on different one particles states

\begin{equation}
  \begin{split}
    S_+ \left| S(p_1) \right \rangle & = \int{ {dp}^3 \frac{1}{2\pi^3} \frac{1}{\sqrt{2E_{p_1}}} \hat{a}_p^\dagger e^{-ip.x} \frac{1}{\sqrt{V}} \hat{a}_{p_1}^\dagger \left | 0 \right \rangle}\\
    &=\int {{dp}^3 \frac{1}{2\pi^3} \frac{1}{\sqrt{2E_{p_1}}} e^{-ip.x} (\hat{a}_p \hat{a}_{p_1}^\dagger - \hat{a}_{p_1}^\dagger \hat{a}_p)\left | 0 \right \rangle }\\
    &=\int {{dp}^3 \frac{1}{2\pi^3} \frac{1}{\sqrt{2E_{p_1}V}} e^{-ip.x} 2\pi^3 \delta(p-p_1) \left | 0 \right \rangle }\\
    &= \frac{1}{\sqrt{2E_{p_1}V}} e^{-ip_1.x} \left | 0 \right \rangle
  \end{split}
\end{equation}

where $\hat{a}_p \left | 0 \right \rangle$ is zero

\begin{equation}
  \begin{split}
    S_+ \left| S(p_2) \right \rangle & = \int{ {dp}^3 \frac{1}{2\pi^3} \frac{1}{\sqrt{2E_{p_2}}} \hat{a}_p^\dagger e^{-ip.x} \frac{1}{\sqrt{V}} \hat{a}_{p_2}^\dagger \left | 0 \right \rangle}\\
    &=\int {{dp}^3 \frac{1}{2\pi^3} \frac{1}{\sqrt{2E_{p_2}}} e^{-ip.x} (\hat{a}_p \hat{a}_{p_2}^\dagger - \hat{a}_{p_2}^\dagger \hat{a}_p)\left | 0 \right \rangle }\\
    &=\int {{dp}^3 \frac{1}{2\pi^3} \frac{1}{\sqrt{2E_{p_2}V}} e^{-ip.x} 2\pi^3 \delta(p-p_2) \left | 0 \right \rangle }\\
    &= \frac{1}{\sqrt{2E_{p_2}V}} e^{-ip_2.x} \left | 0 \right \rangle
  \end{split}
\end{equation}

\begin{equation}
  \begin{split}
    h_+ \left| h(p'_1) \right \rangle & = \int{ {dp}^3 \frac{1}{2\pi^3} \frac{1}{\sqrt{2E_{p'_1}}} \hat{a}_p^\dagger e^{-ip.x} \frac{1}{\sqrt{V}} \hat{a}_{p'_1}^\dagger \left | 0 \right \rangle}\\
    &=\int {{dp}^3 \frac{1}{2\pi^3} \frac{1}{\sqrt{2E_{p'_1}}} e^{-ip.x} (\hat{a}_p \hat{a}_{p'_1}^\dagger - \hat{a}_{p'_1}^\dagger \hat{a}_p)\left | 0 \right \rangle }\\
    &=\int {{dp}^3 \frac{1}{2\pi^3} \frac{1}{\sqrt{2E_{p'_1}V}} e^{-ip.x} 2\pi^3 \delta(p-p'_1) \left | 0 \right \rangle }\\
    &= \frac{1}{\sqrt{2E_{p'_1}V}} e^{-ip'_1.x} \left | 0 \right \rangle
  \end{split}
\end{equation}

his adjoint operator

\begin{equation}
  \left \langle h(p'_1) \right | h_- = \frac{1}{\sqrt{2E_{p'_1}V}} e^{ip'_1.x} \left \langle 0 \right |
\end{equation}

\begin{equation}
  \begin{split}
    h_+ \left| h(p'_2) \right \rangle & = \int{ {dp}^3 \frac{1}{2\pi^3} \frac{1}{\sqrt{2E_{p'_2}}} \hat{a}_p^\dagger e^{-ip.x} \frac{1}{\sqrt{V}} \hat{a}_{p'_2}^\dagger \left | 0 \right \rangle}\\
    &=\int {{dp}^3 \frac{1}{2\pi^3} \frac{1}{\sqrt{2E_{p'_2}}} e^{-ip.x} (\hat{a}_p \hat{a}_{p'_2}^\dagger - \hat{a}_{p'_2}^\dagger \hat{a}_p)\left | 0 \right \rangle }\\
    &=\int {{dp}^3 \frac{1}{2\pi^3} \frac{1}{\sqrt{2E_{p'_2}V}} e^{-ip.x} 2\pi^3 \delta(p-p'_2) \left | 0 \right \rangle }\\
    &= \frac{1}{\sqrt{2E_{p'_2}V}} e^{-ip'_2.x} \left | 0 \right \rangle
  \end{split}
\end{equation}

his adjoint operator

\begin{equation}
  \left \langle h(p'_2) \right | h_- = \frac{1}{\sqrt{2E_{p'_2}V}} e^{ip'_2.x} \left \langle 0 \right |
\end{equation}

the matrix element at first order between the initial and final stetes is 

\begin{equation}
  \begin{split}
    S_{fi}^{(1)} &= -i \frac{\lambda}{2} \int{ d^4x \left \langle h(p'_1) h(p'_2)| h_- h_- S_+ S_+ | S(p_1) S(p_2) \right \rangle }\\
    &=-i \frac{\lambda}{2} \int{ d^4x e^{i(p'_1+p'_2-p_1-p_2)} \frac{1}{\sqrt{2E_{p'_1}V}} \frac{1}{\sqrt{2E_{p'_2}V}} \frac{1}{\sqrt{2E_{p_1}V}} \frac{1}{\sqrt{2E_{p_2}V}} } 
  \end{split}
\end{equation}

since

\begin{align}
  \int{ d^4 x e^{i(p'_1+p'_2-p_1-p_2)} \equiv {(2 \pi)}^4 \delta (p'_1 + p'_2 - p_1 - p_2)}
\end{align}

\begin{equation}
  S_{fi}^{(1)} = \begin{bmatrix}\frac{1}{\sqrt{2E_{p'_1}V}} & \frac{1}{\sqrt{2E_{p_2}V}} & \frac{1}{\sqrt{2E_{p_1}v}} & \frac{1}{\sqrt{2E_{p_2}V}} \end{bmatrix} (2\pi)^4 \delta^4 (p'_1 + p'_2 - p_1 - p_2) \left( \frac{-i\lambda}{2} \right)
\end{equation}

where the amplitude is

\begin{equation}
  \mathcal{M} = -i \frac{\lambda}{2}
\end{equation}

\section{Process calculation cross section}

\begin{figure}
  \centering
  %\includegraphics[scale=0.5]{seccion_eficaz_1}
  \caption{interaction from the center of mass}
\end{figure}

using equation

\begin{equation}
  \frac{d\sigma}{d\Omega} = \frac{1}{64 \pi^2 s} \left\{ \frac{ [s-(m'_1 +m'_2)^2]}{[s-(m_1 + m_2)^2]} \frac{[s-(m'_1 - m'_2)^2 ]}{[s-(m_1 -m_2)^2]} \right\}^{\frac{1}{2}} | \bar{\mathcal{M}} |^2  
\end{equation}

with

\begin{equation}
  \sqrt{s} = E_1 + E_2
\end{equation}

\begin{equation}
  |\bar{ \mathcal{M} } |^2 = \frac{\lambda^2}{4}
\end{equation}

$m'_1 =m'_2 $ and $m_1 =m_2$  

\begin{equation}
  \frac{d\sigma}{d\Omega} = \frac{1}{64 \pi^2 s} \left\{ \frac{ [s-(m'_1 +m'_2)^2]}{[s-(m_1 + m_2)^2]} \right\}^{\frac{1}{2}}  \frac{\lambda^2}{4}  
\end{equation}

the cross section

\begin{equation}
  \begin{split}
    \sigma &= \frac{1}{64 \pi^2 s} \left\{ \frac{ [s-(m'_1 +m'_2)^2]}{[s-(m_1 + m_2)^2]} \right\}^{\frac{1}{2}}  \frac{\lambda^2}{4} \int{ d\Omega}\\
    &= \frac{1}{64 \pi^2 s} \left\{ \frac{ [s-(m'_1 +m'_2)^2]}{[s-(m_1 + m_2)^2]} \right\}^{\frac{1}{2}}  \frac{\lambda^2}{4} \int{ d\theta d\varphi \sin{\theta}}\\
    &= \frac{1}{64 \pi s} \left\{ \frac{ [s-(m'_1 +m'_2)^2]}{[s-(m_1 + m_2)^2]} \right\}^{\frac{1}{2}}  \lambda^2
  \end{split}
\end{equation}

for $m_1=m_2= 50  Gev, m'_1=m'_2=120  Gev,E_1=E_2=500  Gev, \lambda = 0.1$\\

$\sigma = 4.734457651 * 10^{-11} Gev^{-2}$

\section{CalcHEP comparison}

Calhep results\\

for the process:
\begin{align}
  S,S \to H,H 
\end{align}

\begin{figure}
  \centering
  %\includegraphics[scale=0.5]{singlete}
  \caption{interaction from the center of mass}
\end{figure}

contributes only the direct process

\begin{figure}
  \centering
  %\includegraphics[scale=0.5]{diagrama}
  \caption{diagrams}
\end{figure}

\begin{figure}
  \centering
  %\includegraphics[scale=0.5]{seccion_efic}
  \caption{cross section}
\end{figure}

\section{Copyright}
\includegraphics[scale=0.5]{cc} Creative Commons Attribution-Share Alike 3.0 United States License.




%%% Local Variables:
%%% mode: latex
%%% TeX-master: "qft_samples"
%%% End:

