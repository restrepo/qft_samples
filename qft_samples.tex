\documentclass[12pt,letterpaper]{book}
\usepackage{amsmath}
\usepackage{amssymb}
\usepackage{amsthm}
\usepackage[utf8]{inputenc}
\usepackage[colorlinks=true]{hyperref}
\usepackage{graphicx}
\usepackage{cancel}
\usepackage[svgnames]{xcolor}
\usepackage[english]{babel}
\usepackage{bm}
\usepackage{simplewick}
\usepackage[toc,page]{appendix}
\graphicspath{{figures/}}
\usepackage{listings}
%%%%%%%%%
\usepackage{comment}
\includecomment{inprogress}
\specialcomment{inprogress}
{\begingroup\itshape \textbf{Begin Draft:}}{\textbf{End Draft.} \medskip \endgroup}
\excludecomment{inprogress}

%MY COMMANDS
\newcommand{\pr}[1]{ \left( #1 \right) }
\newcommand{\cor}[1]{ \left[ #1 \right] }
\newcommand{\lla}[1]{ \left\{ #1 \right\} }
\newcommand{\ket}{\rangle}
\newcommand{\bra}{\langle}
\newcommand{\eq}[2]{\begin{equation} \label{eq#1} #2 \end{equation}}
%comandos para cosas en QM
\newcommand{\ketm}[1]{\left|#1\right\rangle}
\newcommand{\bram}[2]{\left\langle#1\right|}
\newcommand{\valprom}[1]{\left<#1\right>}
\newcommand{\prodin}[2]{\left<#1\right|#2\left.\right>}
\newcommand{\braket}[3]{\left<#1\right|#2\left|#3\right>}
\newcommand{\conmu}[2]{\left[#1,#2\right]}
\newcommand{\aconmu}[2]{\left\{#1,#2\right\}}

%comandos varios
\newcommand{\detott}[1]{\frac{d}{d#1}}
\newcommand{\detot}[2]{\frac{d#1}{d#2}}
\newcommand{\deparr}[1]{\frac{\partial}{\partial #1}}
\newcommand{\depar}[2]{\frac{\partial #1}{\partial #2}}
\newcommand{\tsr}[3]{{#1}_{#2}^{#3}}

%abreaviación de comandos existentes
\newcommand{\vtr}[1]{\boldsymbol{#1}}
\newcommand{\vecun}[1]{\boldsymbol{\hat{e}_{#1}}}
\newcommand{\mc}[1]{\mathcal{#1}}
\newcommand{\nn}{\nonumber}




%\usepackage{bbold}
\setlength{\textwidth}{180mm}
\setlength{\oddsidemargin}{-1cm}
\setlength{\evensidemargin}{-1cm}
%\setlength{\topmargin}{-40pt}
\def\lt{<} % compatibility with instiki
\def\gt{>} % compatibility with instiki
\newtheoremstyle{example}{\topsep}{\topsep}%
     {}%         Body font
     {}%         Indent amount (empty = no indent, \parindent = para indent)
     {\bfseries}% Thm head font
     {}%        Punctuation after thm head
     {\newline}%     Space after thm head (\newline = linebreak)
     {\thmname{#1}\thmnumber{ #2}\thmnote{ #3}}%         Thm head spec

   \theoremstyle{example}
   \newtheorem{example}{Example}[subsection]
%%comment the aproppiate line
\newcommand{\tofc}[1]{#1} % if full
%\renewcommand{\tofc}[1]{} %if  \includeonly{chapter1}
%\includeonly{introduction}

%%%%%%%%%%%%%%%%%%%
\begin{document}
\tofc{
\renewcommand{\thepage}{\roman{page}}
\tableofcontents{}

\newpage{}
}

\renewcommand{\thepage}{\arabic{page}}
\setcounter{page}{1}

%Uncommet to include chapter

\chapter{Calculation sample}

Author: Diego Restrepo

Description of the processes to be calculated

The Figures must be in PDF and included in the directory: \verb|figures|.

\section{Lagrangian}
Relevant Lagrangian terms.

\section{S-matrix}
Describe here how to obtain the $S$--matrix to the required order.

\section{Process calculation}
Obtain the cross section or the decay width

\section{CalcHEP comparison}
Check the result with CalcHEP. Please give the LanHEP code if necessary.


\section{Copyright}
\includegraphics[scale=0.5]{cc} Creative Commons Attribution-Share Alike 3.0 United States License.




%%% Local Variables: 
%%% mode: latex
%%% TeX-master: "qft_samples"
%%% End: 

% $e^+ e^- \to e^- e^-$: Diego
%\include{e-e-to_e-e-} 
% $e^- e^+ \to \mu^+ \mu^-$: David Noreña
\chapter{Dispersión $e^{-}e^{+}\longrightarrow \mu^{+}\mu^{-}$}


Author: David Andrés Noreña Blandón

El proceso tratado es la dispersión inelástica de un par electrón positrón. A partir de este proceso, se generan 
un par partícula-antipartícula correspondiente a muón-antimuón, en este documento se calcula la sección eficaz de la producción
muón-antimuón. Los campos fermiónicos son: del electrón $\psi_e$, del positrón es $\bar{\psi}_e$, del muón $\psi_{\mu}$ y del antimuón $\bar{\psi}_{\mu}$ \\
La masa del par inicial no se considera debido a que la energía involucrada es mucho mayor que la masa del electrón. La masa del 
 muón y el antimuón es $m_{\mu}$ cuyos cuadrimomentos son $p_{-}, p_{+}$ respectivamente. Los cuadrimomentos del electrón y del positrón son $k_{-}, k_{+}$, 
Este proceso se puede denotar por :
\[
e^{-}(k_{-})e^{+}(k_{+})\longrightarrow \mu^{+}(p_{+})\mu^{-}(p_{-}) 
\]

El objetivo es calcular los elementos de la matriz $S$ y la amplitud de decaimiento $\Gamma$ para este proceso.
En la figura 1 se muestra un diagrama de Feynman del proceso en estudio.

\section{Lagrangiano}
El término relevante del lagrangiano del modelo estándar es:
\begin{equation}
 \mathcal{L}=\bar{e}_L\gamma^{\nu}e_L A_{\nu}\bar{\mu}_L\gamma^{\mu}\mu_L A_{\mu}
\end{equation}

\section{Matriz S}
\begin{figure}
\begin{center}
\includegraphics[width=0.5\textwidth]{./ProcesoDavid.pdf}
 % Proceso_David.pdf: 0x0 pixel, 0dpi, 0.00x0.00 cm, bb=
 \caption{Diagrama de Feynman del proceso}
\end{center}
\end{figure}
El Hamiltoniano de interaccion deriva del lagrangiano de la manera usual. Para la dispersión de este caso

\begin{equation}
  \mathcal{H}_I=-:\mathcal{L}:
=-:\bar{\psi}\gamma^{\mu}\psi\phi:
\end{equation}
Ahora miremos el segundo término en la expansión de la matriz $S$
\[
 S^{(2)}=\frac{(-i)^2}{2!}\int \int d^4x_1 d^4x_2 T\{\mathcal{H}_I(x_1)\mathcal{H}_I(x_2)\}
\]
\[
 = \frac{(-i)^2}{2!}\int \int d^4x_1 d^4x_2 T\{:(\bar{\psi}_e\gamma^{\mu}\psi_e\phi)_{x_1}(\bar{\psi}_{\mu}\gamma^{\mu}\psi_{\mu}\phi)_{x_2}:\}
\]

Donde el único término que contribuye al elemento de matriz en el proceso viene dado por:
\[
 S^{(2)}=\frac{(-i)^2}{2!} \int \int d^4x_1 d^4x_2 \phi(x_1)\phi(x_2):(\bar{\psi}_e\gamma^{\mu}\psi_e)_{x_1}(\bar{\psi}_{\mu}\gamma^{\mu}\psi_{\mu})_{x_2}:
\]
Con 
\[
 \phi(x_1)\phi(x_2)=i\Delta_F (x_1-x_2)
\]
Y además:
\[
:(\bar{\psi}_e\gamma^{\mu}\psi_e)_{x_1}(\bar{\psi}_{\mu}\gamma^{\mu}\psi_{\mu})_{x_2}:=:\bar{\psi}_e(x_1)\gamma^{\mu}\psi_e(x_1)\bar{\psi}_{\mu}(x_2)\gamma^{\mu}\psi_{\mu}(x_2):
\]
Descomponiendo los campos fermiónicos en $\psi_+$ y $\psi_-$, el término que contribuye es:
\[
 -\phi(x_1)\phi(x_2)\bar{\psi}^{\alpha}_{e+}(x_1)\gamma^{\mu}\bar{\psi}^{\beta}_{\mu-}(x_2)\psi^{\alpha}_{e+}(x_1)\gamma^{\mu}\psi^{\beta}_{\mu-}(x_2)
\]
  

El elemento de la matriz $S$ entre el estado inicial y el estado final es, simplificando la notación
\[
 S^{(2)}_{fi}=\frac{(-i)^2}{2!} \int \int d^4x_1 d^4x_2 i\Delta_F (x_1-x_2)
\]
\[
\times \langle \mu^{-}(\mathbf{p}_{-})\mu^{+}(\mathbf{p}_{+})|\bar{\psi}^{\mu}_{-}(x_2)\gamma^{\nu}{\psi}^{\mu}_{-}(x_2)\bar{\psi}^{e}_{+}(x_1)\gamma^{\mu}\psi^{e}_{+}(x_1)|e^{+}(\mathbf{k}_{+})e^{-}(\mathbf{k}_{-}) \rangle
\]
El estado de Fock de dos partículas es: 
\begin{equation}
 |e^{+}(\mathbf{k}_{+})e^{-}(\mathbf{k}_{-})\rangle=\sqrt{\frac{1}{V}}b_{k_{-}}^{\dagger}d_{k_{+}}^{\dagger}|0\rangle
\end{equation}
\begin{equation}
 |\mu^{+}(\mathbf{p}_{+})\mu^{-}(\mathbf{p}_{-})\rangle=\sqrt{\frac{1}{V}}B_{p_{-}}^{\dagger}D_{p_{+}}^{\dagger}|0\rangle
\end{equation}
Donde $b$ y $d$ son los operadores creación y aniquilación para el electrón, $B$ y $D$ son los operadores creación y aniquilación
del muón.\\
Usando la descomposición de Fourier de los campos fermiónicos, se calcula la acción de los operadores de campo sobre los estados
de las partículas, dando como resultado:
\[
 \bar{\psi}^{e}_{+}(x_1)\gamma^{\mu}\psi^{e}_{+}(x_1)|e^{+}(\mathbf{k}_{+})e^{-}(\mathbf{k}_{-}) \rangle=
\]
\[
\frac{1}{(2\pi)^3\sqrt{2E_{k_{+}} V}}\frac{1}{(2\pi)^3\sqrt{2E_{k_{-}} V}}\bar{v}(\mathbf{k_{+}})\gamma^{\mu}\bar{u}(\mathbf{k_{-}})e^{-i(k_{+}+k_{-}).x_1}|0\rangle
\]
\[
\langle \mu^{-}(\mathbf{p}_{-})\mu^{+}(\mathbf{p}_{+})|\bar{\psi}^{\mu}_{-}(x_2)\gamma^{\nu}{\psi}^{\mu}_{-}(x_2)
\]
\[
=\langle 0|\frac{1}{(2\pi)^3\sqrt{2E_{p_{+}} V}}\frac{1}{(2\pi)^3\sqrt{2E_{p_{-}} V}}\bar{u}(\mathbf{p_{+}})\gamma^{\nu}{u}(\mathbf{p_{-}})e^{i(p_{+}+p_{-}).x_2}
\]
Luego de integrar, tomando en cuanta la conservación de la carga, el elemento de matriz es: 

Así nos queda que el elemento de matriz de primer orden entre los estados inicial y final es:
\[
 S^{(2)}_{fi}=-e^2 \int \int d^4x_1d^4x_2 e^{i(p_{+}+p_{-}).x_2-i(k_{+}+k_{-}).x_1}
\]
\[
\times \frac{[\bar{u}(\mathbf{p_-})\gamma^{\rho} v(\mathbf{p_+})][\bar{v}(\mathbf{k_+})\gamma^{\lambda}u(\mathbf{k_-})]}{\sqrt{2E_{k_+} V}\sqrt{2E_{k_-} V}\sqrt{2E_{p_{+}} V}\sqrt{2E_{p_{-}} V}}\frac{g_{\lambda \rho}}{s}
\]

\[
=-e^2\left[\frac{1}{\sqrt{2E_{k_+} V}\sqrt{2E_{k_-} V}\sqrt{2E_p V}\sqrt{2E_{p'}V}} \right](2\pi)^4 \delta^4[(k_+ +k_-)-(p_+ +p_-)]
\]
\begin{equation}
\times i\frac{e^2}{s}[\bar{u}(\mathbf{p_-})\gamma_{\lambda} v(\mathbf{p_+})][\bar{v}(\mathbf{k_+})\gamma^{\lambda}u(\mathbf{k_-})]
\end{equation}
$s$ es la variable de Mandelstam, proveniente del propagador del fotón.


\section{Cálculo del proceso}
De la expresión anterior de $S^{(2)}_{fi}$ se tiene que
\begin{equation}
 i\mathcal{M}_{fi}= i\frac{e^2}{s}\left[\bar{u}(\mathbf{p_-})\gamma_{\lambda} v(\mathbf{p_+})\right] \left[\bar{v}(\mathbf{k_+})\gamma^{\lambda}u(\mathbf{k_-})\right]
\end{equation}
Luego:
\[
 |\mathcal{M}_{fi}|^2=\frac{1}{4}\sum_{\mbox{spin}}\frac{e^4}{s^2}\left\{ \left[\bar{u}(\mathbf{p_-})\gamma_{\lambda} v(\mathbf{p_+})\right] \left[\bar{v}(\mathbf{k_+})\gamma^{\lambda}u(\mathbf{k_-})\right] \right\}^2
\]
\[
 =\frac{e^4}{4s^2}\mbox{Tr}\left[\not k_{+}\gamma^{\lambda} \not k_-\gamma^{\rho} \right]\mbox{Tr}\left[(\not p_- +m_{\mu})\gamma_{\lambda}(\not p_+-m_{ \mu}) \right]
\]
\[
 =\frac{4e^4}{s^2}\left[k_{-}^{\lambda}k_+^{\rho}+ k_{-}^{\rho}k_+^{\lambda}-g_{\lambda \rho}k_-\cdot k_+ \right]\left[ p_{-\lambda}p_{+\rho}+ p_{-\rho}k_{+\lambda}-g_{\lambda \rho}(p_-\cdot p_+ +m_{\mu}^{2})\right]
\]
\[
 =\frac{8e^4}{s^2}\left[(k_-\cdot p_-)^2+(k_-\cdot p_+)^2+m_{\mu}k_-\cdot k_+ \right]
\]
Dado que $k_-\cdot p_-=k_+\cdot p_+$.\\
Para el sistema de referencia del centro de masa, $\theta$ es el ángulo entre el electrón entrante y el muón saliente. $E$ es la energía del 
electrón entrante, $s=4E^2$ y $p=|\mathbf{p_-}|=|\mathbf{p_+}|=\frac{1}{2}\sqrt{s-4m_{\mu}^{2}}$, entonces:
\[
 k_-\cdot p_-=E(E-p \cos \theta)
\]
\[
 k_-\cdot p_+=E(E+p \cos \theta)
\]
\[
 k_-\cdot k_+=2E^2
\]
Luego:
\[
 |\mathcal{M}_{fi}|^2=e^4\left[1+ \frac{1}{E^2}\left(p^2 \cos^2 \theta+m_{\mu}^{2} \right) \right]=e^4\left[1+ \cos^2\theta+\frac{m_{\mu}^{2}}{E^2}\sin^2\theta \right]
\]
Reemplazando en la fórmula general para la sección eficaz diferencial dada por:
\[
 \frac{d\sigma}{d\Omega}=\frac{1}{64 \pi^2 s}\left[\frac{\{s-(m'_1+m'_2)^2\}\{s-(m'_1-m'_2)^2 \}}{\{s-(m_1+m_2)^2\}\{s-(m_1-m_2)^2 \}} \right]^{1/2}|\mathcal{M}_{fi}|^2
\]
Se obtiene:
\[
 \frac{d\sigma}{d\Omega}=\frac{\alpha^2}{4s}\sqrt{1-\frac{4m^{2}_{\mu}}{s}}\left[1+ \cos^2\theta+\frac{m_{\mu}^{2}}{E^2}\sin^2\theta \right]
\]
Por lo tanto, la sección eficaz total es:
\[
 \sigma=\frac{4\pi\alpha^2}{3s}\sqrt{1-\frac{4m^{2}_{\mu}}{s}}\left[1+ \frac{2m_{\mu}^{2}}{s} \right]
\]


\section{Cálculo en CalcHEP}
Al realizar el proceso en CalcHEP, se obtuvo el siguiente resultado, para\\
\texttt{p(c.m.s.): 7000[GeV]}\\
\texttt{Cos(p1,p3): min=-0.999, max=0.999}\\
La sección eficaz es:
\texttt{Cross Section: 0.0005779 [pb]}



\section{Copyright}
\includegraphics[scale=0.5]{cc} Creative Commons Attribution-Share Alike 3.0 United States License.


%%% Local Variables: 
%%% mode: latex
%%% TeX-master: "qft_samples"
%%% End: 




% $\mu^- \to e^- \bar{\nu}_e \nu_\mu$: Cristian
% \include{mudecay}
% Singlet dark matter: Isabel 0909.2799
\chapter{Calculation sample of the singlet scalar dark matter}

Author: Isabel Andrade

Description of the processes to be calculated: we consider the lagrangian that describes the singlet scalar model of dark matter and we choose the direct annihilation of two scalar fields of two bosons Higgs.

The Figures must be in PDF and included in the directory: \verb|figures|.

\section{Lagrangian}

Of the lagrangian that describes the singlet scalar model of dark matter:

\begin{equation}
\mathcal{L}=\mathcal{L}_s_m + \frac{1}{2} \partial_\mu S\partial^\mu S -\frac{m_0 ^2}{2} S^2 - \frac{\lambda_s}{4} S^4 - \lambda S^2H^\dagger H
\end{equation}

where H is the higgs doublet, and S is the singlet scalar field.

take the lagrangian of interaction

\begin{equation}
\mathcal{L}^{int} = - \lambda S^2 H^\dagger H
\end{equation}

where $H$ is scalar doublet

\begin{equation}
H=\begin{bmatrix}{0}\\{\frac{h+\mathfrak{v}}{\sqrt{2}}}\end{bmatrix}
\end{equation}

\begin{equation}
\mathcal{L}^{int}= - \lambda S^2 \begin{bmatrix}{0}&{\frac{h+\mathfrak{v}}{\sqrt{2}}}\end{bmatrix} \begin{bmatrix}{0}\\{\frac{h+\mathfrak{v}}{\sqrt{2}}}\end{bmatrix}
\end{equation}

\section{S-matrix}
Describe here how to obtain the $S$--matrix to the required order.

\section{Process calculation}
Obtain the cross section or the decay width

\section{CalcHEP comparison}
Check the result with CalcHEP. Please give the LanHEP code if necessary.


\section{Copyright}
\includegraphics[scale=0.5]{cc} Creative Commons Attribution-Share Alike 3.0 United States License.




%%% Local Variables:
%%% mode: latex
%%% TeX-master: "qft_samples"
%%% End:


% Singlet dark matter: Isabel 0909.2799
\chapter{Cálculo de la aniquilación $SS\rightarrow Z^0Z^0$}
Author: Sebastian Bustamante

Descripcion del proceso: 

\section{Descripción del proceso.}

La aproximación usada en \cite{Goud} para la materia oscura corresponde a un singlete escalar que solo está acoplado con el Higgs del modelo estándar, el lagrangiano más general que describe esta partícula es por tanto
\eq{1}
{ \mathcal{L} = \mathcal{L}_{SM} + \frac{1}{2}\partial_\mu S\partial^\mu S - \frac{m_0^2}{2}S^2 - \frac{\lambda_S}{4}S^4 - \lambda S^2 H^\dag H }
donde $S$ es el campo escalar propuesto, $m_0$ la masa del campo y $\lambda$ es un parámetro que caracteriza el acople con el Higgs.

Del lagrangiano del modelo estándar \cite{profe}, los únicos estados finales por decaimiento de este campo $S$ a través del canal s son $f\bar f$, $W^+W^-$, $Z^0Z^0$ y $hh$. En especial en este documento se calculará todo lo referente al proceso $SS\rightarrow Z^0Z^0$, para el cual el diagrama de Feynmann asociado es

\begin{figure}[htbp]
	\centering
	\includegraphics[width=0.5\textwidth]{Sebastian_fig01.png}
	\caption{Diagrama de Feynmann para el proceso $SS\rightarrow Z^0Z^0$.}
	\label{fig:fig01}
\end{figure}

Puesto que el proceso es a través del canal s este debe tener dos vértices, tal como se ve en el diagrama anterior.

\section{Lagrangiano del proceso}

El único término de interés del lagrangiano de interacción del modelo estándar es aquel que acopla dos bosones $Z^0$ con un higgs, este está dado por \cite{profe}
\eq{2}
{ \mathcal{L}_{SM,I} = \frac{m_Z^2}{v} Z^2 h }
y por tanto el lagrangiano de interacción total es
\eq{3}
{ \mathcal{L}_{T} = \frac{m_Z^2}{v} Z^2 h - \lambda S^2 H^\dag H   }
donde por simplicidad en notación se ha hecho $Z\equiv Z^0$.

Se hace conviente introducir el doblete de higgs del modelo estándar de la forma
\[ H = \frac{1}{\sqrt 2}
\begin{pmatrix}
  0 \\\ h(x) + v
\end{pmatrix} \]
de esto
\[ H^\dag H = \frac{1}{2}\pr{ v^2 + hh + 2vh } \]
y luego
\[ \mathcal{L}_{T} = \frac{m_Z^2}{v} Z^2 h - \frac{\lambda }{2}S^2\pr{ v^2 + hh + 2vh }  \]
Teniendo en cuenta el estado inicial del proceso en estudio, se tiene que solo es interesante el término de acople de $S^2$ con un higgs $h$, por tanto el lagrangiano de interacción relevante para los cálculos que serán realizados es finalmente
\eq{4}
{ \mathcal{L}_{I} = \frac{m_Z^2}{v} Z^2 h - \lambda vS^2 h  }
como no hay dependencia en las derivadas de los campos, el hamiltoniano de interacción es
\eq{5}
{ \mathcal{H}_{I}(x) = \mathcal H_1(x) + \mathcal H_2(x) }
donde se han definido los hamiltonianos
\[ \mathcal H_1(x) \equiv \lambda vS(x)S(x) h(x) \ \ \ \ \ \ \ \ \ \ \mathcal H_2(x) \equiv -\frac{m_Z^2}{v} Z(x)Z(x) h(x)   \]
Puesto que ninguno de los tres hamiltonianos exhibe acople entre las tres partículas simultáneamente, se concluye que el primer orden de la matriz $S$ no contribuye al proceso, tan solo lo hace el segundo orden. Esto es equivalente a decir que el proceso es de dos vértices, tal como se ha mostrado en la sección anterior.

\section{Matriz S}
De la discusión anterior, se procede entonces a calcular el segundo orden de la matriz $S$
\[ S^{(2)} = \frac{(-i)^2}{2!}\int dx_1^4 \int dx_2^4 \mathcal{T}\cor{ :\mathcal H_I(x_1):\ :\mathcal H_I(x_2): } \]
expandiendo el argumento del operador de ordenamiento espacio-temporal se obtiene
\[ :\mathcal H_I(x_1):\ :\mathcal H_I (x_2): \ = \]
\[:\mathcal H_1 (x_1):\ :\mathcal H_1 (x_2): + :\mathcal H_1 (x_1):\ :\mathcal H_2 (x_2): + :\mathcal H_2 (x_1):\ :\mathcal H_1(x_2): + :\mathcal H_2 (x_1):\ :\mathcal H_2 (x_2): \]
de la figura (\ref{fig:fig01}) y de la definición de cada hamiltoniano es claro que solo el segundo término contribuye al proceso deseado, por tanto
\[S^{(2)} = \frac{(-i)^2}{2!}\int dx_1^4 \int dx_2^4 \mathcal{T}\cor{ :\mathcal H_1(x_1):\ :\mathcal H_2(x_2): } \]
introduciendo la forma explícita de $\mathcal H_1 y \mathcal H_2$
\eq{6}
{ \bcontraction{S^{(2)} = -\frac{(-i)^2}{2!}\lambda m_Z^2\int dx_1^4 \int dx_2^4 \mathcal{T}[:[S^2\ }{h}{](x_1):\ :[Z^2}{h}
S^{(2)} = -\frac{(-i)^2}{2!}\lambda m_Z^2\int dx_1^4 \int dx_2^4 \mathcal{T}\cor{ :[S^2h](x_1):\ :[Z^2h](x_2): } }
Donde la contracción de Wick se hace sobre el higgs puesto que es la partícula mediadora del proceso.

Introduciendo los operadores de creación y aniquilación, 
\[ Z = Z^+ + Z^- \ \ \ \ \ \ \ \ \ \ \ \ \ \ \ \ S = S^+ + S^-\]
y el producto ordenado del integrando de la matriz $S^{(2)}$ queda
\[ \bcontraction{ \mathcal{T}[:[S^2\ }{h}{](x_1):\ :[Z^2}{h}  \bcontraction{\mathcal{T}\cor{ :[S^2h](x_1):\ :[Z^2h](x_2): } = }{h}{(x_1)}{h}    \mathcal{T}\cor{ :[S^2h](x_1):\ :[Z^2h](x_2): } = h(x_1)h(x_2)Z_1^-(x_2)Z_2^-(x_2)S_1^+(x_1)S_2^+(x_1) \]
donde los subíndices en los operadores de aniquilación y creación indican la partícula $1$ o $2$.

\newpage

Teniendo en cuenta que la contracción de Wick del anterior campo escalar es proporcional al propagador
\[ \bcontraction{}{h}{(x_1)}{h}  h(x_1)h(x_2) = i\Delta_F(x_1-x_2)  \]
La matriz $S^{(2)}$ queda finalmente
\eq{7}
{S^{(2)} = -\frac{(-i)^2}{2!}\lambda m_Z^2\int dx_1^4 \int dx_2^4 i\Delta_F(x_1-x_2)Z_1^-(x_2)Z_2^-(x_2)S_1^+(x_1)S_2^+(x_1) }
Calculando el elemento de matriz asociado al proceso
\[ S^{(2)}_{fi} = \bra f|S^{(2)}|i\ket = \bra Z^0(p_1') Z^0(p_2') | S^{(2)} | S(p_1) S(p_2) \ket \]
con las siguiente relaciones
\[ | S(p_1) \ket = \sqrt{\frac{2\pi^3}{V}}\hat a^\dag _{p_1}|0\ket \ \ \ \ \ \ \ \ | S(p_2) \ket  = \sqrt{\frac{2\pi^3}{V}}\hat a^\dag _{p_2}|0\ket \]
\[ | Z^0(p_1') \ket = \sqrt{\frac{2\pi^3}{V}}\hat b^\dag _{p_1'}|0\ket \ \ \ \ \ \ \ \ | Z^0(p_2') \ket = \sqrt{\frac{2\pi^3}{V}}\hat b^\dag _{p_2'}|0\ket \]
donde $\hat a_{p}$ es un operador que crea una partícula $S$ con momentum $p$, y $\hat b_{p'}$ crea un bosón $Z^0$ con momentum $p'$.

El elemento de matriz queda entonces
\[ S^{(2)}_{fi} =-\frac{(-i)^2}{2!}\lambda m_Z^2\int dx_1^4 \int dx_2^4 i\Delta_F(x_1-x_2)\times\]
\[\bra Z^0(p_1') Z^0(p_2') |Z_1^-(x_2)Z_2^-(x_2)S_1^+(x_1)S_2^+(x_1)| S(p_1) S(p_2) \ket \]
y teniendo en cuenta que 
\[ S^+(x) = \sqrt{\frac{V}{2\pi^3}}\int \frac{d^4k}{\sqrt{2E_k V}}e^{-ik\cdot x} \hat a_{k} \]
se obtiene
\eq{10}
{ S_1^+(x_1)S_2^+(x_1)| S(p_1) S(p_2) \ket = \int \frac{d^4k'}{\sqrt{2E_{k'} V}} \int \frac{d^4k}{\sqrt{2E_k V}}e^{-ik'\cdot x_1}e^{-ik\cdot x_1} \hat a_{k} \hat a_{k'}\hat a^\dag _{p_1}\hat a^\dag _{p_2}|0\ket }
Para simplificar esto se calcula el siguiente conmutador
\[ [ \hat a_{k}\hat a_{k'},\hat a^\dag _{p_1}\hat a^\dag _{p_2} ] \]
expandiendo según la identidad
\[ [AB,CD] = AC[B,D] + A[B,C]D + C[A,D]B + [A,C]DB \]
y la relación de conumtación bosónica
\[ [\hat a_{p}, \hat a^\dag _{p'}] = \delta(p-p') \]
se llega a 
\eq{8}
{ [ \hat a_{k}\hat a_{k'},\hat a^\dag _{p_1}\hat a^\dag _{p_2} ] = \hat a_{k}\hat a^\dag _{p_1}\delta( k' - p_2 ) + \hat a_{k} \hat a^\dag _{p_2} \delta( k' - p_1 ) + \hat a^\dag _{p_1}\hat a_{k'}\delta( k-p_2 ) + \hat a^\dag _{p_2}\hat a_{k'}\delta(k - p_1) }
Recordando que un operador de aniquilación actuando sobre el estado de vacío da un efecto nulo, se concluye que
\[ [ \hat a_{k}\hat a_{k'},\hat a^\dag _{p_1}\hat a^\dag _{p_2} ]|0\ket = \hat a_{k}\hat a_{k'}\hat a^\dag _{p_1}\hat a^\dag _{p_2} |0\ket \]
usando (\ref{eq8})
\[ \hat a_{k}\hat a_{k'}\hat a^\dag _{p_1}\hat a^\dag _{p_2} |0\ket = \delta( k' - p_2 )\hat a_{k}\hat a^\dag _{p_1}|0\ket + \delta( k' - p_1 )\hat a_{k} \hat a^\dag _{p_2} |0\ket \]
y aplicando los operadores sobre el estado del vacío, se llega a
\eq{9}
{ \hat a_{k}\hat a_{k'}\hat a^\dag _{p_1}\hat a^\dag _{p_2} |0\ket = \delta( k' - p_2 )\delta(k-p_1)|0\ket + \delta( k' - p_1 )\delta(k-p_2) |0\ket }

\newpage

Con (\ref{eq9}) es fácil evaluar (\ref{eq10}), quedando
\eq{11}
{ S_1^+(x_1)S_2^+(x_1)| S(p_1) S(p_2) \ket =  \frac{1}{\sqrt{2E_{p_2} V}} \frac{1}{\sqrt{2E_{p_1} V}}2e^{-ip_2\cdot x_1}e^{-ip_1\cdot x_1} |0\ket }
Asumiendo que el bosón $Z^0$ está polarizado puede tomarse como un escalar, por tanto el resultado anterior es igualmente valido para este
\eq{11}
{ \bra Z^0(p'_1) Z^0(p'_2) |Z_1^-(x_2)Z_2^-(x_2) = \bra 0 | \frac{1}{\sqrt{2E_{p'_2} V}} \frac{1}{\sqrt{2E_{p'_1} V}}2e^{ip'_2\cdot x_2}e^{ip'_1\cdot x_2} }

\

El elemento de matriz de $S^{(2)}$ queda
\[ S^{(2)}_{fi} =-4\frac{(-i)^2}{2!}\lambda m_Z^2\cor{\frac{1}{\sqrt{2E_{p'_2} V}} \frac{1}{\sqrt{2E_{p'_1} V}}\frac{1}{\sqrt{2E_{p_2} V}} \frac{1}{\sqrt{2E_{p_1} V}}}\times \] \[
\int dx_1^4 \int dx_2^4 i\Delta_F(x_1-x_2) e^{-ip_2\cdot x_1}e^{-ip_1\cdot x_1}e^{ip'_2\cdot x_2}e^{ip'_1\cdot x_2} \]
Introduciendo la transformada de Fourier inversa del propagador
\[ \Delta_F(x)  = \int\frac{d^4q}{(2\pi)^4}\Delta_F(q)e^{iq \cdot x}  \]
\[ S^{(2)}_{fi} =-4\frac{(-i)^2}{2!}\lambda m_Z^2\cor{\frac{1}{\sqrt{2E_{p'_2} V}} \frac{1}{\sqrt{2E_{p'_1} V}}\frac{1}{\sqrt{2E_{p_2} V}} \frac{1}{\sqrt{2E_{p_1} V}}}\times \] \[
\int dx_1^4 \int dx_2^4 \int\frac{d^4q}{(2\pi)^4}i\Delta_F(q)e^{iq \cdot (x_1-x_2)} e^{-ip_2\cdot x_1}e^{-ip_1\cdot x_1}e^{ip'_2\cdot x_2}e^{ip'_1\cdot x_2} \]
Recordando la transformada de la delta de dirac
\[ \delta^4( p-p' ) = \frac{1}{(2\pi)^4}\int d^4 x e^{-i(p-p')\cdot x} \]
y usando 
\[ \delta(q-x)\delta(q-y) = \delta( q-x )\delta(x-y) \]
se obtiene finalmente
\[ S^{(2)}_{fi} =-4\frac{(-i)^2}{2!}\lambda m_Z^2\cor{\frac{1}{\sqrt{2E_{p'_2} V}} \frac{1}{\sqrt{2E_{p'_1} V}}\frac{1}{\sqrt{2E_{p_2} V}} \frac{1}{\sqrt{2E_{p_1} V}}}\times \] 
\eq{13}
{ (2\pi)^4\delta^4(p_1+p_2-p'_1-p'_2)i\Delta_F( p_1 + p_2 ) }
De la relación definitoria de la matriz $\mathcal{M}_{fi}$ \cite{lah}
\eq{14}
{ S_{fi} = \delta_{fi} + i(2\pi)^4\delta^4\pr{ \Sigma_i p_i - \Sigma_f p_f }\prod_i\frac{1}{\sqrt{2E_i V}}\prod_f\frac{1}{\sqrt{2E_f V}}\mathcal{M}_{fi} }
y comparando con (\ref{eq13}) se obtiene que 
\eq{15}
{\mathcal{M}_{fi} = 2\lambda m_Z^2 \Delta_F( p_1 + p_2 ) }
invocando la conservación del cuadrimomentum, o de la misma expresión (\ref{eq13}), esto último puede reexpresarse como
\eq{16}
{\mathcal{M}_{fi} = 2\lambda m_Z^2 \Delta_F( p'_1 + p'_2 ) }
aunque como el estado inicial es el conocido, es más conveniente usar (\ref{eq15}).

\newpage

Usando la definición del propagador escalar en el espacio de momentos dada en \cite{lah}
\[ \Delta_F(p) = \frac{1}{(p)^2 - m_h^2} \]
donde $E$ es la energía inicial del proceso y $p_0$ la masa en reposo de las partículas asociadas, y haciendo $\epsilon = 0$ se obtiene
\eq{17}
{ \mathcal{M}_{fi} = \frac{2\lambda m_Z^2}{(E_1+E_2)^2 - m_h^2} }
donde $E_1$ y $E_2$ es la energía de cada una de las partículas de materia oscura.


\section{Cross section}

Usando la expresión general dada en \cite{profe} para la sección eficaz diferencial
\eq{18}
{ \frac{d\sigma}{d\Omega} = \frac{1}{64\pi^2 s}\lla{ \frac{[ s - (m_1' + m_2')^2 ]}{[ s - (m_1 + m_2)^2 ]} \frac{[ s - (m_1' - m_2')^2 ]}{[ s - (m_1 - m_2)^2 ]} }^{1/2}|\mathcal{M}_{fi}|^2 }
donde se ha definido
\[ \sqrt s = E_1+E_2 \]
Puesto que las partículas iniciales son iguales y las finales también, se hace $m_1 = m_2 = m_0$ y $m_1' = m_2'=m_Z$, luego
\[ \frac{d\sigma}{d\Omega} = \frac{1}{64\pi^2 s}\lla{ \frac{[ s - 4m_Z^2 ]}{[ s - 4m_0^2 ]} }^{1/2}|\mathcal{M}_{fi}|^2  \]
e introduciendo (\ref{eq17})
\eq{19}
{ \frac{d\sigma}{d\Omega} = \frac{1}{64\pi^2 s}\lla{ \frac{[ s - 4m_Z^2 ]}{[ s - 4m_0^2 ]} }^{1/2} \pr{\frac{2\lambda m_Z^2}{m_h^2 - s}}^2   }
puesto que la expresión anterior es completamente isotrópica, la sección eficaz integral es
\eq{19}
{ \sigma = \frac{1}{16\pi s}\lla{ \frac{[ s - 4m_Z^2 ]}{[ s - 4m_0^2 ]} }^{1/2} \pr{\frac{2\lambda m_Z^2}{m_h^2 - s}}^2   }

En especial tomando los siguiente valores de los parámetros
\begin{table}[htbp]
\centering
\begin{tabular}{|l|r|} \hline
\textbf{Parámetro} & \textbf{Valor} \\ \hline
Masa de $S$ ($m_0$) & $50\ \mbox{GeV}$ \\ \hline
Masa de $Z^0$ ($m_Z$) & $91.2\ \mbox{GeV}$ \\ \hline
Masa de $h$ ($m_h$) & $150\ \mbox{GeV}$ \\ \hline
Parámetro de acople ($\lambda$) & 0.1 \\ \hline
Energía inicial ($E_1 = E_2$) & $500\ \mbox{GeV}$ \\ \hline
\end{tabular}
%\caption{Parámetros usados.}
\end{table}

se obtiene la siguiente sección eficaz
\eq{20}
{ \sigma = 4.039\times 10^{-10} \mbox{GeV}^{-2} = 0.157\ \mu\mbox{barns} }

\section{CalcHEP}

Calculando en CalcHEP, usando los parámetros de la tabla anterior y el modelo lagrangiano discutido inicialmente, se obtiene

\begin{figure}[htbp]
	\centering
	\includegraphics[width=0.5\textwidth]{Sebastian_fig02.png}
	\caption{CalcHEP Calculos.}
	\label{fig:fig01}
\end{figure}

\section{Copyright}
\includegraphics[scale=0.5]{cc} Creative Commons Attribution-Share Alike 3.0 United States License.


%REFERENCIAS ================================================================================

\begin{thebibliography}{}
\bibitem[1]{Goud} Goudelis A., Mambrini Y., Yaguna C., \textit{Antimatter signals of singlet scalar dark matter}. 2009. ArXiv:0909.2799v2 [hep-ph]
\bibitem[2]{profe} Restrepo D., \textit{Notas del curso de mecánica cuántica relativista}. 2011.
\bibitem[3]{lah} Lahiri A., Palash B. P. \textit{A first book of Quantum Field Theory 2ed}. 2005. Alpha Science International Ltd.
\end{thebibliography}


%%% Local Variables: 
%%% mode: latex
%%% TeX-master: "qft_samples"
%%% End: 

% Singlet Dark matter I: Juan David 1011.1411
\include{sdm3}
% Inert Dark matter II: Jehison 1011.1411
%\include{idm2}
% Radiative see saw I: Alvaro 0808.3340
\chapter{Calculation sample}

Author: Alvaro Ramirez

Descripcion del proceso: se tiene un proceso de aniquilacion directa de dos campos fermionicos(neutrinos derechos) a dos campos fermionicos(neutrinos izquierdos) $NN\to \nu\nu$.

\section{Lagrangiano}

Lagrangiano mas importante:

\begin {equation}
	\mathcal{L}_{Y}=f_{ij}({\phi}^{-}{\nu}_{i}+{\phi}^{0}{l}_{i}){l}_{i}^c+{h}_{ij}(\bar{{\nu}_{i}}{\eta}^0-{l}_{j}{\nu}^{\dagger}){N}_{j}+h.c
\end{equation}

Asi la parte relevante del lagrangino anterior es:

\begin {equation}
\mathcal{L}_{Y}={h}_{ij}(\bar{{\nu}_{i}}{\eta}^0-{l}_{j}{\nu}^{\dagger}){N}_{j}+hc
\end{equation}
\begin {equation}
\mathcal{L}_{Y}=h\bar{{\nu}_{3}}{\eta}^0{N}+h.c
\end{equation}
\begin {equation}
\mathcal{L}_{Y}=h\bar{{\nu}_{3}}{\eta}^0{N}+h(\nu_{3}{\eta}^{0}N)^{\dagger}
\end{equation}

\begin {equation}
\mathcal{L}_{Y}=h\bar{{\nu}_{3}}{\eta}^0{N}_{j}+h\bar{N}\eta^{0}\nu_{3}
\end{equation}


\section{S-matrix}

Calculo a segundo orden

\begin{gather}
\begin{split}
&<\nu(p'_{1})\nu(p'_{2})|(\bar{\nu}^{\alpha}_{+}(x_1)+\bar{\nu}^{\alpha}_{-}(x_1))({N}^{\alpha}_{+}(x_1)+\\&{N}^{\alpha}_{-}(x_1))(\bar{\nu}^{\beta}_{+}(x_2)+\bar{\nu}^{\beta}_{-}(x_2))(N^{\alpha}_{+}(x_2)+N^{\alpha}_{-}(x_2))|N(P_1)N(P_2)>\\
&=<\nu(p'_{1})\nu(p'_{2})|(\bar{\nu}^{\alpha}_{-}(x_1)({N}^{\alpha}_{+}(x_1)+{N}^{\alpha}_{-}(x_1)))(\bar{\nu}^{\beta}_{+}(x_2)+\\&\bar{\nu}^{\beta}_{-}(x_2)){N}^{\beta}_+(x_2)|N(P_1)N(P_2)>\\
&=<\nu(p'_{1})\nu(p'_{2})|\bar{\nu}^{\alpha}_{-}(x_1)({N}^{\alpha}_{+}(x_1)\bar{\nu}^{\beta}_{+}(x_2)+\\&{N}^{\alpha}_{+}(x_1)\bar{\nu}^{\beta}_{-}(x_2)+{N}^{\alpha}_{-}(x_1)\bar{\nu}^{\beta}_{+}(x_2)+{N}^{\alpha}_{-}(x_1)\bar{\nu}^{\beta}_{-}(x_2)){N}^{\beta}_{+}(x_2)|N(P_1)N(P_2)>\\
&=<\nu(p'_{1})\nu(p'_{2})|-\bar{\nu}^{\alpha}_{-}(x_1)(-\bar{\nu}^{\beta}_{+}(x_2){N}^{\alpha}_{+}(x_1)-\bar{\nu}^{\beta}_{-}(x_2){N}^{\alpha}_{+}(x_1)-\\&\bar{\nu}^{\beta}_{+}(x_2){N}^{\alpha}_{-}(x_1)-\bar{\nu}^{\beta}_{-}(x_2){N}^{\alpha}_{-}(x_1)){N}^{\beta}_{+}(x_2)|N(P_1)N(P_2)>\\
&=-<\nu(p'_{1})\nu(p'_{2})|\bar{\nu}^{\alpha}_{-}(x_1)\bar{\nu}^{\beta}_{-}(x_2){N}^{\alpha}_{+}(x_1){N}^{\beta}_{+}(x_2))|N(P_1)N(P_2)>\\
&=-\eta^{0}(x_1)\eta^{0}(x_2)<{\nu}^{\alpha}(p'_{1}){\nu}^{\beta}(p'_{2})|\bar{\nu}^{\alpha}_{-}(x_{1})\bar{\nu}^{\beta}_{-}(x_{2}){N}^{\alpha}_{+}(x_1){N}^{\beta}_{+}(x_2))|N(P_1)N(P_2)>\\
\end{split}
\end{gather}

\begin {equation}
|N(P_1)N(P_2)>=\frac{(2\pi)^{3}}{V}f^{\dagger}(p_2)f^{\dagger}(p_1)|0>
\end{equation}

\begin{gather}
\begin{split}
{N}^{\alpha}_{+}(x_1){N}^{\beta}(x_2)|N(\textbf{P}_1)N(\textbf{P}_2)>&= \int \frac{{d}^{3}k}{\sqrt{2E_{k}V}}\int \frac{{d}^{3}k'}{\sqrt{2E_{k'}V}}\nu^{\alpha}(k)\nu^{\beta}(k')e^{-ik-x_{1}}e^{-ik'-x_{2}}f(\textbf{k})\\&f(\textbf{k'})f^{\dagger}(\textbf{P}_2)f^{\dagger}(\textbf{P}_1)|0>
\end{split}
\end{gather}

haciendo uso de la siguiente identidad

\begin {equation}
[AB,CD]_{+}=A[B,C]_{+}D-[A,C]_{+}BD+CA[B,D]_{+}-C[A,D]_{+}B
\end {equation}

lo cual se cumple para cualesquier cuatro operadores A,B,C,D y utilizando las relaciones de anticonmutacion:

\begin {equation}
[f_s(\textbf{P}),f^{\dagger}_{s'}(\textbf{P'})]_{+}=[\hat{f}_{s}(\textbf{P}),\hat{f}^{\dagger}_{s'}(\textbf{P'})]_{+}=\delta_{ss'}\delta^{3}(\textbf{P}-\textbf{P'})
\end {equation}
se tiene

\begin{gather}
\begin{split}
[f(\textbf{K})f(\textbf{K'},f^{\dagger}(\textbf{P}_2)f^{\dagger}(\textbf{P}_1)]_+ &= f(\textbf{K})[f(\textbf{K'}),f^{\dagger}(\textbf{P}_2)]_+f^{\dagger}(\textbf{P}_1)-[f(\textbf{K}),f^{\dagger}(\textbf{P}_2)]_+f(\textbf{K'}f^{\dagger}(\textbf{P}_1)+\\&f^{\dagger}(\textbf{P}_2)f(\textbf{K})[f(\textbf{K'},f^{\dagger}(\textbf{P}_1)]_+-f^{\dagger}(\textbf{P}_2)[f(\textbf{K}),f^{\dagger}(\textbf{P}_1)]_+f(\textbf{K'})\\&=f(\textbf{K})\delta^{3}(\textbf{K'}-\textbf{P}_2)f^{\dagger}(\textbf{P}_1)-\delta^{3}(\textbf{K}-\textbf{P}_2)f(\textbf{K})f^{\dagger}(\textbf{P}_1)\\&+f^{\dagger}(\textbf{P}_2)f(\textbf{K})\delta^{3}(\textbf{K'}-\textbf{P}_1)-f^{\dagger}(\textbf{P}_2)\delta^{3}(\textbf{K}-\textbf{P}_1)f(\textbf{K'}
\end{split}
\end{gather}

\begin{gather}\begin{split}
[f(\textbf{K})f(\textbf{K'}),f^{\dagger}(\textbf{P}_2)f^{\dagger}(\textbf{P}_1)]_+&=\delta^{3}(\textbf{K'}-\textbf{P}_2)f(\textbf{K})f^{\dagger}(\textbf{P}_1)-\delta^{3}(\textbf{K}-\textbf{P}_2)f(\textbf{K'}f^{\dagger}(\textbf{P}_1)+\\&\delta^{3}(\textbf{K'}-\textbf{P}_1)f^{\dagger}(\textbf{P}_2)f(\textbf{K})-\delta^{3}(\textbf{K}-\textbf{P}_1)f^{\dagger}(\textbf{P}_2)f(\textbf{K'})
\end{split}
\end{gather}

Aplicando nuevamente las relaciones de anticonmutaciòn para $f(\textbf{P})$ y $f^{\dagger}(\textbf{P'})$ se tiene:

\begin{gather}
\begin{split}
[f(\textbf{K}),f^{\dagger}(\textbf{P}_1)]=\delta^{3}(\textbf{K}-\textbf{P}_1)\\
f(\textbf{K})f^{\dagger}(\textbf{P}_1)+f^{\dagger}(\textbf{P}_1)f(\textbf{K})=\delta^{3}(\textbf{K}-\textbf{P}_1)\\
f(\textbf{K})f^{\dagger}(\textbf{P}_1)=\delta^{3}(\textbf{K}-\textbf{P}_1)-f^{\dagger}(\textbf{P}_1)f(\textbf{K})\\
f(\textbf{K})f^{\dagger}(\textbf{P}_1)=\delta^{3}(\textbf{K}-\textbf{P}_1)
\end{split}
\end{gather}

ya que $f(\textbf{K})$ aniquila al vacio, analogamente para
\begin{gather}
\begin{split}
[f(\textbf{K'},f^{\dagger}(\textbf{P}_1)]=\delta^{3}(\textbf{K'}-\textbf{P}_1)\\
f(\textbf{K'}f^{\dagger}(\textbf{P}_1)+f^{\dagger}(\textbf{P}_1)f(\textbf{K'}=\delta^{3}(\textbf{K'}-\textbf{P}_1)\\
f(\textbf{K'}f^{\dagger}(\textbf{P}_1)=\delta^{3}(\textbf{K'}-\textbf{P}_1)-f^{\dagger}(\textbf{P}_1)f(\textbf{K'}
\end{split}
\end{gather}
ya que $f(\textbf{K'}$ aniquila al vacio

reemplazando los 2 resultados en (17)
\begin{gather}
\begin{split}
[f(\textbf{K})f(\textbf{K'},f^{\dagger}(\textbf{P}_2)f^{\dagger}(\textbf{P}_1)]=\delta^{3}(\textbf{K'}-\textbf{P}_2)\delta^{3}(\textbf{K}-\textbf{P}_1)-\delta^{3}(\textbf{K}-\textbf{P}_2)\delta^{3}(\textbf{K'}-\textbf{P}_1)
\end{split}
\end{gather}

por otro lado
\begin{gather}
\begin{split}
[f(\textbf{K})f(\textbf{K'}),f^{\dagger}(\textbf{P}_2)f^{\dagger}(\textbf{P}_1)]_{+}=f(\textbf{K})f(\textbf{K'})f^{\dagger}(\textbf{P}_2)f^{\dagger}(\textbf{P}_1)+f^{\dagger}(\textbf{P}_2)f^{\dagger}(\textbf{P}_1)f(\textbf{K})f(\textbf{K'})
\end{split}
\end{gather}

notese que el segundo termino del lado derecho aniquila el vacio luego,

\begin{gather}
\begin{split}
[f(\textbf{K})f(\textbf{K'}),f^{\dagger}(\textbf{P}_2)f^{\dagger}(\textbf{P}_1)]_{+}=f(\textbf{K})f(\textbf{K'})f^{\dagger}(\textbf{P}_2)f^{\dagger}(\textbf{P}_1)
\end{split}
\end{gather}

igualando 21 y 22

\begin{gather}
\begin{split}
f(\textbf{K})f(\textbf{K'})f^{\dagger}(\textbf{P}_2)f^{\dagger}(\textbf{P}_1)|0>=(\delta^{3}(\textbf{K'}-\textbf{P}_2)\delta^{3}(\textbf{K}-\textbf{P}_1)-\delta^{3}(\textbf{K}-\textbf{P}_2)\delta^{3}(\textbf{K'}-\textbf{P}_1))|0>
\end{split} 
\end{gather}

reemplazando 8 en 13

\begin{gather}
\begin{split}
{N}^{\alpha}_{+}(x_1){N}^{\beta}_{+}(x_2)|{N}(\textbf{P}_1){N}(\textbf{P}_2)>&=\int \frac{{d}^{3}k}{\sqrt{2E_{k}V}}\int \frac{{d}^{3}k'}{\sqrt{2E_{k'}V}}\nu^{\alpha}(k)\nu^{\beta}(k')e^{-ik-x_{1}}e^{-ik'-x_{2}}\\&(\delta^{3}(\textbf{K'}-\textbf{P}_2)\delta^{3}(\textbf{K}-\textbf{P}_1)-\delta^{3}(\textbf{K}-\textbf{P}_2)\delta^{3}(\textbf{K'}-\textbf{P}_1))
\end{split}
\end{gather}

despues de integrar 

\begin{gather}
\begin{split}
{N}^{\alpha}_{+}(x_1){N}^{\beta}_{+}(x_2)|{N}(\textbf{P}_1){N}(\textbf{P}_2)>&=[\frac{1}{\sqrt{2E_{1}V}}\frac{1}{\sqrt{2E_{2}V}}\nu^{\alpha}({P}_1)\nu^{\beta}({P}_{2})e^{-i{P}_{1}-x_{1}}e^{-iP_{2}-x_{2}}-\\& \frac{1}{\sqrt{2E_{2}V}}\frac{1}{\sqrt{2E_{1}V}}\nu^{\alpha}({P}_2)\nu^{\beta}({P}_{1})e^{-i{P}_{2}-x_{1}}e^{-iP_{1}-x_{2}}])|0>
\end{split}
\end{gather}

en donde $E_{i}=\sqrt{\vec{P_i^2}+m^2}$ con $i=1,2$

\begin{gather}
\begin{split}
{N}^{\alpha}_{+}(x_1){N}^{\beta}_{+}(x_2)|{N}(\textbf{P}_1){N}(\textbf{P}_2)>&=\frac{1}{\sqrt{2E_{1}V}}\frac{1}{\sqrt{2E_{2}V}}[\nu^{\alpha}({P}_1)\nu^{\beta}({P}_2)e^{-i{P}_{1}-x_{1}}e^{-iP_{2}-x_{2}}-\\&\nu^{\alpha}({P}_2)\nu^{\beta}({P}_1)e^{-i{P}_{2}-x_{1}}e^{-iP_{1}-x_{2}}]|0>
\end{split}
\end{gather}

ahora se considerara el estado final

\begin{gather}
\begin{split}
<\nu({P'}_1)\nu({P'}_2)|\bar{\nu}^{\alpha}_{-}(x_1)nu(\textbf{P'}_2)>|\bar{\nu}^{\beta}_{-}(x_2)&=\frac{1}{\sqrt{2E'_{1}V}}\frac{1}{\sqrt{2E'_{2}V}}<0|[\bar{U}^{\alpha}(P'_1)\bar{U}^{\beta}(P'_2)]e^{i{P'}_{1}x_{1}}e^{i{P'}_{2}x_{2}}\\&-\bar{U}^{\alpha}(P'_2)\bar{U}^{\beta}(P'_1)]e^{i{P'}_{2}x_{1}}e^{i{P'}_{1}x_{2}}
\end{split}
\end{gather}

reemplazando (9) y (10) en ()

\begin{gather}
\begin{split}
\mathcal{S}^{2}_{fi}&=-\frac{(-ih)^2}{2!}\int{d^4x_1}\int{d^4x_2}\Delta_F(x_1-x_2)<\bar{\nu}(\textbf{P'}_1)\bar{\nu}(\textbf{P'}_2)|\bar{\nu}^\alpha_-(x_1)\bar{\nu}^\beta_-(x_2){N}^{\alpha}_{+}(x_1){N}^{\beta}_{+}(x_2)|N(P_1)N(P_2)>\\\mathcal{S}^{2}_{fi}&=-\frac{(-ih)^2}{2!}\int{d^4x_1}\int{d^4x_2}\int{\frac{d^4x_q}{{2\pi}^4}}i\Delta_{f}(q)e^{i{q}(x_{1}-x_{2})}[\bar{U}^{\alpha}(P'_1)\bar{U}^{\beta}(P'_2)]e^{i{P'}_{1}x_{1}}e^{i{P'}_{2}x_{2}}\\&-\bar{U}^{\alpha}(P'_2)\bar{U}^{\beta}(P'_1)]e^{i{P'}_{2}x_{1}}e^{i{P'}_{1}x_{2}}][\nu^{\alpha}({P}_1)\nu^{\beta}({P}_2)e^{-i{P}_{1}-x_{1}}e^{-iP_{2}-x_{2}}-\\&\nu^{\alpha}({P}_2)\nu^{\beta}({P}_1)e^{-i{P}_{2}-x_{1}}e^{-iP_{1}-x_{2}}][\frac{1}{\sqrt{2E_{1}V}}\frac{1}{\sqrt{2E_{2}V}}\frac{1}{\sqrt{2E'_{1}V}}\frac{1}{\sqrt{2E'_{2}V}}]
\end{split}
\end{gather}

La amplitud de Feynman a segundo orden en la expansion de la matriz S está dada por


\begin{gather}
\begin{split}
\mathcal{M}^{(2)}_{fi}&=(ih)^2[\bar{U}^{\alpha}({P'}_2)\bar{U}^{\beta}({P'}_1)i\Delta_f(P_1-P'_2)-\bar{U}^{\alpha}({P'}_1)\bar{U}^{\beta}({P'}_2)]i\Delta_f(P_1-P'_1)]{U}^{\alpha}({P}_1){U}^{\alpha}({P}_2)
\end{split}
\end{gather}

\begin{gather}
\begin{split}
{|\mathcal{M}_{fi}|}^2&=[\bar{U}^{\alpha}({P'}_2)\bar{U}^{\beta}({P'}1){U}^{\alpha}({P}_1){U}^{\beta}({P}_2)-\bar{U}^{\alpha}({P'}_1)\bar{U}^{\beta}({P'}_2){U}^{\alpha}({P}_1){U}^{\beta}({P}_2)][\bar{U}^{\alpha}({P'}_2)\bar{U}^{\beta}({P'}1)]{U}^{\alpha}({P}_1){U}^{\beta}({P}_2)-\\&\bar{U}^{\alpha}({P'}_1)\bar{U}^{\beta}({P'}_2){U}^{\alpha}({P}_1){U}^{\beta}({P}_2)]^\dagger
\end{split}
\end{gather}


permutando los centrales

\begin{gather}
\begin{split}
{|\mathcal{M}_{fi}|}^2&=[\bar{U}^{\alpha}({P'}_2){U}^{\alpha}({P}_1)\bar{U}^{\beta}({P'}_1){U}^{\beta}({P}_2)-\bar{U}^{\alpha}({P'}_1){U}^{\alpha}({P}_1)\bar{U}^{\beta}({P'}_2){U}^{\beta}({P}_2)][\bar{U}^{\alpha}({P'}_2){U}^{\alpha}({P}_1)\bar{U}^{\beta}({P'}_1){U}^{\beta}({P}_2)-\\&\bar{U}^{\alpha}({P'}_1){U}^{\alpha}({P}_1)\bar{U}^{\beta}({P'}_2){U}^{\beta}({P}_2]^\dagger
\end{split}
\end{gather}

haciendo el cambio de componentes a vectores

\begin{gather}
\begin{split}
{|\mathcal{M}_{fi}|}^2&=[\bar{U}({P'}_2){U}({P}_1)\bar{U}({P'}_1){U}({P}_2)-\bar{U}({P'}_1){U}({P}_1)\bar{U}({P'}_2){U}({P}_2)][\bar{U}({P'}_2){U}({P}_1)\bar{U}({P'}_1){U}({P}_2)-\\&\bar{U}({P'}_1){U}({P}_1)\bar{U}({P'}_2){U}({P}_2]^\dagger
\end{split}
\end{gather}

hallando el segundo termino del hermitico conjugado
\begin{gather}
\begin{split}
&[\bar{U}({P'}_2){U}({P}_1)\bar{U}({P'}_1){U}({P}_2)]^{\dagger}-[\bar{U}({P'}_1){U}({P}_1)\bar{U}({P'}_2){U}({P}_2)]^{\dagger}\\
&=[\bar{U}({P'}_1){U}({P}_2)]^{\dagger}[\bar{U}({P'}_2){U}({P}_1)]^{\dagger}-[\bar{U}({P'}_2){U}({P}_2)]^{\dagger}[\bar{U}({P'}_1){U}({P}_1)]^{\dagger}
\end{split}
\end{gather}

haciendo lo mismo para la anterior expresion

\begin{gather}
\begin{split}
=[\bar{U}({P}_2){U}({P'}_1)][\bar{U}({P}_1){U}({P'}_2)]-[\bar{U}({P}_2){U}({P'}_2)][\bar{U}({P}_1){U}({P'}_1)]
\end{split}
\end{gather}

ahora expresando en terminos de componentes

\begin{gather}
\begin{split}
&=[\bar{U}^{\alpha}({P}_2){U}^{\alpha}({P'}_1)\bar{U}^{\beta}({P}_1){U}^{\beta}({P'}_2)-\bar{U}^{\alpha}({P}_2){U}^{\alpha}({P'}_2)\bar{U}^{\beta}({P}_1){U}^{\beta}({P'}_1)]
\end{split}
\end{gather}


\begin{gather}
\begin{split}
{|\mathcal{M}_{fi}|}^2&=[\bar{U}^{\alpha}({P'}_2){U}^{\alpha}({P}_1)\bar{U}^{\beta}({P'}_1){U}^{\beta}({P}_2)-\bar{U}^{\alpha}({P'}_1){U}^{\alpha}({P}_1)\bar{U}^{\beta}({P'}_2){U}^{\beta}({P}_2)]\\&[\bar{U}^{\alpha}({P}_2){U}^{\alpha}({P'}_1)\bar{U}^{\beta}({P}_1){U}^{\beta}({P'}_2)-\bar{U}^{\alpha}({P}_2){U}^{\alpha}({P'}_2)\bar{U}^{\beta}({P}_1){U}^{\beta}({P'}_1)]
\end{split}
\end{gather}

\begin{gather}
\begin{split}
[{|\mathcal{M}_{fi}|}^2&=[\bar{U}^{\alpha}({P'}_2){U}^{\alpha}({P}_1)\bar{U}^{\beta}({P'}_1){U}^{\beta}({P}_2)][\bar{U}^{\alpha}({P}_2){U}^{\alpha}({P'}_1)\bar{U}^{\beta}({P}_1){U}^{\beta}({P'}_2)]-\\&[\bar{U}^{\alpha}({P'}_1){U}^{\alpha}({P}_1)\bar{U}^{\beta}({P'}_2){U}^{\beta}({P}_2)][\bar{U}^{\alpha}({P}_2){U}^{\alpha}({P'}_1)\bar{U}^{\beta}({P}_1){U}^{\beta}({P'}_2)]-\\&[\bar{U}^{\alpha}({P'}_2){U}^{\alpha}({P}_1)\bar{U}^{\beta}({P'}_1){U}^{\beta}({P}_2)][\bar{U}^{\alpha}({P}_2){U}^{\alpha}({P'}_2)\bar{U}^{\beta}({P}_1){U}^{\beta}({P'}_1)]+\\&[\bar{U}^{\alpha}({P'}_1){U}^{\alpha}({P}_1)\bar{U}^{\beta}({P'}_2){U}^{\beta}({P}_2)][\bar{U}^{\alpha}({P}_2){U}^{\alpha}({P'}_2)\bar{U}^{\beta}({P}_1){U}^{\beta}({P'}_1)]
\end{split}
\end{gather}

Efectuando la multiplicacion de los dos primero terminos
\begin{gather}
\begin{split}
&[\bar{U}^{\alpha}({P'}_2){U}^{\alpha}({P}_1)\bar{U}^{\beta}({P'}_1)\bar{U}^{\beta}({P}_1){U}^{\beta}({P}_2)][\bar{U}^{\alpha}({P}_2){U}^{\alpha}({P'}_1)\bar{U}^{\beta}({P}_1){U}^{\beta}({P'}_2)]\\&={U}^{\beta}({P'}_2)\bar{U}^{\alpha}({P'}_2){U}^{\alpha}({P}_1)\bar{U}^{\beta}({P}_1){U}^{\beta}({P}_2)\bar{U}^{\alpha}({P}_2)\bar{U}^{\beta}({P'}_1){U}^{\alpha}({P'}_1)
\end{split}
\end{gather}

\begin{gather}
\begin{split}
={U}^{\beta}({P'}_2)\bar{U}^{\alpha}({P'}_2){U}^{\alpha}({P}_1)\bar{U}^{\beta}({P}_1){U}^{\beta}({P}_2)\bar{U}^{\alpha}({P}_2){U}^{\alpha}({P'}_1)\bar{U}^{\beta}({P'}_1)
\end{split}
\end{gather}

\begin{gather}
\begin{split}
=&(P'_2+m'_2)_{\beta \alpha}(P'_1+m'_1)_{\alpha \beta}(P_2+m_2)_{\beta \alpha}(P_1+m_1)_{\alpha \beta}
\\&=Tr[(P'_2+m'_2)(P'_1+m'_1)]Tr[P_2+m_2)(P_1+m_1)]
\\&=Tr[(\gamma_{mu}P'_2{^\mu}+,m'_2)(\gamma_{nu}P'_1{^\nu}+,m'_1)]
\\&=4(P'_{2}.P'_{1}+m'_{1}m'{_2})
\end{split}
\end{gather}

de la misma forma

\begin{gather}
\begin{split}
&Tr[(P_2+m_2)(P_1+m_1)]Tr[P_2+m_2)(P_1+m_1)]
\\&=4(P_{2}.P_{1}+m_{1}m{_2})
\end{split}
\end{gather}

\begin{gather}
\begin{split}
{|\mathcal{M}_{fi}|}^2&=4({P'}_{2}.{P'}_{1}+{m'}_{1}{m'}_{2}).4(P_{2}.P_{1}+{m}_{1}m{_2}).4({P'}_{2}.{P'}_{1}+{m'}_{1}{m'}_{2}).4(P_{2}.P_{1}+m_{1}m{_2}).\\&4(P'_{2}.P'_{1}+m'_{1}m'{_2}).4(P_{2}.P_{1}+m_{1}m{_2}).4(P'_{2}.P'_{1}+m'_{1}m'{_2}).4(P_{2}.P_{1}+m_{1}m{_2})\\
{|\mathcal{M}_{fi}|}^2&=65536({P'}_{2}.{P'}_{1}+{m'}_{1}{m'}_{2})^{4}(P_{2}.P_{1}+{m}_{1}{m}_{2})^{4}
\end{split}
\end{gather}

tenemos que en CM
\begin{gather}
\begin{split}
\textbf{P}_1+\textbf{P}_2-\textbf{P'}_1+\textbf{P'}_2=0
\end{split}
\end{gather}

lo cual implica que
\begin{gather}
\begin{split}
&\textbf{P}_1=-\textbf{P}_2\\
&\textbf{P'}_1=-\textbf{P'}_2
\end{split}
\end{gather}

tomando en cuenta que $|\textbf{P'}_1|=\sqrt{{E'}_{1}^2-{m'}_{1}^2}$ se tiene

\begin{gather}
\begin{split}
&\textbf{P'}_1^2=\textbf{P'}_2^2\\
&{E'}_{1}^2-{m'}_{1}^2={E'}_{2}^2-{m'}_{2}^2\\
&{E'}_{2}^2={E'}_{1}^2-{m'}_{1}^2+{m'}_{2}^2
\end{split}
\end{gather}

Ademas, podemos definir la energia del centro de masa como
\begin{gather}
\begin{split}
\sqrt{s}=E_{1}+E_{2}
\end{split}
\end{gather}

igualmente se tiene
\begin{gather}
\begin{split}
|{P'}_{1}|=\sqrt{{E'}_{1}^{2}-{m'}_{1}^2}
\end{split}
\end{gather}

por otra parte
\begin{gather}
\begin{split}
{P}_{1}.{P}_{2}-{m}_{f}^{2}&={E}_{1}{E}_{2}-\textbf{P}_{1}.\textbf{P}_{2}-{m}_{1}{m}_{2}\\
&\frac{1}{2}[S^{2}-(m_{1}-m_{2})^{2}]
\end{split}
\end{gather}

\begin{gather}
\begin{split}
{P'}_{1}.{P'}_{2}=-{E'}_{1}{m}_{1}^{2}+{E'}_{1}{m}_{2}^{2}+2{E'}_{1}{m}_{1}-{m}_{1}^2
\end{split}
\end{gather}

considerando los siguientes dados iniciales 
\begin{gather}
\begin{split}
&{m}_{1}={m}_{2}=50 GeV\\
&{m'}_{1}={m'}_{2}=0 GeV\\
&{E}_{1}={E}_{2}=500 Gev\\
&{E'}_{1}={E'}_{2}=500 Gev\\
&{s}=10^{6}{GeV}^{2}
\end{split}
\end{gather}

\section{Process calculation}

Usando la ecuacion

\begin{equation}
  \frac{d\sigma}{d\Omega} = \frac{1}{64 \pi^2 s} \left\{ \frac{ [s-(m'_1 +m'_2)^2]}{[s-(m_1 + m_2)^2]} \frac{[s-(m'_1 - m'_2)^2 ]}{[s-(m_1 -m_2)^2]} \right\}^{\frac{1}{2}} | \bar{\mathcal{M}} |^2
\end{equation}


\begin{gather}
\begin{split}
\sigma=\frac{1}{64\pi^2s}{(\frac{s}{s-2m_{1}})}^{2}{|\mathcal{M}_{fi}|}^{2}\int{sen(\theta)}d\theta d\phi
\end{split}
\end{gather}


Reemplazando los valores predeterminados de (50) en (52), se tiene

\begin{gather}
\begin{split}
\sigma=10,110\times10^{-45}{cm}^{2}
\end{split}
\end{gather}

\includegraphics{imagen_1.pdf}

\section{CalcHEP comparison}

Check the result with CalcHEP. Please give the LanHEP code if necessaáry.

\section{Copyright}
\includegraphics[scale=0.5]{cc} Creative Commons Attribution-Share Alike 3.0 United States License.

%%% Local Variables:
%%% mode: latex
%%% TeX-master: "qft_samples"
%%% End:


% Radiative see saw II: Wady 0808.3340

\chapter{$\eta \eta \to h h$}

Author:  Wady Alexander Ríos Herrera

Proceso: \textbf{Radiative seesaw II}\\

Experimentos de reactores  de neutrinos han demostrado que los neutrinos tienen masa.


\section{Lagrangian}


El modelo considerado es una extensión del modelo estándar, el contiene SU($2$)$\times$ U($1$)$_{Y}$ de singletes $N_{i}$ y un segundo doblete de Higgs $\eta$. En adición, una simetría discreta exacta $Z_{2}$ es asumida tal que el nuevo campo son impares bajo $Z_{2}$, mientras que en el modelo estándar son pares. El lagrangiano de interacción de  Yukawa inducido por el nuevo doblete Higgs es dada por  \cite{1}
\begin{equation}
L_{Y}=\epsilon _{ab}h_{\alpha j}\bar{N}_{j}P_{L}L_{\alpha}^{a}\eta ^{b}+H.c,
\label{1}
\end{equation}
 
L son doblete de leptón izquierdo, $\alpha , j$ son indices de generación (Greek indices etiquetado sabor leptonico e,$\mu$,$\tau$) y $\epsilon _{ab}$ es el tensor antisimetrico completo.\\

A partir del lagrangiano de interacción de Yukawa se determina el langrangiano interacción escalar cuadrático  \cite{2} 

\begin{equation}
L_{I}=\lambda_{3}(\Phi^{+}\Phi)( \eta^{+}\eta)+\lambda_{2}(\Phi^{+}\eta)( \eta^{+}\Phi) \frac{\lambda _{5}}{2}(\Phi^{+}\eta + H.C)
\label{2}
\end{equation}


Donde $\Phi$ es el modelo estándar del doblete de Higgs y es solo relevante para generación de masa neutrina. Ya que $Z_{2}$ es asumida  ser  exactamente simétrica del modelo $\eta$ tiene valor esperado del vacío cero. La Física escalar de los bosones son por lo tanto $R_{e}\Phi ^{0},\eta^{\pm},\eta_{0}^{R}\equiv Re\eta_{0}, \eta_{0}^{I}\equiv Im\eta_{0} =0 $ y $Im\eta^{+} =0$.\\


para ecuación \ref{2} se tiene que 
\begin{equation}
\Phi=\begin{pmatrix}
0 & \frac{h+v}{\sqrt{2}}
\end{pmatrix}, \ \eta= \begin{pmatrix}
\eta^{+} & \\
\eta_{0} & 
\end{pmatrix} 
\label{2.1}
\end{equation}

donde $h$ y $v$ son parámetros reales por tanto $\Phi =\Phi ^{+}$


Remplazando \ref{2.1} y las partes reales de $\eta$ el lagrangiano \ref{2} se determina el termina el termino relevante

\begin{equation*}
L_{I}=\lambda_{l} \left[\begin{pmatrix}
0 & \frac{h+v}{\sqrt{2}}
\end{pmatrix}\begin{pmatrix}
\eta^{+} & \\
\eta_{0} & 
\end{pmatrix}\right]^{2}= \lambda_{l}\left[0\eta^{+}+\left( \frac{h+v}{\sqrt{2}}\right)\eta_{0}\right]^{2}
\end{equation*}
\begin{equation*}
=\lambda_{l}(h+v)^{2}(\eta_{0})^{2}=\lambda_{l}(h^{2}+2hv+v^{2})(\eta_{0})^{2}
\end{equation*}
\begin{equation}
=\underbrace{\lambda_{l}h^{2}\eta_{0}^{2}}_{\star}+2\lambda_{l}hv\eta_{0}^{2}+\lambda_{l}v^{2}\eta_{0}^{2}, 
\label{3}
\end{equation}

Con 
\begin{equation}
\lambda_{l}=\frac{\lambda _{3}+\lambda _{4}+\lambda _{5}}{2}
\end{equation}

donde el termino relevante de la expresión \ref{3} es denotado con $\star$, ya que este genera el siguiente proceso 

\begin{figure*}[htb]
    \centering
    \includegraphics[width=0.5\textwidth]{1}%(eps preferiblemente)
    \caption[electrones]{Proceso donde dos etas se destruyen y se crean dos Higgs}
    \label{fi}
\end{figure*}
 



Por tanto el lagrangiano de interés es
\begin{equation}
L_{i}=\lambda_{l}h^{2}\eta_{0}^{2}
\label{4}
\end{equation}










\section{S-matriz}
 El orden a calcular la matriz $S$ es de orden uno ya que solo tenemos un termino de interacción \ref{4} el cual  determina el proceso de figuara \ref{fi}.\\
 Consideremos el proceso de la figura \ref{fi} donde $\eta(p_{1})\eta(p_{2})$ decae a un par de $h(p_{1}^{'})h(p_{2}^{'})$.
 El elemento de matriz $S_{fi}^{1}$ entre el estado inicial  y el estado final es 
 
 \begin{equation}
S_{fi}^{(1)}=-\lambda_{l}\int d^{4}x\bra{h(p_{1}^{'})h(p_{2}^{'})}h^{2}\eta_{0}^{2}\ket{\eta(p_{1})\eta(p_{2})}, 
 \label{5}
 \end{equation}
 
 sabemos que
 
 \begin{equation}
\eta_{0}=\eta_{+}+\eta_{-}=\frac{1}{\sqrt{2E_{p}V}}\left[ae^{-ip.x}+a^{+}e^{ip.x}\right]
 \label{6}
 \end{equation}

\begin{equation}
h=h_{+}+h_{-}=\frac{1}{\sqrt{2E_{p}V}}\left[ae^{-ip.x}+a^{+}e^{ip.x}\right]
 \label{7}
 \end{equation}


Por lo tanto
\begin{equation*}
h^{2}=\left(h_{+}+h_{-}\right)^{2}=(h_{+}+h_{-})(h_{+}+h_{-})
\end{equation*}
 \begin{equation}
=h_{+}h_{+}+h_{+}h_{-}+h_{-}h_{+}+h_{-}h_{-}
 \label{8}
 \end{equation}

\begin{equation*}
\eta_{0}^{2}=\left(\eta_{+}+\eta_{-}\right)^{2}=(\eta_{+}+\eta_{-})(\eta_{+}+\eta_{-})
\end{equation*}
\begin{equation}
=\eta_{+}\eta_{+}+\eta_{+}\eta_{-}+\eta_{-}\eta_{+}+\eta_{-}\eta_{-}
\label{9}
\end{equation}

El termino $\frac{1}{\sqrt{2E_{p}V}}$ de la expresión \ref{6} y \ref{7} es de normalización.\\

Calculemos  el termino $h^{2}\eta_{0}^{2}$:
\begin{equation*}
h^{2}\eta_{0}^{2}=(h_{+}h_{+}+h_{+}h_{-}+h_{-}h_{+}+h_{-}h_{-})(\eta_{+}\eta_{+}+\eta_{+}\eta_{-}+\eta_{-}\eta_{+}+\eta_{-}\eta_{-})
\end{equation*}
\begin{equation*}
=h_{+}h_{+}\eta_{+}\eta_{+}+h_{+}h_{+}\eta_{+}\eta_{-}+h_{+}h_{+}\eta_{-}\eta_{+}+h_{+}h_{+}\eta_{-}\eta_{-}+
\end{equation*}
\begin{equation*}
h_{+}h_{-}\eta_{+}\eta_{+}+h_{+}h_{-}\eta_{+}\eta_{-}+h_{+}h_{-}\eta_{-}\eta_{+}+h_{+}h_{-}\eta_{-}\eta_{-}+
\end{equation*}
\begin{equation*}
h_{-}h_{+}\eta_{+}\eta_{+}+h_{-}h_{+}\eta_{+}\eta_{-}+h_{-}h_{+}\eta_{-}\eta_{+}+h_{-}h_{+}\eta_{-}\eta_{-}+
\end{equation*}
\begin{equation}
h_{-}h_{-}\eta_{+}\eta_{+}+h_{-}h_{-}\eta_{+}\eta_{-}+h_{-}h_{-}\eta_{-}\eta_{+}+h_{-}h_{-}\eta_{-}\eta_{-}
\label{10}
\end{equation}
 Por otro lado  sabemos que 
 
 \begin{equation}
 h_{+}\ket{h}=\frac{1}{\sqrt{2E_{p}V}}e^{-ip.x}\ket0 ,\ h_{-}\ket{h}=\frac{1}{\sqrt{2E_{p}V}}e^{ip.x}\ket{1_{h}}
 \label{11}
 \end{equation}
 \begin{equation}
 \bra h h_{-}=\frac{1}{\sqrt{2E_{p}V}}e^{ip.x}\bra{0} ,\ \bra h h_{+}=\frac{1}{\sqrt{2E_{p}V}}e^{-ip.x}\bra{1_{h}}
 \label{12}
 \end{equation}


\begin{equation}
 \eta_{+}\ket{\eta}=\frac{1}{\sqrt{2E_{p}V}}e^{-ip.x}\ket0 ,\ \eta_{-}\ket{\eta}=\frac{1}{\sqrt{2E_{p}V}}e^{ip.x}\ket{1_{\eta}}
 \label{13}
 \end{equation}
 \begin{equation}
 \bra \eta \eta_{-}=\frac{1}{\sqrt{2E_{p}V}}e^{ip.x}\bra{0} ,\ \bra \eta \eta_{+}=\frac{1}{\sqrt{2E_{p}V}}e^{-ip.x}\bra {1_{\eta}}
 \label{14}
 \end{equation}
 
 y
 
 \begin{equation}
\bra{h(p_{1}^{'})h(p_{2}^{'})}h_{+}h_{+}=\frac{1}{\sqrt{2E_{p_{1}^{'}}V}}\frac{1}{\sqrt{2E_{p_{2}^{'}}V}}e^{-ip_{1}^{'}.x}e^{-ip_{2}^{'}.x}\bra{1_{h}1_{h}}
 \label{15}
 \end{equation}
\begin{equation}
\bra{h(p_{1}^{'})h(p_{2}^{'})}h_{+}h_{-}=\frac{1}{\sqrt{2E_{p_{1}^{'}}V}}\frac{1}{\sqrt{2E_{p_{2}^{'}}V}}e^{-ip_{1}^{'}.x}e^{ip_{2}^{'}.x}\bra{1_{h}0}
 \label{16}
 \end{equation}
\begin{equation}
\bra{h(p_{1}^{'})h(p_{2}^{'})}h_{-}h_{+}=\frac{1}{\sqrt{2E_{p_{1}^{'}}V}}\frac{1}{\sqrt{2E_{p_{2}^{'}}V}}e^{ip_{1}^{'}.x}e^{-ip_{2}^{'}.x}\bra{01_{h}}
 \label{17}
 \end{equation}
\begin{equation}
\bra{h(p_{1}^{'})h(p_{2}^{'})}h_{-}h_{-}=\frac{1}{\sqrt{2E_{p_{1}^{'}}V}}\frac{1}{\sqrt{2E_{p_{2}^{'}}V}}e^{ip_{1}^{'}.x}e^{ip_{2}^{'}.x}\bra{00}
 \label{18}
 \end{equation}

\begin{equation}
\eta_{+}\eta_{+}\ket{\eta(p_{1})\eta(p_{2})}=\frac{1}{\sqrt{2E_{p_{1}}V}}\frac{1}{\sqrt{2E_{p_{2}}V}}e^{-ip_{1}.x}e^{-ip_{2}.x}\ket{00}
 \label{19}
 \end{equation}


\begin{equation}
\eta_{+}\eta_{-}\ket{\eta(p_{1})\eta(p_{2})}=\frac{1}{\sqrt{2E_{p_{1}}V}}\frac{1}{\sqrt{2E_{p_{2}}V}}e^{-ip_{1}.x}e^{ip_{2}.x}\ket{01_{\eta}}
 \label{20}
 \end{equation}
\begin{equation}
\eta_{-}\eta_{+}\ket{\eta(p_{1})\eta(p_{2})}=\frac{1}{\sqrt{2E_{p_{1}}V}}\frac{1}{\sqrt{2E_{p_{2}}V}}e^{ip_{1}.x}e^{-ip_{2}.x}\ket{1_{\eta}0}
 \label{21}
 \end{equation}
\begin{equation}
\eta_{-}\eta_{-}\ket{\eta(p_{1})\eta(p_{2})}=\frac{1}{\sqrt{2E_{p_{1}}V}}\frac{1}{\sqrt{2E_{p_{2}}V}}e^{ip_{1}.x}e^{ip_{2}.x}\ket{1_{\eta}1_{\eta}}
 \label{22}
 \end{equation}

Por tanto san duchando el termino $h^{2}\eta_{0}^{2}$ con las relaciones desde \ref{15} hasta \ref{22} el termino que sobrevive es


\begin{equation*}
\bra{h(p_{1}^{'})h(p_{2}^{'})}h^{2}\eta_{0}^{2}\ket{\eta(p_{1})\eta(p_{2})}=\bra{h(p_{1}^{'})h(p_{2}^{'})}h_{-}h_{-}\eta_{+}\eta_{+}\ket{\eta(p_{1})\eta(p_{2})}
\end{equation*}
\begin{equation}
=\frac{1}{\sqrt{2E_{p_{1}^{'}}V}}\frac{1}{\sqrt{2E_{p_{2}^{'}}V}}\frac{1}{\sqrt{2E_{p_{1}}V}}\frac{1}{\sqrt{2E_{p_{2}}V}}e^{i(p_{1}^{'}+p_{2}^{'}-p_{1}-p_{2})}\bra{00}\ket{00}
\label{23}
\end{equation}

remplazando \ref{23} en matriz $S_{fi}^{(1)}$ obtenemos

\begin{equation}
S_{fi}^{(1)}=-i\lambda_{l}\frac{1}{\sqrt{2E_{p_{1}^{'}}V}}\frac{1}{\sqrt{2E_{p_{2}^{'}}V}}\frac{1}{\sqrt{2E_{p_{1}}V}}\frac{1}{\sqrt{2E_{p_{2}}V}}\int d^{4}x e^{i(p_{1}^{'}+p_{2}^{'}-p_{1}-p_{2})}\bra{00}\ket{00}
\label{24}
\end{equation}


El valor de la integral es $(2\pi)^{4}\delta^{4}(p_{1}+p_{2}-p_{1}^{'}-p_{2}^{'})$ y $\bra{00}\ket{00}=1$ por tanto el elemento de matriz obtenido es

\begin{equation}
S_{fi}^{(1)}=-i\lambda_{l}\frac{1}{\sqrt{2E_{p_{1}^{'}}V}}\frac{1}{\sqrt{2E_{p_{2}^{'}}V}}\frac{1}{\sqrt{2E_{p_{1}}V}}\frac{1}{\sqrt{2E_{p_{2}}V}}(2\pi)^{4}\delta^{4}(p_{1}+p_{2}-p_{1}^{'}-p_{2}^{'})
\label{25}
\end{equation}



\section{Process calculation}

 De la ecuación \ref{25} la función $\delta$  es la conservación de los $4$-momentos en el proceso general, el cual es multiplicado por $(2\pi)^{4}$. Hay un factor $(2EV)^{\frac{-1}{2} }$ para cada partícula del estado inicial y del estado final de energía $E$. El restos de termino de la expresión \ref{25} el cual depende sobre la naturaleza exacta de la interacción es llamada la amplitud de Feynman, es denotada por $iM_{fi}$.\\
 Para nuestro caso  
 \begin{equation}
iM_{fi}= -i\lambda_{l}
  \label{26}
 \end{equation}
              
   Para determinar la sección eficaz de nuestro modelo se determina a partir de expresión\cite{3}
   
   \begin{equation}
\frac{d\sigma}{d\Omega}=\frac{1}{64 \small{\pi} ^{2}s}\left[\frac{(s-(m_{1}^{'}+m_{2}^{'})^{2})(s-(m_{1}^{'}-m_{2}^{'})^{2}))}{(s-(m_{1}+m_{2})^{2})(s-(m_{1}-m_{2})^{2}))}\right]^{\frac{1}{2}}\overline{\vert M_{fi}\vert^{2}}
   \label{27}
   \end{equation}            
              
     Donde $m_{1}^{'}=m_{2}^{'}=m_{h}$ es la masa del Higgs, $m_{1}=m_{2}=m_{\eta}$ es la masa del eta y  $\overline{\vert M_{fi}\vert^{2}}=\lambda_{l}$ es la amplitud de Feynman, remplazando estos términos en la expresión \ref{27} se obtiene
     
     \begin{equation}
      \frac{d\sigma}{d\Omega}=\frac{1}{64\small{\pi} ^{2}s}\left[\frac{(s-4m_{h}^{2})}{(s-4m_{\eta}^{2})}\right]^{\frac{1}{2}}\left(\lambda_{l}\right)^{2}
          \label{28}
     \end{equation}
     
     
Con $s=\sqrt{E_{1}+E_{2}}$ y $E=\sqrt{p^{2}+m^{2}}$, la integral sobre $d\Omega$ es $4\pi$ por tanto la sección eficaz es 


\begin{equation}
  \sigma=\frac{(\lambda_{l})^{2}}{64\small{\pi} s}\left[\frac{(s-4m_{h}^{2})}{(s-4m_{\eta}^{2})}\right]^{\frac{1}{2}}
  \label{29}
\end{equation}  
          
     
     
Para determinar un valor númerico de la sección eficaz tomamos los siguientes valores: $ m_{\eta}=50GeV, m_{h}=120 GeV, \lambda_{l}=0.1$ y $\sqrt{s}=1016.68GeV $, Por tanto la expresión \ref{29} que así

\begin{equation*}
\sigma=\frac{(0.1)^{2}}{64\small{\pi} (1016.68GeV)^{2}}\left[\frac{(1016.68GeV)^{2}-4(120GeV)^{2}}{(1016.68GeV)^{2}-4(50GeV)^{2}}\right]^{\frac{1}{2}}
\label{30}GeV
\end{equation*}
     \begin{equation*}
    \frac{(0.1)^{2}}{64\small{\pi} (1016.68GeV)^{2}}\left[\frac{976055.3134 (GeV)^{2}}{1023638.232(GeV)^{2}}\right]^{\frac{1}{2}}
     \end{equation*}
              
\begin{equation*}
   = \frac{(0.1)^{2}}{64\small{\pi} (1016.68GeV)^{2}}(0,9764)
     \end{equation*}   
     \begin{equation}
    =4,69\times 10^{-11}(GeV)^{-2}
     \end{equation}              
              
\section{CalcHEP comparison}
Para ser la simulación se instala los programas de LanHep y CalHep.\\
Después se baja los archivos del proceso a simular de github, nuestro archivo pertinente es la carpeta idm para nuestro sistema. Se  instala la carpeta idm  en LanHep y CalHep y se ejecuta  CalHep desde el directorio idm para iniciar la simulación.\\
Los parámetros de la simulación son $~H0,~H0->H,H$ donde se descoge el diagrama de la figura \ref{fi} luego se ingresa los parámetros de la simulación que son $\lambda_{l}=0.1, m_{\eta}=50 GeV, m_{h}=120 GeV$, momento inicial $500 GeV$ y momento final es de $500GeV$ por tanto al sección eficaz de la simulación es de $0.145983[pb]$  

\section{Copyright}
\includegraphics[scale=0.5]{cc} Creative Commons Attribution-Share Alike 3.0 United States License.

\begin{thebibliography}{9}

\bibitem{1} D. Aristizabal Sierra, Jisuke Kubo and Daijiro Suematsu, Physical Review D 79, 013001 (2009)
\bibitem{2} Ernast Ma, Physical Review D 73, 0077301(2006)
\bibitem{3} Quantum Fiel Theory, Amitabha Lahiri and Palash B. Pal

\end{thebibliography}

%%% Local Variables: 
%%% mode: latex
%%% TeX-master: "qft_samples"
%%% End: 

% Radiative see saw III: 
%
\chapter{$\eta \eta \to h h$}

Author:  Wady Alexander Ríos Herrera

Proceso: \textbf{Radiative seesaw II}\\

Experimentos de reactores  de neutrinos han demostrado que los neutrinos tienen masa.


\section{Lagrangian}


El modelo considerado es una extensión del modelo estándar, el contiene SU($2$)$\times$ U($1$)$_{Y}$ de singletes $N_{i}$ y un segundo doblete de Higgs $\eta$. En adición, una simetría discreta exacta $Z_{2}$ es asumida tal que el nuevo campo son impares bajo $Z_{2}$, mientras que en el modelo estándar son pares. El lagrangiano de interacción de  Yukawa inducido por el nuevo doblete Higgs es dada por  \cite{1}
\begin{equation}
L_{Y}=\epsilon _{ab}h_{\alpha j}\bar{N}_{j}P_{L}L_{\alpha}^{a}\eta ^{b}+H.c,
\label{1}
\end{equation}
 
L son doblete de leptón izquierdo, $\alpha , j$ son indices de generación (Greek indices etiquetado sabor leptonico e,$\mu$,$\tau$) y $\epsilon _{ab}$ es el tensor antisimetrico completo.\\

A partir del lagrangiano de interacción de Yukawa se determina el langrangiano interacción escalar cuadrático  \cite{2} 

\begin{equation}
L_{I}=\lambda_{3}(\Phi^{+}\Phi)( \eta^{+}\eta)+\lambda_{2}(\Phi^{+}\eta)( \eta^{+}\Phi) \frac{\lambda _{5}}{2}(\Phi^{+}\eta + H.C)
\label{2}
\end{equation}


Donde $\Phi$ es el modelo estándar del doblete de Higgs y es solo relevante para generación de masa neutrina. Ya que $Z_{2}$ es asumida  ser  exactamente simétrica del modelo $\eta$ tiene valor esperado del vacío cero. La Física escalar de los bosones son por lo tanto $R_{e}\Phi ^{0},\eta^{\pm},\eta_{0}^{R}\equiv Re\eta_{0}, \eta_{0}^{I}\equiv Im\eta_{0} =0 $ y $Im\eta^{+} =0$.\\


para ecuación \ref{2} se tiene que 
\begin{equation}
\Phi=\begin{pmatrix}
0 & \frac{h+v}{\sqrt{2}}
\end{pmatrix}, \ \eta= \begin{pmatrix}
\eta^{+} & \\
\eta_{0} & 
\end{pmatrix} 
\label{2.1}
\end{equation}

donde $h$ y $v$ son parámetros reales por tanto $\Phi =\Phi ^{+}$


Remplazando \ref{2.1} y las partes reales de $\eta$ el lagrangiano \ref{2} se determina el termina el termino relevante

\begin{equation*}
L_{I}=\lambda_{l} \left[\begin{pmatrix}
0 & \frac{h+v}{\sqrt{2}}
\end{pmatrix}\begin{pmatrix}
\eta^{+} & \\
\eta_{0} & 
\end{pmatrix}\right]^{2}= \lambda_{l}\left[0\eta^{+}+\left( \frac{h+v}{\sqrt{2}}\right)\eta_{0}\right]^{2}
\end{equation*}
\begin{equation*}
=\lambda_{l}(h+v)^{2}(\eta_{0})^{2}=\lambda_{l}(h^{2}+2hv+v^{2})(\eta_{0})^{2}
\end{equation*}
\begin{equation}
=\underbrace{\lambda_{l}h^{2}\eta_{0}^{2}}_{\star}+2\lambda_{l}hv\eta_{0}^{2}+\lambda_{l}v^{2}\eta_{0}^{2}, 
\label{3}
\end{equation}

Con 
\begin{equation}
\lambda_{l}=\frac{\lambda _{3}+\lambda _{4}+\lambda _{5}}{2}
\end{equation}

donde el termino relevante de la expresión \ref{3} es denotado con $\star$, ya que este genera el siguiente proceso 

\begin{figure*}[htb]
    \centering
    \includegraphics[width=0.5\textwidth]{1}%(eps preferiblemente)
    \caption[electrones]{Proceso donde dos etas se destruyen y se crean dos Higgs}
    \label{fi}
\end{figure*}
 



Por tanto el lagrangiano de interés es
\begin{equation}
L_{i}=\lambda_{l}h^{2}\eta_{0}^{2}
\label{4}
\end{equation}










\section{S-matriz}
 El orden a calcular la matriz $S$ es de orden uno ya que solo tenemos un termino de interacción \ref{4} el cual  determina el proceso de figuara \ref{fi}.\\
 Consideremos el proceso de la figura \ref{fi} donde $\eta(p_{1})\eta(p_{2})$ decae a un par de $h(p_{1}^{'})h(p_{2}^{'})$.
 El elemento de matriz $S_{fi}^{1}$ entre el estado inicial  y el estado final es 
 
 \begin{equation}
S_{fi}^{(1)}=-\lambda_{l}\int d^{4}x\bra{h(p_{1}^{'})h(p_{2}^{'})}h^{2}\eta_{0}^{2}\ket{\eta(p_{1})\eta(p_{2})}, 
 \label{5}
 \end{equation}
 
 sabemos que
 
 \begin{equation}
\eta_{0}=\eta_{+}+\eta_{-}=\frac{1}{\sqrt{2E_{p}V}}\left[ae^{-ip.x}+a^{+}e^{ip.x}\right]
 \label{6}
 \end{equation}

\begin{equation}
h=h_{+}+h_{-}=\frac{1}{\sqrt{2E_{p}V}}\left[ae^{-ip.x}+a^{+}e^{ip.x}\right]
 \label{7}
 \end{equation}


Por lo tanto
\begin{equation*}
h^{2}=\left(h_{+}+h_{-}\right)^{2}=(h_{+}+h_{-})(h_{+}+h_{-})
\end{equation*}
 \begin{equation}
=h_{+}h_{+}+h_{+}h_{-}+h_{-}h_{+}+h_{-}h_{-}
 \label{8}
 \end{equation}

\begin{equation*}
\eta_{0}^{2}=\left(\eta_{+}+\eta_{-}\right)^{2}=(\eta_{+}+\eta_{-})(\eta_{+}+\eta_{-})
\end{equation*}
\begin{equation}
=\eta_{+}\eta_{+}+\eta_{+}\eta_{-}+\eta_{-}\eta_{+}+\eta_{-}\eta_{-}
\label{9}
\end{equation}

El termino $\frac{1}{\sqrt{2E_{p}V}}$ de la expresión \ref{6} y \ref{7} es de normalización.\\

Calculemos  el termino $h^{2}\eta_{0}^{2}$:
\begin{equation*}
h^{2}\eta_{0}^{2}=(h_{+}h_{+}+h_{+}h_{-}+h_{-}h_{+}+h_{-}h_{-})(\eta_{+}\eta_{+}+\eta_{+}\eta_{-}+\eta_{-}\eta_{+}+\eta_{-}\eta_{-})
\end{equation*}
\begin{equation*}
=h_{+}h_{+}\eta_{+}\eta_{+}+h_{+}h_{+}\eta_{+}\eta_{-}+h_{+}h_{+}\eta_{-}\eta_{+}+h_{+}h_{+}\eta_{-}\eta_{-}+
\end{equation*}
\begin{equation*}
h_{+}h_{-}\eta_{+}\eta_{+}+h_{+}h_{-}\eta_{+}\eta_{-}+h_{+}h_{-}\eta_{-}\eta_{+}+h_{+}h_{-}\eta_{-}\eta_{-}+
\end{equation*}
\begin{equation*}
h_{-}h_{+}\eta_{+}\eta_{+}+h_{-}h_{+}\eta_{+}\eta_{-}+h_{-}h_{+}\eta_{-}\eta_{+}+h_{-}h_{+}\eta_{-}\eta_{-}+
\end{equation*}
\begin{equation}
h_{-}h_{-}\eta_{+}\eta_{+}+h_{-}h_{-}\eta_{+}\eta_{-}+h_{-}h_{-}\eta_{-}\eta_{+}+h_{-}h_{-}\eta_{-}\eta_{-}
\label{10}
\end{equation}
 Por otro lado  sabemos que 
 
 \begin{equation}
 h_{+}\ket{h}=\frac{1}{\sqrt{2E_{p}V}}e^{-ip.x}\ket0 ,\ h_{-}\ket{h}=\frac{1}{\sqrt{2E_{p}V}}e^{ip.x}\ket{1_{h}}
 \label{11}
 \end{equation}
 \begin{equation}
 \bra h h_{-}=\frac{1}{\sqrt{2E_{p}V}}e^{ip.x}\bra{0} ,\ \bra h h_{+}=\frac{1}{\sqrt{2E_{p}V}}e^{-ip.x}\bra{1_{h}}
 \label{12}
 \end{equation}


\begin{equation}
 \eta_{+}\ket{\eta}=\frac{1}{\sqrt{2E_{p}V}}e^{-ip.x}\ket0 ,\ \eta_{-}\ket{\eta}=\frac{1}{\sqrt{2E_{p}V}}e^{ip.x}\ket{1_{\eta}}
 \label{13}
 \end{equation}
 \begin{equation}
 \bra \eta \eta_{-}=\frac{1}{\sqrt{2E_{p}V}}e^{ip.x}\bra{0} ,\ \bra \eta \eta_{+}=\frac{1}{\sqrt{2E_{p}V}}e^{-ip.x}\bra {1_{\eta}}
 \label{14}
 \end{equation}
 
 y
 
 \begin{equation}
\bra{h(p_{1}^{'})h(p_{2}^{'})}h_{+}h_{+}=\frac{1}{\sqrt{2E_{p_{1}^{'}}V}}\frac{1}{\sqrt{2E_{p_{2}^{'}}V}}e^{-ip_{1}^{'}.x}e^{-ip_{2}^{'}.x}\bra{1_{h}1_{h}}
 \label{15}
 \end{equation}
\begin{equation}
\bra{h(p_{1}^{'})h(p_{2}^{'})}h_{+}h_{-}=\frac{1}{\sqrt{2E_{p_{1}^{'}}V}}\frac{1}{\sqrt{2E_{p_{2}^{'}}V}}e^{-ip_{1}^{'}.x}e^{ip_{2}^{'}.x}\bra{1_{h}0}
 \label{16}
 \end{equation}
\begin{equation}
\bra{h(p_{1}^{'})h(p_{2}^{'})}h_{-}h_{+}=\frac{1}{\sqrt{2E_{p_{1}^{'}}V}}\frac{1}{\sqrt{2E_{p_{2}^{'}}V}}e^{ip_{1}^{'}.x}e^{-ip_{2}^{'}.x}\bra{01_{h}}
 \label{17}
 \end{equation}
\begin{equation}
\bra{h(p_{1}^{'})h(p_{2}^{'})}h_{-}h_{-}=\frac{1}{\sqrt{2E_{p_{1}^{'}}V}}\frac{1}{\sqrt{2E_{p_{2}^{'}}V}}e^{ip_{1}^{'}.x}e^{ip_{2}^{'}.x}\bra{00}
 \label{18}
 \end{equation}

\begin{equation}
\eta_{+}\eta_{+}\ket{\eta(p_{1})\eta(p_{2})}=\frac{1}{\sqrt{2E_{p_{1}}V}}\frac{1}{\sqrt{2E_{p_{2}}V}}e^{-ip_{1}.x}e^{-ip_{2}.x}\ket{00}
 \label{19}
 \end{equation}


\begin{equation}
\eta_{+}\eta_{-}\ket{\eta(p_{1})\eta(p_{2})}=\frac{1}{\sqrt{2E_{p_{1}}V}}\frac{1}{\sqrt{2E_{p_{2}}V}}e^{-ip_{1}.x}e^{ip_{2}.x}\ket{01_{\eta}}
 \label{20}
 \end{equation}
\begin{equation}
\eta_{-}\eta_{+}\ket{\eta(p_{1})\eta(p_{2})}=\frac{1}{\sqrt{2E_{p_{1}}V}}\frac{1}{\sqrt{2E_{p_{2}}V}}e^{ip_{1}.x}e^{-ip_{2}.x}\ket{1_{\eta}0}
 \label{21}
 \end{equation}
\begin{equation}
\eta_{-}\eta_{-}\ket{\eta(p_{1})\eta(p_{2})}=\frac{1}{\sqrt{2E_{p_{1}}V}}\frac{1}{\sqrt{2E_{p_{2}}V}}e^{ip_{1}.x}e^{ip_{2}.x}\ket{1_{\eta}1_{\eta}}
 \label{22}
 \end{equation}

Por tanto san duchando el termino $h^{2}\eta_{0}^{2}$ con las relaciones desde \ref{15} hasta \ref{22} el termino que sobrevive es


\begin{equation*}
\bra{h(p_{1}^{'})h(p_{2}^{'})}h^{2}\eta_{0}^{2}\ket{\eta(p_{1})\eta(p_{2})}=\bra{h(p_{1}^{'})h(p_{2}^{'})}h_{-}h_{-}\eta_{+}\eta_{+}\ket{\eta(p_{1})\eta(p_{2})}
\end{equation*}
\begin{equation}
=\frac{1}{\sqrt{2E_{p_{1}^{'}}V}}\frac{1}{\sqrt{2E_{p_{2}^{'}}V}}\frac{1}{\sqrt{2E_{p_{1}}V}}\frac{1}{\sqrt{2E_{p_{2}}V}}e^{i(p_{1}^{'}+p_{2}^{'}-p_{1}-p_{2})}\bra{00}\ket{00}
\label{23}
\end{equation}

remplazando \ref{23} en matriz $S_{fi}^{(1)}$ obtenemos

\begin{equation}
S_{fi}^{(1)}=-i\lambda_{l}\frac{1}{\sqrt{2E_{p_{1}^{'}}V}}\frac{1}{\sqrt{2E_{p_{2}^{'}}V}}\frac{1}{\sqrt{2E_{p_{1}}V}}\frac{1}{\sqrt{2E_{p_{2}}V}}\int d^{4}x e^{i(p_{1}^{'}+p_{2}^{'}-p_{1}-p_{2})}\bra{00}\ket{00}
\label{24}
\end{equation}


El valor de la integral es $(2\pi)^{4}\delta^{4}(p_{1}+p_{2}-p_{1}^{'}-p_{2}^{'})$ y $\bra{00}\ket{00}=1$ por tanto el elemento de matriz obtenido es

\begin{equation}
S_{fi}^{(1)}=-i\lambda_{l}\frac{1}{\sqrt{2E_{p_{1}^{'}}V}}\frac{1}{\sqrt{2E_{p_{2}^{'}}V}}\frac{1}{\sqrt{2E_{p_{1}}V}}\frac{1}{\sqrt{2E_{p_{2}}V}}(2\pi)^{4}\delta^{4}(p_{1}+p_{2}-p_{1}^{'}-p_{2}^{'})
\label{25}
\end{equation}



\section{Process calculation}

 De la ecuación \ref{25} la función $\delta$  es la conservación de los $4$-momentos en el proceso general, el cual es multiplicado por $(2\pi)^{4}$. Hay un factor $(2EV)^{\frac{-1}{2} }$ para cada partícula del estado inicial y del estado final de energía $E$. El restos de termino de la expresión \ref{25} el cual depende sobre la naturaleza exacta de la interacción es llamada la amplitud de Feynman, es denotada por $iM_{fi}$.\\
 Para nuestro caso  
 \begin{equation}
iM_{fi}= -i\lambda_{l}
  \label{26}
 \end{equation}
              
   Para determinar la sección eficaz de nuestro modelo se determina a partir de expresión\cite{3}
   
   \begin{equation}
\frac{d\sigma}{d\Omega}=\frac{1}{64 \small{\pi} ^{2}s}\left[\frac{(s-(m_{1}^{'}+m_{2}^{'})^{2})(s-(m_{1}^{'}-m_{2}^{'})^{2}))}{(s-(m_{1}+m_{2})^{2})(s-(m_{1}-m_{2})^{2}))}\right]^{\frac{1}{2}}\overline{\vert M_{fi}\vert^{2}}
   \label{27}
   \end{equation}            
              
     Donde $m_{1}^{'}=m_{2}^{'}=m_{h}$ es la masa del Higgs, $m_{1}=m_{2}=m_{\eta}$ es la masa del eta y  $\overline{\vert M_{fi}\vert^{2}}=\lambda_{l}$ es la amplitud de Feynman, remplazando estos términos en la expresión \ref{27} se obtiene
     
     \begin{equation}
      \frac{d\sigma}{d\Omega}=\frac{1}{64\small{\pi} ^{2}s}\left[\frac{(s-4m_{h}^{2})}{(s-4m_{\eta}^{2})}\right]^{\frac{1}{2}}\left(\lambda_{l}\right)^{2}
          \label{28}
     \end{equation}
     
     
Con $s=\sqrt{E_{1}+E_{2}}$ y $E=\sqrt{p^{2}+m^{2}}$, la integral sobre $d\Omega$ es $4\pi$ por tanto la sección eficaz es 


\begin{equation}
  \sigma=\frac{(\lambda_{l})^{2}}{64\small{\pi} s}\left[\frac{(s-4m_{h}^{2})}{(s-4m_{\eta}^{2})}\right]^{\frac{1}{2}}
  \label{29}
\end{equation}  
          
     
     
Para determinar un valor númerico de la sección eficaz tomamos los siguientes valores: $ m_{\eta}=50GeV, m_{h}=120 GeV, \lambda_{l}=0.1$ y $\sqrt{s}=1016.68GeV $, Por tanto la expresión \ref{29} que así

\begin{equation*}
\sigma=\frac{(0.1)^{2}}{64\small{\pi} (1016.68GeV)^{2}}\left[\frac{(1016.68GeV)^{2}-4(120GeV)^{2}}{(1016.68GeV)^{2}-4(50GeV)^{2}}\right]^{\frac{1}{2}}
\label{30}GeV
\end{equation*}
     \begin{equation*}
    \frac{(0.1)^{2}}{64\small{\pi} (1016.68GeV)^{2}}\left[\frac{976055.3134 (GeV)^{2}}{1023638.232(GeV)^{2}}\right]^{\frac{1}{2}}
     \end{equation*}
              
\begin{equation*}
   = \frac{(0.1)^{2}}{64\small{\pi} (1016.68GeV)^{2}}(0,9764)
     \end{equation*}   
     \begin{equation}
    =4,69\times 10^{-11}(GeV)^{-2}
     \end{equation}              
              
\section{CalcHEP comparison}
Para ser la simulación se instala los programas de LanHep y CalHep.\\
Después se baja los archivos del proceso a simular de github, nuestro archivo pertinente es la carpeta idm para nuestro sistema. Se  instala la carpeta idm  en LanHep y CalHep y se ejecuta  CalHep desde el directorio idm para iniciar la simulación.\\
Los parámetros de la simulación son $~H0,~H0->H,H$ donde se descoge el diagrama de la figura \ref{fi} luego se ingresa los parámetros de la simulación que son $\lambda_{l}=0.1, m_{\eta}=50 GeV, m_{h}=120 GeV$, momento inicial $500 GeV$ y momento final es de $500GeV$ por tanto al sección eficaz de la simulación es de $0.145983[pb]$  

\section{Copyright}
\includegraphics[scale=0.5]{cc} Creative Commons Attribution-Share Alike 3.0 United States License.

\begin{thebibliography}{9}

\bibitem{1} D. Aristizabal Sierra, Jisuke Kubo and Daijiro Suematsu, Physical Review D 79, 013001 (2009)
\bibitem{2} Ernast Ma, Physical Review D 73, 0077301(2006)
\bibitem{3} Quantum Fiel Theory, Amitabha Lahiri and Palash B. Pal

\end{thebibliography}

%%% Local Variables: 
%%% mode: latex
%%% TeX-master: "qft_samples"
%%% End: 

% zee model I: Andrés hep-ph/0604012
\chapter{$h^-\to \tau^-\nu_{\tau}$}
Author: Andr\'es Felipe Estrada Guerra


Description of the processes to be calculated: $h^-\to \tau^-\nu_{\tau}$

\section{Preliminares}

El Lagrangiano utilizado en (1) está dado por
\begin{eqnarray}\label{lagrangiano-1}
-\mc{L}_Y=\Bar{L}_i(\Pi_a)_{ij}H_ae_{R_j}+\epsilon_{\alpha\beta}\Bar{L}^\alpha_if_{ij}C(\Bar{L})^\beta_jh^-+\mbox{H.c},
\end{eqnarray}
en donde $L_i$ son dobletes leptónicos, $e_{R_j}$ son singletes leptónicos, $C$ es el operador de conjugación de carga, $\Pi_a$
($a=1,2$) y $f$ son matrices $3\times3$ en el espacio de sabor, $\epsilon_{\alpha\beta}$ ($\alpha,\beta=1,2$) es el tensor
antisimétrico para SU(2)$_L$ y $i,j=1,2,3$ son los índices de familia. La matriz $f$ es antisimétrica debida a la estadística de
Fermi.\\
En los eigenestados de masa para los escalares cargados, el lagrangiano anterior se escribe de la siguiente manera:
\begin{eqnarray}\label{lagrangiano-2}
-\mc{L}_Y  & \supset & \Bar{\nu}_{L_i'}O_{i'j}e_{R_j}(h_1^+\cos\varphi -h_2^+\sin\varphi) \nonumber \\
& & +(\nu_{L_i})^TC(2f_{ij})e_{L_j}(h_1^+\sin\varphi +h_2^+\cos\varphi)+\mbox{H.c.},
\end{eqnarray}
en donde $W_{11}=W_{22}=\cos\varphi$ y $W_{12}=-W_{21}=\sin\varphi$.\\
Estoy interesado en  el decaimiento del escalar cargado $h_1^+$ en dos leptones: $\nu_\tau$ y $\Bar{\tau}$. Por lo tanto, la
parte del lagrangiano anterior que describe este decaimiento está dado por:
\begin{eqnarray}\label{lagrangiano-3}
\mc{L}_Y&=&-(\nu_{L_i})^TC(2f_{ij})e_{L_j}h_1^+\sin\varphi +\mbox{H.c.} \nonumber \\
&=&-(\nu_{L_i})^T(2f_{ij})\Bar{e}_{L_j}h_1^+\sin\varphi +\mbox{H.c.},
\end{eqnarray}
por lo que el Hamiltoniano es:
\begin{eqnarray}\label{hamilton-1}
H=(\nu_{L_i})^T(2f_{ij})\Bar{e}_{L_j}h_1^+\sin\varphi +\mbox{H.c.}.
\end{eqnarray}
%En la siguiente figura se ilustra el proceso bajo estudio:
%\begin{center}
%\begin{figure}[h]
 %\includegraphics[scale=1]{grafica.png}
 %\caption{Grafica del proceso $h_1^+(k)\,\rightarrow\,v_\tau(k)\,\tau^+(p)$.}
%\end{figure}
%\end{center}
La matriz $S$ a primer orden está dada por:
\begin{eqnarray}\label{s-matriz-1}
S^{(1)}&=&\mc{T}\left[-i\int d^4x\,:\mc{H}:\right] \nonumber \\
&=&-i(2f)\sin\varphi\int d^4x\,:\psi(x)\Bar{\psi}(x)\phi(x): ,
\end{eqnarray}
en donde se ha omitido el símbolo de ordenamiento temporal, debido a que existe un solo punto en el espacio-tiempo y, por lo
tanto, un solo tiempo. Debido a esto, el teorema de Wick no es necesario para este orden de la matriz $S$.\\
Ahora bien, la descomposición de Fourier para los campos involucrados está dada por:
\begin{eqnarray}\label{fourier-campos-1}
\phi(x)&=&\int\frac{d^3p}{\sqrt{(2\pi)^32E_p}}\left[a(\vtr{p})e^{-ip\cdot x}+a^\dagger(\vtr{p})e^{ip\cdot x}\right] \nonumber \\
&=&\phi_++\phi_-, \nonumber \\
\psi(x)&=&\int\frac{d^3p}{\sqrt{(2\pi)^32E_p}}\sum_{s=1,2}\left[f_s(\vtr{p})u_s(\vtr{p})e^{-ip\cdot
x}+\hat{f}_s^\dagger(\vtr{p})v_s(\vtr{p})e^{ip\cdot x}\right] \nonumber \\
&=&\psi_++\psi_-, \nonumber \\
\bar{\psi}(x)&=&\int\frac{d^3p}{\sqrt{(2\pi)^32E_p}}\sum_{s=1,2}\left[f_s^\dagger(\vtr{p})\bar{u}_s(\vtr{p})e^{ip\cdot
x}+\hat{f}_s(\vtr{p})\bar{v}_s(\vtr{p})e^{-ip\cdot x}\right] \nonumber \\
&=&\bar{\psi}_-+\bar{\psi}_+.
\end{eqnarray}
Los estados de una sola partícula están definidos de la siguiente manera:
\begin{equation}
  \label{estados-1-part-1}
\begin{split}
\ketm{h_1^+(\vtr{k})}&=\sqrt{\frac{(2\pi)^3}{V}}a^\dagger(\vtr{k})\ketm{0} \\
\ketm{\nu_\tau(\vtr{k},s)}&=\sqrt{\frac{(2\pi)^3}{V}}f_s^\dagger(\vtr{k})\ketm{0} \\
\ketm{\tau^+(\vtr{p},s)}&=\sqrt{\frac{(2\pi)^3}{V}}\Hat{f}_s^\dagger(\vtr{p})\ketm{0}. 
\end{split}
\end{equation}
Utilizando las relaciones de conmutación y anticonmutación entre los operadores de creación y destrucción:
\begin{eqnarray*}
\conmu{a(\vtr{k})}{a(\vtr{k}')}&=&\delta^3(\vtr{k}-\vtr{k}') \\
\aconmu{f_s(\vtr{k})}{f_{s'}^\dagger(\vtr{k}')}&=&\delta_{ss'}\delta^3(\vtr{k}-\vtr{k}') \\
\aconmu{\hat{f}_s(\vtr{p})}{\hat{f}_{s'}^\dagger(\vtr{p}')}&=&\delta_{ss'}\delta^3(\vtr{p}-\vtr{p}'),
\end{eqnarray*}
se llega entonces a que los estados de una sola partícula están normalizados de la siguiente manera:
\begin{eqnarray*}
\prodin{h_1^+(\vtr{k})}{h_1^+(\vtr{k}')}&=&\frac{(2\pi)^3}{V}\delta^3(\vtr{k}-\vtr{k}')\\
\prodin{\nu_\tau(\vtr{k},s)}{\nu_\tau(\vtr{k}',s')}&=&\frac{(2\pi)^3}{V}\delta_{ss'}\delta^3(\vtr{k}-\vtr{k}')\\
\prodin{\tau^+(\vtr{p},s)}{\tau^+(\vtr{p}',s')}&=&\frac{(2\pi)^3}{V}\delta_{ss'}\delta^3(\vtr{p}-\vtr{p}').
\end{eqnarray*}
Tal como se demuestra en las notas del curso, es conveniente trabajar en el límite discreto, en donde:
\begin{eqnarray*}
\delta^3(0)_p=\frac{V}{(2\pi)^3}.
\end{eqnarray*}
Con esta definición, los estados de una sola partícula en (\ref{estados-1-part-1}) están normalizados a la unidad, siempre y
cuando se tome el límite de $V\rightarrow \infty$ al final del cálculo de cualquier cantidad física.\\
La acción de los operadores de campo sobre los estados de una partícula están dados por:
\begin{eqnarray}\label{accion-operadores-campos-1}
\phi_+(x)\ketm{h_1^+(\vtr{k})}&=&\int\frac{d^3p}{\sqrt{(2\pi)^32\omega_p}}a(\vtr{p})e^{-ip\cdot
x}\sqrt{\frac{(2\pi)^3}{V}}a^\dagger(\vtr{k})\ketm{0} \nonumber \\
&=&\int\frac{d^3p}{2\omega_pV}\left[a^\dagger(\vtr{k})a(\vtr{p})+\delta^3(\vtr{p}-\vtr{k})\right]e^{-ip\cdot x}\ketm{0} \nonumber \\
&=&\frac{1}{\sqrt{2\omega_kV}}e^{-ik\cdot x}\ketm{0}, \nonumber \\
\psi_+(x)\ketm{\nu_\tau(\vtr{k},s)}&=&\int\frac{d^3p}{\sqrt{(2\pi)^32\omega_p}}\sum_{s'=1,2}f_{s'}(\vtr{p})u_{s'}(\vtr{p})e^{
-ip\cdot x}\sqrt{\frac{(2\pi)^3}{V}}f_s^\dagger(\vtr{k})\ketm{0} \nonumber \\
&=&\int\frac{d^3p}{\sqrt{2\omega_pV}}\sum_{s'=1,2}\left[-f_s^\dagger(\vtr{k})f_{s'}(\vtr{p})+\delta_{ss'}\delta^3(\vtr{k}-\vtr{p}
)\right ] u_{s'}(\vtr{p})e^{-ip\cdot x}\ketm{0} \nonumber \\
&=&\frac{1}{\sqrt{2\omega_kV}}u_s(\vtr{k})e^{-ik\cdot x}\ketm{0},\nonumber \\
\Bar{\psi}_+(x)\ketm{\tau^+(\vtr{p},s)}&=&\int\frac{d^3p'}{\sqrt{(2\pi)^32E_{p'}}}\sum_{s'=1,2}\hat{f}_{s'}(\vtr{p}')\Bar{v}_{s'}
(\vtr { p}')e^{-ip'\cdot x}\sqrt{\frac{(2\pi)^3}{V}}\Hat{f}_s(\vtr{p})\ketm{0}\nonumber \\
&=&\frac{1}{\sqrt{2E_pV}}\Bar{v}_s(\vtr{p})e^{-ip\cdot x}\ketm{0}.
\end{eqnarray}
De manera totalmente análoga se obtiene para los operadores adjuntos:
\begin{eqnarray}\label{accion-operadores-campos-2}
\left<h_1^+(\vtr{k})\right|\phi_-(x)&=&\frac{1}{\sqrt{2\omega_kV}}e^{ik\cdot x}\left<0\right|, \nonumber \\
\left<\nu_\tau(\vtr{k},s)\right|\Bar{\psi}_-(x)&=&\frac{1}{\sqrt{2\omega_kV}}\Bar{u}_s(\vtr{k})e^{ik\cdot x}\left<0\right|, \nonumber \\
\left<\tau^+(\vtr{p},s)\right|\psi_-(x)&=&\frac{1}{\sqrt{2E_pV}}v_s(\vtr{p})e^{ip\cdot x}\left<0\right|.
\end{eqnarray}
El estado inicial $\ketm{i}$ y final $\ketm{f}$ del proceso están dados por:
\begin{eqnarray*}
\ketm{i}=\ketm{h_1^+(\vtr{k})}, \quad \ketm{f}=\ketm{\tau^+(\vtr{p},s),\nu_\tau(\vtr{k}',s')}.
\end{eqnarray*}
De la expresión (\ref{s-matriz-1}) se tiene lo siguiente:
\begin{eqnarray*}
:\psi\bar{\psi}\phi:&=&:(\psi_++\psi_-)(\bar{\psi}_++\bar{\psi}_-)(\phi_++\phi_-): \\
&=&:(\psi_+\bar{\psi}_+ +\psi_+\bar{\psi}_- +\psi_-\bar{\psi}_+ +\psi_-\bar{\psi}_-)(\phi_++\phi_-): \\
&=&:\psi_+\bar{\psi}_+\phi_- +\psi_+\bar{\psi}_+\phi_- +\psi_+\bar{\psi}_-\phi_+ + \psi_+\bar{\psi}_-\phi_- +
\psi_-\bar{\psi}_+\phi_+ \\
& & +\psi_-\bar{\psi}_+\phi_- +\psi_-\bar{\psi}_-\phi_+ +\psi_-\bar{\psi}_-\phi_-:
\end{eqnarray*}
y por lo tanto, el único término que contribuye al elemento matricial del proceso es cuestión es:
\begin{eqnarray*}
-i(2f)\sin\varphi\int d^4x\,\psi_-\bar{\psi}_-\phi_+
\end{eqnarray*}

\section{Cálculo del elemento matricial}

Teniendo en cuenta la expresión anterior, se tiene entonces que:
\begin{eqnarray}\label{calculo-1}
S_{fi}^{(1)}&=&-i(2f)\sin\varphi\int
d^4x\,\braket{\tau^+(\vtr{p},s),\nu_\tau(\vtr{k}',s')}{\psi_-\bar{\psi}_-\phi_+}{h_1^+(\vtr{k})},
\end{eqnarray}
y haciendo uso de las expresiones (\ref{accion-operadores-campos-1}) y (\ref{accion-operadores-campos-2}), obtenemos:
\begin{eqnarray}\label{calculo-2}
S_{fi}^{(1)}&=&-i(2f)\sin\varphi v_{s}(\vtr{p})\bar{u}_{s'}(\vtr{k}')\nonumber \\
& &\times \int d^4x\,e^{i(p+k'-k)\cdot
x}\frac{1}{\sqrt{2E_pV}}\frac{1}{\sqrt{2\omega_{k'}V}}\frac{1}{\sqrt{2\omega_kV}}.
\end{eqnarray}
En la integración sobre $x$, se tiene en cuenta la condición $V\rightarrow\infty$, y por lo tanto:
\begin{eqnarray*}
\int d^4x\,e^{i(p+k'-k)\cdot x}=(2\pi)^4\delta^4(k-k'-p),
\end{eqnarray*}
y colocando esto en la expresión (\ref{calculo-2}):
\begin{eqnarray}\label{calculo-3}
S_{fi}^{(1)}&=&\left[-i(2f)\sin\varphi\right]\left[v_s(\vtr{p})\bar{u}_{s'}(\vtr{k}')\right]\left[(2\pi)^4\delta^4(k-k'-p)\right]
 \nonumber \\
& &\times\left [ \frac { 1 } { \sqrt { 2E_pV } } \frac { 1 }{\sqrt{2\omega_{k'}V}}\frac{1}{\sqrt{2\omega_kV}}\right]
\end{eqnarray}


\section{Tasa de decaimiento}
Para una partícula que decae en cierto número de partículas en el estado final, suponiendo que los estados inicial y final son
diferentes, el elemento matricial $S$ se puede escribir como
\begin{eqnarray}\label{elem-matricial-s-gen}
S_{fi}=i(2\pi)^4\delta^4(p_i-\sum_{f}p_f)\frac{1}{\sqrt{2E_iV}}\prod_f\frac{1}{\sqrt{2E_fV}}\mc{M}_{fi},
\end{eqnarray}
siendo $i\mc{M}_{fi}$ la amplitud de Feynmann. Comparando la expresión anterior con la dada en (\ref{calculo-3}), tenemos que:
\begin{eqnarray}\label{amplitud-feynmann-1}
i\mc{M}_{fi}^{(1)}=-i(2f)\sin\varphi\,v_s(\vtr{p})\bar{u}_{s'}(\vtr{k}').
\end{eqnarray}
La probabilidad de transición del estado inicial al estado final está dada por $\left|S_{fi}^{(1)}\right|^2$, luego:
\begin{eqnarray}\label{prob-transicion-1}
\left|S_{fi}^{(1)}\right|^2=(2\pi)^4\delta^4\left(k-k'-p\right)VT\frac{1}{2\omega_kV}\frac{1}{2\omega_{k'}V}\frac{1}{2E_pV}
\left|\mc{M}_{fi}^{(1)}\right|^2.
\end{eqnarray}
La probabilidad de transición por unidad de tiempo se obtiene dividiendo la expresión anterior por $T$, que da como resultado:
\begin{eqnarray}\label{prob-transicion-tiempo-1}
\frac{\left|S_{fi}^{(1)}\right|^2}{T}=(2\pi)^4\delta^4\left(k-k'-p\right)\frac{1}{2\omega_k}\frac{1}{2\omega_{k'}V}\frac{1}{
2E_pV}
\left|\mc{M}_{fi}^{(1)}\right|^2.
\end{eqnarray}
Esta es la probabilidad por unidad de tiempo de obtener un estado final con valores específicos del momento. Cuando se utiliza el
límite $V\rightarrow\infty$, los valores del momento caen en el continuo, por lo que no se buscan valores específicos del
momento, sino más bien por momentos finales en algún rango específico. Esto se realiza discretizando el espacio de fases en
celdas de volumen $(2\pi\hbar)^3=(2\pi)^3$ ($\hbar=1$ en unidades naturales) y colocando un estado en cada celda. Por lo tanto,
el número de estados de una sola partícula en volumen $d^3p$ es el espacio de momentos está dado por:
\begin{eqnarray*}
\frac{V}{(2\pi)^3}d^3p,
\end{eqnarray*}
y por lo tanto, multiplicando por este factor para todos los estados finales e integrando sobre los momentos finales, se obtiene
la tasa de decaimiento $\Gamma$. Por lo tanto, realizando esto con la expresión (\ref{prob-transicion-tiempo-1}):
\begin{eqnarray*}
\Gamma=\frac{1}{2\omega_k}\iint\frac{d^3k'}{(2\pi)^32\omega_{k'}}\frac{d^3p}{(2\pi)^32E_p}
(2\pi)^4\delta^4(k-k'-p)\left|\mc{M}_{fi}^{(1)}\right|^2.
\end{eqnarray*}
Suponiendo que la masa de la partícula que decae, $h_1^+$ es $M$, y considerando el decaimiento en un sistema de referencia en el
cual la partícula inicial está en reposo, entonces $\omega_k=M$, y teniendo en cuenta la expresión (\ref{amplitud-feynmann-1}):
\begin{eqnarray}\label{tasa-decaimiento-1}
\Gamma=\frac{1}{2M}(2f)^2\sin^2\varphi\iint\frac{d^3k'}{(2\pi)^32\omega_{k'}}\frac{d^3p}{(2\pi)^32E_p}
(2\pi)^4\delta^4(k-k'-p)\left|v_s(\vtr{p})\bar{u}_{s'}(\vtr{k}')\right|^2.
\end{eqnarray}
La anterior expresión es la tasa de decaimiento para valores específicos de las proyecciones de espín. Como no estamos
interesados en dichas proyecciones, debemos sumar sobre todas las posibilidades de proyección:
\begin{eqnarray}\label{tasa-decaimiento-2}
\Gamma&=&\frac{1}{2M}(2f)^2\sin^2\varphi\iint\frac{d^3k'}{(2\pi)^32\omega_{k'}}\frac{d^3p}{(2\pi)^32E_p} \nonumber \\
& &\times(2\pi)^4\delta^4(k-k'-p)\sum_{s,s'}\left|v_s(\vtr{p})\bar{u}_{s'}(\vtr{k}')\right|^2.
\end{eqnarray}
Como se demuestra en (citar Palash):
\begin{eqnarray*}
\sum_{\mbox{espin}}=\mbox{Tr}\left[(\gamma_\mu k'^\mu+m_{\nu_\tau})(\gamma_\nu p^nu-m_{\tau^+})\right],
\end{eqnarray*}
y teniendo en cuenta que estamos tomando la masa del neutrino como $m_{\nu_\tau}=0$, entonces:
\begin{eqnarray*}
\sum_{\mbox{espin}}&=&\mbox{Tr}\left[\gamma_\mu k'^\mu(\gamma_\nu p^nu-m_{\tau^+})\right]\\
&=&\mbox{Tr}[\gamma_\mu\gamma_\nu k'^\mu p^\nu-\gamma_\mu k'^\mu m] \\
&=&k'^\mu p^\nu\mbox{Tr}[\gamma_\mu\gamma_\nu]\\
&=&4g_{\mu\nu}k'^\mu p^\nu \\
&=&4k'\cdot p,
\end{eqnarray*}
en donde
\begin{eqnarray*}
k'^\mu=(k'_0,\vtr{k}')=(\omega_{k'},\vtr{k}), \quad p^\mu=(p_0,\vtr{p})=(E_p,\vtr{p}).
\end{eqnarray*}
Debido a la conservación del momento en la delta de Dirac en (\ref{tasa-decaimiento-2}), se tiene que $k=k'+p$, y por lo tanto:
\begin{eqnarray*}
k^2=k'^2+2k'\cdot p+p^2 \quad \rightarrow \quad 2k'\cdot p=k^2-k'^2-p^2,
\end{eqnarray*}
en donde se tiene que $k^2=M$, $p^2=E_p^2-\vtr{p}^2=m^2$ y $k'^2=0$, luego:
\begin{eqnarray*}
2k'\cdot p=M^2-m^2 \quad \rightarrow \quad 4k'\cdot p=2(M^2-m^2),
\end{eqnarray*}
entonces:
\begin{eqnarray*}
\Gamma&=&\frac{1}{2M}(2f)^2\sin^2\varphi
2(M^2-m^2)\iint\frac{d^3k'}{(2\pi)^32\omega_{k'}}\frac{d^3p}{(2\pi)^32E_p}(2\pi)^4\delta^4(k-p-k').
\end{eqnarray*}
Ahora, se tiene que:
\begin{eqnarray*}
\delta^4(k-p-k')&=&\delta(k_0-p_0-k'_0)\delta^3(\vtr{k}-\vtr{p}-\vtr{k}') \\
&=&\delta(M-\omega_{k'}-E_p)\delta^3(\vtr{p}+\vtr{k'}),
\end{eqnarray*}
en donde, como se dijo anteriormente, se está suponiendo la partícula inicial el reposo. Entonces:
\begin{eqnarray*}
\Gamma&=&\frac{1}{M}(2f)^2\sin^2\varphi(M^2-m^2)\iint\frac{d^3k'}{(2\pi)^32\omega_{k'}}\frac{d^3p}{(2\pi)^32E_p}(2\pi)^4\\
& &\times\delta(M-\omega_{ k'}-E_p)\delta^3(\vtr{p}+\vtr{k'}),
\end{eqnarray*}
pero por la delta de Dirac, se tiene que $\vtr{p}=-\vtr{k}'$, luego:
\begin{eqnarray*}
\vtr{p}^2=\vtr{k}'^2=E_p^2-m^2=\omega_{k'}^2, \quad \rightarrow \quad \omega_{k'}=\sqrt{E_p^2-m^2},
\end{eqnarray*}
y por lo tanto:
\begin{eqnarray*}
\Gamma&=&\frac{1}{M}(2f)^2\sin^2\varphi\frac{(M^2-m^2)}{4(2\pi)^2}4\pi\int\frac{dp\,p^2
}{E_p\sqrt{E_p^2-m^2}}\delta(M-\sqrt{E_p^2-m^2} -E_p).
\end{eqnarray*}
Ahora, se tiene que:
\begin{eqnarray*}
dp\,p=dE_p\,E_p \quad \rightarrow \quad dp\,p^2=dE_p\,Ep=dE_p\,Ep=dE_p\,E_p\sqrt{E_p-m^2},
\end{eqnarray*}
y por lo tanto:
\begin{eqnarray*}
\Gamma&=&\frac{1}{M}(2f)^2\sin^2\varphi\frac{(M^2-m^2)}{4(2\pi)^2}4\pi\int\frac{dE_p\,E_p\sqrt{E_p^2-m^2}}{E_p\sqrt{E_p^2-m^2}}
\delta(M-\sqrt { E_p^2-m^2 } -E_p)\\
&=&\frac{1}{M}(2f)^2\sin^2\varphi\frac{M^2-m^2}{8\pi}\int dE_p\,\delta(M-\sqrt{E_p^2-m^2}-E_p).
\end{eqnarray*}
Se tiene que:
\begin{eqnarray*}
\int f(x)\delta[g(x)]\,dx=\sum_{k=0}^N\left[f(x)\left(\detot{g(x)}{x}\right)^{-1}\right]_{x=x_k},
\end{eqnarray*}
en donde $x_k$ son las raíces de la función $g(x)$. Por lo tanto, debemos hallar las raíces de la función
$M-\sqrt{E_p^2-m^2}-E_p$:
\begin{eqnarray*}
M-\sqrt{E_p^2-m^2}-E_p=0 \quad \rightarrow \quad E_p=\frac{M^2+m^2}{2M}.
\end{eqnarray*}
Por lo tanto:
\begin{eqnarray*}
\frac{d}{dE_p}\left(M-\sqrt{E_p^2-m^2}-E_p\right)=-\left(\frac{E_p}{\sqrt{E_p^2-m^2}}+1\right),
\end{eqnarray*}
y evaluando esta derivada en la raíz encontrada:
\begin{eqnarray*}
\int dE_p\,\delta(M-\sqrt{E_p^2-m^2}-E_p)&=&\left[-\left(\frac{M^2+m^2}{\sqrt{2M(M^2+m^2)-4M^2m^2}}+1\right)^{-1}\right],
\end{eqnarray*}
y por lo tanto, se tiene entonces que la tasa de decaimiento está dada por:
\begin{eqnarray}\label{tasa-decaimiento-total}
\Gamma&=&\frac{1}{M}(2f)^2\sin^2\varphi\frac{M^2-m^2}{8\pi} \nonumber \\
& &\times\left[-\left(\frac{M^2+m^2}{\sqrt{2M(M^2+m^2)-4M^2m^2}}+1\right)^{-1} \right].
\end{eqnarray}

\section{Copyright}
\includegraphics[scale=0.5]{cc} Creative Commons Attribution-Share Alike 3.0 United States License.


%%% Local Variables: 
%%% mode: latex
%%% TeX-master: "qft_samples"
%%% End: 

% zee model II:
%
\chapter{$\eta \eta \to h h$}

Author:  Wady Alexander Ríos Herrera

Proceso: \textbf{Radiative seesaw II}\\

Experimentos de reactores  de neutrinos han demostrado que los neutrinos tienen masa.


\section{Lagrangian}


El modelo considerado es una extensión del modelo estándar, el contiene SU($2$)$\times$ U($1$)$_{Y}$ de singletes $N_{i}$ y un segundo doblete de Higgs $\eta$. En adición, una simetría discreta exacta $Z_{2}$ es asumida tal que el nuevo campo son impares bajo $Z_{2}$, mientras que en el modelo estándar son pares. El lagrangiano de interacción de  Yukawa inducido por el nuevo doblete Higgs es dada por  \cite{1}
\begin{equation}
L_{Y}=\epsilon _{ab}h_{\alpha j}\bar{N}_{j}P_{L}L_{\alpha}^{a}\eta ^{b}+H.c,
\label{1}
\end{equation}
 
L son doblete de leptón izquierdo, $\alpha , j$ son indices de generación (Greek indices etiquetado sabor leptonico e,$\mu$,$\tau$) y $\epsilon _{ab}$ es el tensor antisimetrico completo.\\

A partir del lagrangiano de interacción de Yukawa se determina el langrangiano interacción escalar cuadrático  \cite{2} 

\begin{equation}
L_{I}=\lambda_{3}(\Phi^{+}\Phi)( \eta^{+}\eta)+\lambda_{2}(\Phi^{+}\eta)( \eta^{+}\Phi) \frac{\lambda _{5}}{2}(\Phi^{+}\eta + H.C)
\label{2}
\end{equation}


Donde $\Phi$ es el modelo estándar del doblete de Higgs y es solo relevante para generación de masa neutrina. Ya que $Z_{2}$ es asumida  ser  exactamente simétrica del modelo $\eta$ tiene valor esperado del vacío cero. La Física escalar de los bosones son por lo tanto $R_{e}\Phi ^{0},\eta^{\pm},\eta_{0}^{R}\equiv Re\eta_{0}, \eta_{0}^{I}\equiv Im\eta_{0} =0 $ y $Im\eta^{+} =0$.\\


para ecuación \ref{2} se tiene que 
\begin{equation}
\Phi=\begin{pmatrix}
0 & \frac{h+v}{\sqrt{2}}
\end{pmatrix}, \ \eta= \begin{pmatrix}
\eta^{+} & \\
\eta_{0} & 
\end{pmatrix} 
\label{2.1}
\end{equation}

donde $h$ y $v$ son parámetros reales por tanto $\Phi =\Phi ^{+}$


Remplazando \ref{2.1} y las partes reales de $\eta$ el lagrangiano \ref{2} se determina el termina el termino relevante

\begin{equation*}
L_{I}=\lambda_{l} \left[\begin{pmatrix}
0 & \frac{h+v}{\sqrt{2}}
\end{pmatrix}\begin{pmatrix}
\eta^{+} & \\
\eta_{0} & 
\end{pmatrix}\right]^{2}= \lambda_{l}\left[0\eta^{+}+\left( \frac{h+v}{\sqrt{2}}\right)\eta_{0}\right]^{2}
\end{equation*}
\begin{equation*}
=\lambda_{l}(h+v)^{2}(\eta_{0})^{2}=\lambda_{l}(h^{2}+2hv+v^{2})(\eta_{0})^{2}
\end{equation*}
\begin{equation}
=\underbrace{\lambda_{l}h^{2}\eta_{0}^{2}}_{\star}+2\lambda_{l}hv\eta_{0}^{2}+\lambda_{l}v^{2}\eta_{0}^{2}, 
\label{3}
\end{equation}

Con 
\begin{equation}
\lambda_{l}=\frac{\lambda _{3}+\lambda _{4}+\lambda _{5}}{2}
\end{equation}

donde el termino relevante de la expresión \ref{3} es denotado con $\star$, ya que este genera el siguiente proceso 

\begin{figure*}[htb]
    \centering
    \includegraphics[width=0.5\textwidth]{1}%(eps preferiblemente)
    \caption[electrones]{Proceso donde dos etas se destruyen y se crean dos Higgs}
    \label{fi}
\end{figure*}
 



Por tanto el lagrangiano de interés es
\begin{equation}
L_{i}=\lambda_{l}h^{2}\eta_{0}^{2}
\label{4}
\end{equation}










\section{S-matriz}
 El orden a calcular la matriz $S$ es de orden uno ya que solo tenemos un termino de interacción \ref{4} el cual  determina el proceso de figuara \ref{fi}.\\
 Consideremos el proceso de la figura \ref{fi} donde $\eta(p_{1})\eta(p_{2})$ decae a un par de $h(p_{1}^{'})h(p_{2}^{'})$.
 El elemento de matriz $S_{fi}^{1}$ entre el estado inicial  y el estado final es 
 
 \begin{equation}
S_{fi}^{(1)}=-\lambda_{l}\int d^{4}x\bra{h(p_{1}^{'})h(p_{2}^{'})}h^{2}\eta_{0}^{2}\ket{\eta(p_{1})\eta(p_{2})}, 
 \label{5}
 \end{equation}
 
 sabemos que
 
 \begin{equation}
\eta_{0}=\eta_{+}+\eta_{-}=\frac{1}{\sqrt{2E_{p}V}}\left[ae^{-ip.x}+a^{+}e^{ip.x}\right]
 \label{6}
 \end{equation}

\begin{equation}
h=h_{+}+h_{-}=\frac{1}{\sqrt{2E_{p}V}}\left[ae^{-ip.x}+a^{+}e^{ip.x}\right]
 \label{7}
 \end{equation}


Por lo tanto
\begin{equation*}
h^{2}=\left(h_{+}+h_{-}\right)^{2}=(h_{+}+h_{-})(h_{+}+h_{-})
\end{equation*}
 \begin{equation}
=h_{+}h_{+}+h_{+}h_{-}+h_{-}h_{+}+h_{-}h_{-}
 \label{8}
 \end{equation}

\begin{equation*}
\eta_{0}^{2}=\left(\eta_{+}+\eta_{-}\right)^{2}=(\eta_{+}+\eta_{-})(\eta_{+}+\eta_{-})
\end{equation*}
\begin{equation}
=\eta_{+}\eta_{+}+\eta_{+}\eta_{-}+\eta_{-}\eta_{+}+\eta_{-}\eta_{-}
\label{9}
\end{equation}

El termino $\frac{1}{\sqrt{2E_{p}V}}$ de la expresión \ref{6} y \ref{7} es de normalización.\\

Calculemos  el termino $h^{2}\eta_{0}^{2}$:
\begin{equation*}
h^{2}\eta_{0}^{2}=(h_{+}h_{+}+h_{+}h_{-}+h_{-}h_{+}+h_{-}h_{-})(\eta_{+}\eta_{+}+\eta_{+}\eta_{-}+\eta_{-}\eta_{+}+\eta_{-}\eta_{-})
\end{equation*}
\begin{equation*}
=h_{+}h_{+}\eta_{+}\eta_{+}+h_{+}h_{+}\eta_{+}\eta_{-}+h_{+}h_{+}\eta_{-}\eta_{+}+h_{+}h_{+}\eta_{-}\eta_{-}+
\end{equation*}
\begin{equation*}
h_{+}h_{-}\eta_{+}\eta_{+}+h_{+}h_{-}\eta_{+}\eta_{-}+h_{+}h_{-}\eta_{-}\eta_{+}+h_{+}h_{-}\eta_{-}\eta_{-}+
\end{equation*}
\begin{equation*}
h_{-}h_{+}\eta_{+}\eta_{+}+h_{-}h_{+}\eta_{+}\eta_{-}+h_{-}h_{+}\eta_{-}\eta_{+}+h_{-}h_{+}\eta_{-}\eta_{-}+
\end{equation*}
\begin{equation}
h_{-}h_{-}\eta_{+}\eta_{+}+h_{-}h_{-}\eta_{+}\eta_{-}+h_{-}h_{-}\eta_{-}\eta_{+}+h_{-}h_{-}\eta_{-}\eta_{-}
\label{10}
\end{equation}
 Por otro lado  sabemos que 
 
 \begin{equation}
 h_{+}\ket{h}=\frac{1}{\sqrt{2E_{p}V}}e^{-ip.x}\ket0 ,\ h_{-}\ket{h}=\frac{1}{\sqrt{2E_{p}V}}e^{ip.x}\ket{1_{h}}
 \label{11}
 \end{equation}
 \begin{equation}
 \bra h h_{-}=\frac{1}{\sqrt{2E_{p}V}}e^{ip.x}\bra{0} ,\ \bra h h_{+}=\frac{1}{\sqrt{2E_{p}V}}e^{-ip.x}\bra{1_{h}}
 \label{12}
 \end{equation}


\begin{equation}
 \eta_{+}\ket{\eta}=\frac{1}{\sqrt{2E_{p}V}}e^{-ip.x}\ket0 ,\ \eta_{-}\ket{\eta}=\frac{1}{\sqrt{2E_{p}V}}e^{ip.x}\ket{1_{\eta}}
 \label{13}
 \end{equation}
 \begin{equation}
 \bra \eta \eta_{-}=\frac{1}{\sqrt{2E_{p}V}}e^{ip.x}\bra{0} ,\ \bra \eta \eta_{+}=\frac{1}{\sqrt{2E_{p}V}}e^{-ip.x}\bra {1_{\eta}}
 \label{14}
 \end{equation}
 
 y
 
 \begin{equation}
\bra{h(p_{1}^{'})h(p_{2}^{'})}h_{+}h_{+}=\frac{1}{\sqrt{2E_{p_{1}^{'}}V}}\frac{1}{\sqrt{2E_{p_{2}^{'}}V}}e^{-ip_{1}^{'}.x}e^{-ip_{2}^{'}.x}\bra{1_{h}1_{h}}
 \label{15}
 \end{equation}
\begin{equation}
\bra{h(p_{1}^{'})h(p_{2}^{'})}h_{+}h_{-}=\frac{1}{\sqrt{2E_{p_{1}^{'}}V}}\frac{1}{\sqrt{2E_{p_{2}^{'}}V}}e^{-ip_{1}^{'}.x}e^{ip_{2}^{'}.x}\bra{1_{h}0}
 \label{16}
 \end{equation}
\begin{equation}
\bra{h(p_{1}^{'})h(p_{2}^{'})}h_{-}h_{+}=\frac{1}{\sqrt{2E_{p_{1}^{'}}V}}\frac{1}{\sqrt{2E_{p_{2}^{'}}V}}e^{ip_{1}^{'}.x}e^{-ip_{2}^{'}.x}\bra{01_{h}}
 \label{17}
 \end{equation}
\begin{equation}
\bra{h(p_{1}^{'})h(p_{2}^{'})}h_{-}h_{-}=\frac{1}{\sqrt{2E_{p_{1}^{'}}V}}\frac{1}{\sqrt{2E_{p_{2}^{'}}V}}e^{ip_{1}^{'}.x}e^{ip_{2}^{'}.x}\bra{00}
 \label{18}
 \end{equation}

\begin{equation}
\eta_{+}\eta_{+}\ket{\eta(p_{1})\eta(p_{2})}=\frac{1}{\sqrt{2E_{p_{1}}V}}\frac{1}{\sqrt{2E_{p_{2}}V}}e^{-ip_{1}.x}e^{-ip_{2}.x}\ket{00}
 \label{19}
 \end{equation}


\begin{equation}
\eta_{+}\eta_{-}\ket{\eta(p_{1})\eta(p_{2})}=\frac{1}{\sqrt{2E_{p_{1}}V}}\frac{1}{\sqrt{2E_{p_{2}}V}}e^{-ip_{1}.x}e^{ip_{2}.x}\ket{01_{\eta}}
 \label{20}
 \end{equation}
\begin{equation}
\eta_{-}\eta_{+}\ket{\eta(p_{1})\eta(p_{2})}=\frac{1}{\sqrt{2E_{p_{1}}V}}\frac{1}{\sqrt{2E_{p_{2}}V}}e^{ip_{1}.x}e^{-ip_{2}.x}\ket{1_{\eta}0}
 \label{21}
 \end{equation}
\begin{equation}
\eta_{-}\eta_{-}\ket{\eta(p_{1})\eta(p_{2})}=\frac{1}{\sqrt{2E_{p_{1}}V}}\frac{1}{\sqrt{2E_{p_{2}}V}}e^{ip_{1}.x}e^{ip_{2}.x}\ket{1_{\eta}1_{\eta}}
 \label{22}
 \end{equation}

Por tanto san duchando el termino $h^{2}\eta_{0}^{2}$ con las relaciones desde \ref{15} hasta \ref{22} el termino que sobrevive es


\begin{equation*}
\bra{h(p_{1}^{'})h(p_{2}^{'})}h^{2}\eta_{0}^{2}\ket{\eta(p_{1})\eta(p_{2})}=\bra{h(p_{1}^{'})h(p_{2}^{'})}h_{-}h_{-}\eta_{+}\eta_{+}\ket{\eta(p_{1})\eta(p_{2})}
\end{equation*}
\begin{equation}
=\frac{1}{\sqrt{2E_{p_{1}^{'}}V}}\frac{1}{\sqrt{2E_{p_{2}^{'}}V}}\frac{1}{\sqrt{2E_{p_{1}}V}}\frac{1}{\sqrt{2E_{p_{2}}V}}e^{i(p_{1}^{'}+p_{2}^{'}-p_{1}-p_{2})}\bra{00}\ket{00}
\label{23}
\end{equation}

remplazando \ref{23} en matriz $S_{fi}^{(1)}$ obtenemos

\begin{equation}
S_{fi}^{(1)}=-i\lambda_{l}\frac{1}{\sqrt{2E_{p_{1}^{'}}V}}\frac{1}{\sqrt{2E_{p_{2}^{'}}V}}\frac{1}{\sqrt{2E_{p_{1}}V}}\frac{1}{\sqrt{2E_{p_{2}}V}}\int d^{4}x e^{i(p_{1}^{'}+p_{2}^{'}-p_{1}-p_{2})}\bra{00}\ket{00}
\label{24}
\end{equation}


El valor de la integral es $(2\pi)^{4}\delta^{4}(p_{1}+p_{2}-p_{1}^{'}-p_{2}^{'})$ y $\bra{00}\ket{00}=1$ por tanto el elemento de matriz obtenido es

\begin{equation}
S_{fi}^{(1)}=-i\lambda_{l}\frac{1}{\sqrt{2E_{p_{1}^{'}}V}}\frac{1}{\sqrt{2E_{p_{2}^{'}}V}}\frac{1}{\sqrt{2E_{p_{1}}V}}\frac{1}{\sqrt{2E_{p_{2}}V}}(2\pi)^{4}\delta^{4}(p_{1}+p_{2}-p_{1}^{'}-p_{2}^{'})
\label{25}
\end{equation}



\section{Process calculation}

 De la ecuación \ref{25} la función $\delta$  es la conservación de los $4$-momentos en el proceso general, el cual es multiplicado por $(2\pi)^{4}$. Hay un factor $(2EV)^{\frac{-1}{2} }$ para cada partícula del estado inicial y del estado final de energía $E$. El restos de termino de la expresión \ref{25} el cual depende sobre la naturaleza exacta de la interacción es llamada la amplitud de Feynman, es denotada por $iM_{fi}$.\\
 Para nuestro caso  
 \begin{equation}
iM_{fi}= -i\lambda_{l}
  \label{26}
 \end{equation}
              
   Para determinar la sección eficaz de nuestro modelo se determina a partir de expresión\cite{3}
   
   \begin{equation}
\frac{d\sigma}{d\Omega}=\frac{1}{64 \small{\pi} ^{2}s}\left[\frac{(s-(m_{1}^{'}+m_{2}^{'})^{2})(s-(m_{1}^{'}-m_{2}^{'})^{2}))}{(s-(m_{1}+m_{2})^{2})(s-(m_{1}-m_{2})^{2}))}\right]^{\frac{1}{2}}\overline{\vert M_{fi}\vert^{2}}
   \label{27}
   \end{equation}            
              
     Donde $m_{1}^{'}=m_{2}^{'}=m_{h}$ es la masa del Higgs, $m_{1}=m_{2}=m_{\eta}$ es la masa del eta y  $\overline{\vert M_{fi}\vert^{2}}=\lambda_{l}$ es la amplitud de Feynman, remplazando estos términos en la expresión \ref{27} se obtiene
     
     \begin{equation}
      \frac{d\sigma}{d\Omega}=\frac{1}{64\small{\pi} ^{2}s}\left[\frac{(s-4m_{h}^{2})}{(s-4m_{\eta}^{2})}\right]^{\frac{1}{2}}\left(\lambda_{l}\right)^{2}
          \label{28}
     \end{equation}
     
     
Con $s=\sqrt{E_{1}+E_{2}}$ y $E=\sqrt{p^{2}+m^{2}}$, la integral sobre $d\Omega$ es $4\pi$ por tanto la sección eficaz es 


\begin{equation}
  \sigma=\frac{(\lambda_{l})^{2}}{64\small{\pi} s}\left[\frac{(s-4m_{h}^{2})}{(s-4m_{\eta}^{2})}\right]^{\frac{1}{2}}
  \label{29}
\end{equation}  
          
     
     
Para determinar un valor númerico de la sección eficaz tomamos los siguientes valores: $ m_{\eta}=50GeV, m_{h}=120 GeV, \lambda_{l}=0.1$ y $\sqrt{s}=1016.68GeV $, Por tanto la expresión \ref{29} que así

\begin{equation*}
\sigma=\frac{(0.1)^{2}}{64\small{\pi} (1016.68GeV)^{2}}\left[\frac{(1016.68GeV)^{2}-4(120GeV)^{2}}{(1016.68GeV)^{2}-4(50GeV)^{2}}\right]^{\frac{1}{2}}
\label{30}GeV
\end{equation*}
     \begin{equation*}
    \frac{(0.1)^{2}}{64\small{\pi} (1016.68GeV)^{2}}\left[\frac{976055.3134 (GeV)^{2}}{1023638.232(GeV)^{2}}\right]^{\frac{1}{2}}
     \end{equation*}
              
\begin{equation*}
   = \frac{(0.1)^{2}}{64\small{\pi} (1016.68GeV)^{2}}(0,9764)
     \end{equation*}   
     \begin{equation}
    =4,69\times 10^{-11}(GeV)^{-2}
     \end{equation}              
              
\section{CalcHEP comparison}
Para ser la simulación se instala los programas de LanHep y CalHep.\\
Después se baja los archivos del proceso a simular de github, nuestro archivo pertinente es la carpeta idm para nuestro sistema. Se  instala la carpeta idm  en LanHep y CalHep y se ejecuta  CalHep desde el directorio idm para iniciar la simulación.\\
Los parámetros de la simulación son $~H0,~H0->H,H$ donde se descoge el diagrama de la figura \ref{fi} luego se ingresa los parámetros de la simulación que son $\lambda_{l}=0.1, m_{\eta}=50 GeV, m_{h}=120 GeV$, momento inicial $500 GeV$ y momento final es de $500GeV$ por tanto al sección eficaz de la simulación es de $0.145983[pb]$  

\section{Copyright}
\includegraphics[scale=0.5]{cc} Creative Commons Attribution-Share Alike 3.0 United States License.

\begin{thebibliography}{9}

\bibitem{1} D. Aristizabal Sierra, Jisuke Kubo and Daijiro Suematsu, Physical Review D 79, 013001 (2009)
\bibitem{2} Ernast Ma, Physical Review D 73, 0077301(2006)
\bibitem{3} Quantum Fiel Theory, Amitabha Lahiri and Palash B. Pal

\end{thebibliography}

%%% Local Variables: 
%%% mode: latex
%%% TeX-master: "qft_samples"
%%% End: 

% THDM I:  Yubian hep-ph/0504050
%
\chapter{$\eta \eta \to h h$}

Author:  Wady Alexander Ríos Herrera

Proceso: \textbf{Radiative seesaw II}\\

Experimentos de reactores  de neutrinos han demostrado que los neutrinos tienen masa.


\section{Lagrangian}


El modelo considerado es una extensión del modelo estándar, el contiene SU($2$)$\times$ U($1$)$_{Y}$ de singletes $N_{i}$ y un segundo doblete de Higgs $\eta$. En adición, una simetría discreta exacta $Z_{2}$ es asumida tal que el nuevo campo son impares bajo $Z_{2}$, mientras que en el modelo estándar son pares. El lagrangiano de interacción de  Yukawa inducido por el nuevo doblete Higgs es dada por  \cite{1}
\begin{equation}
L_{Y}=\epsilon _{ab}h_{\alpha j}\bar{N}_{j}P_{L}L_{\alpha}^{a}\eta ^{b}+H.c,
\label{1}
\end{equation}
 
L son doblete de leptón izquierdo, $\alpha , j$ son indices de generación (Greek indices etiquetado sabor leptonico e,$\mu$,$\tau$) y $\epsilon _{ab}$ es el tensor antisimetrico completo.\\

A partir del lagrangiano de interacción de Yukawa se determina el langrangiano interacción escalar cuadrático  \cite{2} 

\begin{equation}
L_{I}=\lambda_{3}(\Phi^{+}\Phi)( \eta^{+}\eta)+\lambda_{2}(\Phi^{+}\eta)( \eta^{+}\Phi) \frac{\lambda _{5}}{2}(\Phi^{+}\eta + H.C)
\label{2}
\end{equation}


Donde $\Phi$ es el modelo estándar del doblete de Higgs y es solo relevante para generación de masa neutrina. Ya que $Z_{2}$ es asumida  ser  exactamente simétrica del modelo $\eta$ tiene valor esperado del vacío cero. La Física escalar de los bosones son por lo tanto $R_{e}\Phi ^{0},\eta^{\pm},\eta_{0}^{R}\equiv Re\eta_{0}, \eta_{0}^{I}\equiv Im\eta_{0} =0 $ y $Im\eta^{+} =0$.\\


para ecuación \ref{2} se tiene que 
\begin{equation}
\Phi=\begin{pmatrix}
0 & \frac{h+v}{\sqrt{2}}
\end{pmatrix}, \ \eta= \begin{pmatrix}
\eta^{+} & \\
\eta_{0} & 
\end{pmatrix} 
\label{2.1}
\end{equation}

donde $h$ y $v$ son parámetros reales por tanto $\Phi =\Phi ^{+}$


Remplazando \ref{2.1} y las partes reales de $\eta$ el lagrangiano \ref{2} se determina el termina el termino relevante

\begin{equation*}
L_{I}=\lambda_{l} \left[\begin{pmatrix}
0 & \frac{h+v}{\sqrt{2}}
\end{pmatrix}\begin{pmatrix}
\eta^{+} & \\
\eta_{0} & 
\end{pmatrix}\right]^{2}= \lambda_{l}\left[0\eta^{+}+\left( \frac{h+v}{\sqrt{2}}\right)\eta_{0}\right]^{2}
\end{equation*}
\begin{equation*}
=\lambda_{l}(h+v)^{2}(\eta_{0})^{2}=\lambda_{l}(h^{2}+2hv+v^{2})(\eta_{0})^{2}
\end{equation*}
\begin{equation}
=\underbrace{\lambda_{l}h^{2}\eta_{0}^{2}}_{\star}+2\lambda_{l}hv\eta_{0}^{2}+\lambda_{l}v^{2}\eta_{0}^{2}, 
\label{3}
\end{equation}

Con 
\begin{equation}
\lambda_{l}=\frac{\lambda _{3}+\lambda _{4}+\lambda _{5}}{2}
\end{equation}

donde el termino relevante de la expresión \ref{3} es denotado con $\star$, ya que este genera el siguiente proceso 

\begin{figure*}[htb]
    \centering
    \includegraphics[width=0.5\textwidth]{1}%(eps preferiblemente)
    \caption[electrones]{Proceso donde dos etas se destruyen y se crean dos Higgs}
    \label{fi}
\end{figure*}
 



Por tanto el lagrangiano de interés es
\begin{equation}
L_{i}=\lambda_{l}h^{2}\eta_{0}^{2}
\label{4}
\end{equation}










\section{S-matriz}
 El orden a calcular la matriz $S$ es de orden uno ya que solo tenemos un termino de interacción \ref{4} el cual  determina el proceso de figuara \ref{fi}.\\
 Consideremos el proceso de la figura \ref{fi} donde $\eta(p_{1})\eta(p_{2})$ decae a un par de $h(p_{1}^{'})h(p_{2}^{'})$.
 El elemento de matriz $S_{fi}^{1}$ entre el estado inicial  y el estado final es 
 
 \begin{equation}
S_{fi}^{(1)}=-\lambda_{l}\int d^{4}x\bra{h(p_{1}^{'})h(p_{2}^{'})}h^{2}\eta_{0}^{2}\ket{\eta(p_{1})\eta(p_{2})}, 
 \label{5}
 \end{equation}
 
 sabemos que
 
 \begin{equation}
\eta_{0}=\eta_{+}+\eta_{-}=\frac{1}{\sqrt{2E_{p}V}}\left[ae^{-ip.x}+a^{+}e^{ip.x}\right]
 \label{6}
 \end{equation}

\begin{equation}
h=h_{+}+h_{-}=\frac{1}{\sqrt{2E_{p}V}}\left[ae^{-ip.x}+a^{+}e^{ip.x}\right]
 \label{7}
 \end{equation}


Por lo tanto
\begin{equation*}
h^{2}=\left(h_{+}+h_{-}\right)^{2}=(h_{+}+h_{-})(h_{+}+h_{-})
\end{equation*}
 \begin{equation}
=h_{+}h_{+}+h_{+}h_{-}+h_{-}h_{+}+h_{-}h_{-}
 \label{8}
 \end{equation}

\begin{equation*}
\eta_{0}^{2}=\left(\eta_{+}+\eta_{-}\right)^{2}=(\eta_{+}+\eta_{-})(\eta_{+}+\eta_{-})
\end{equation*}
\begin{equation}
=\eta_{+}\eta_{+}+\eta_{+}\eta_{-}+\eta_{-}\eta_{+}+\eta_{-}\eta_{-}
\label{9}
\end{equation}

El termino $\frac{1}{\sqrt{2E_{p}V}}$ de la expresión \ref{6} y \ref{7} es de normalización.\\

Calculemos  el termino $h^{2}\eta_{0}^{2}$:
\begin{equation*}
h^{2}\eta_{0}^{2}=(h_{+}h_{+}+h_{+}h_{-}+h_{-}h_{+}+h_{-}h_{-})(\eta_{+}\eta_{+}+\eta_{+}\eta_{-}+\eta_{-}\eta_{+}+\eta_{-}\eta_{-})
\end{equation*}
\begin{equation*}
=h_{+}h_{+}\eta_{+}\eta_{+}+h_{+}h_{+}\eta_{+}\eta_{-}+h_{+}h_{+}\eta_{-}\eta_{+}+h_{+}h_{+}\eta_{-}\eta_{-}+
\end{equation*}
\begin{equation*}
h_{+}h_{-}\eta_{+}\eta_{+}+h_{+}h_{-}\eta_{+}\eta_{-}+h_{+}h_{-}\eta_{-}\eta_{+}+h_{+}h_{-}\eta_{-}\eta_{-}+
\end{equation*}
\begin{equation*}
h_{-}h_{+}\eta_{+}\eta_{+}+h_{-}h_{+}\eta_{+}\eta_{-}+h_{-}h_{+}\eta_{-}\eta_{+}+h_{-}h_{+}\eta_{-}\eta_{-}+
\end{equation*}
\begin{equation}
h_{-}h_{-}\eta_{+}\eta_{+}+h_{-}h_{-}\eta_{+}\eta_{-}+h_{-}h_{-}\eta_{-}\eta_{+}+h_{-}h_{-}\eta_{-}\eta_{-}
\label{10}
\end{equation}
 Por otro lado  sabemos que 
 
 \begin{equation}
 h_{+}\ket{h}=\frac{1}{\sqrt{2E_{p}V}}e^{-ip.x}\ket0 ,\ h_{-}\ket{h}=\frac{1}{\sqrt{2E_{p}V}}e^{ip.x}\ket{1_{h}}
 \label{11}
 \end{equation}
 \begin{equation}
 \bra h h_{-}=\frac{1}{\sqrt{2E_{p}V}}e^{ip.x}\bra{0} ,\ \bra h h_{+}=\frac{1}{\sqrt{2E_{p}V}}e^{-ip.x}\bra{1_{h}}
 \label{12}
 \end{equation}


\begin{equation}
 \eta_{+}\ket{\eta}=\frac{1}{\sqrt{2E_{p}V}}e^{-ip.x}\ket0 ,\ \eta_{-}\ket{\eta}=\frac{1}{\sqrt{2E_{p}V}}e^{ip.x}\ket{1_{\eta}}
 \label{13}
 \end{equation}
 \begin{equation}
 \bra \eta \eta_{-}=\frac{1}{\sqrt{2E_{p}V}}e^{ip.x}\bra{0} ,\ \bra \eta \eta_{+}=\frac{1}{\sqrt{2E_{p}V}}e^{-ip.x}\bra {1_{\eta}}
 \label{14}
 \end{equation}
 
 y
 
 \begin{equation}
\bra{h(p_{1}^{'})h(p_{2}^{'})}h_{+}h_{+}=\frac{1}{\sqrt{2E_{p_{1}^{'}}V}}\frac{1}{\sqrt{2E_{p_{2}^{'}}V}}e^{-ip_{1}^{'}.x}e^{-ip_{2}^{'}.x}\bra{1_{h}1_{h}}
 \label{15}
 \end{equation}
\begin{equation}
\bra{h(p_{1}^{'})h(p_{2}^{'})}h_{+}h_{-}=\frac{1}{\sqrt{2E_{p_{1}^{'}}V}}\frac{1}{\sqrt{2E_{p_{2}^{'}}V}}e^{-ip_{1}^{'}.x}e^{ip_{2}^{'}.x}\bra{1_{h}0}
 \label{16}
 \end{equation}
\begin{equation}
\bra{h(p_{1}^{'})h(p_{2}^{'})}h_{-}h_{+}=\frac{1}{\sqrt{2E_{p_{1}^{'}}V}}\frac{1}{\sqrt{2E_{p_{2}^{'}}V}}e^{ip_{1}^{'}.x}e^{-ip_{2}^{'}.x}\bra{01_{h}}
 \label{17}
 \end{equation}
\begin{equation}
\bra{h(p_{1}^{'})h(p_{2}^{'})}h_{-}h_{-}=\frac{1}{\sqrt{2E_{p_{1}^{'}}V}}\frac{1}{\sqrt{2E_{p_{2}^{'}}V}}e^{ip_{1}^{'}.x}e^{ip_{2}^{'}.x}\bra{00}
 \label{18}
 \end{equation}

\begin{equation}
\eta_{+}\eta_{+}\ket{\eta(p_{1})\eta(p_{2})}=\frac{1}{\sqrt{2E_{p_{1}}V}}\frac{1}{\sqrt{2E_{p_{2}}V}}e^{-ip_{1}.x}e^{-ip_{2}.x}\ket{00}
 \label{19}
 \end{equation}


\begin{equation}
\eta_{+}\eta_{-}\ket{\eta(p_{1})\eta(p_{2})}=\frac{1}{\sqrt{2E_{p_{1}}V}}\frac{1}{\sqrt{2E_{p_{2}}V}}e^{-ip_{1}.x}e^{ip_{2}.x}\ket{01_{\eta}}
 \label{20}
 \end{equation}
\begin{equation}
\eta_{-}\eta_{+}\ket{\eta(p_{1})\eta(p_{2})}=\frac{1}{\sqrt{2E_{p_{1}}V}}\frac{1}{\sqrt{2E_{p_{2}}V}}e^{ip_{1}.x}e^{-ip_{2}.x}\ket{1_{\eta}0}
 \label{21}
 \end{equation}
\begin{equation}
\eta_{-}\eta_{-}\ket{\eta(p_{1})\eta(p_{2})}=\frac{1}{\sqrt{2E_{p_{1}}V}}\frac{1}{\sqrt{2E_{p_{2}}V}}e^{ip_{1}.x}e^{ip_{2}.x}\ket{1_{\eta}1_{\eta}}
 \label{22}
 \end{equation}

Por tanto san duchando el termino $h^{2}\eta_{0}^{2}$ con las relaciones desde \ref{15} hasta \ref{22} el termino que sobrevive es


\begin{equation*}
\bra{h(p_{1}^{'})h(p_{2}^{'})}h^{2}\eta_{0}^{2}\ket{\eta(p_{1})\eta(p_{2})}=\bra{h(p_{1}^{'})h(p_{2}^{'})}h_{-}h_{-}\eta_{+}\eta_{+}\ket{\eta(p_{1})\eta(p_{2})}
\end{equation*}
\begin{equation}
=\frac{1}{\sqrt{2E_{p_{1}^{'}}V}}\frac{1}{\sqrt{2E_{p_{2}^{'}}V}}\frac{1}{\sqrt{2E_{p_{1}}V}}\frac{1}{\sqrt{2E_{p_{2}}V}}e^{i(p_{1}^{'}+p_{2}^{'}-p_{1}-p_{2})}\bra{00}\ket{00}
\label{23}
\end{equation}

remplazando \ref{23} en matriz $S_{fi}^{(1)}$ obtenemos

\begin{equation}
S_{fi}^{(1)}=-i\lambda_{l}\frac{1}{\sqrt{2E_{p_{1}^{'}}V}}\frac{1}{\sqrt{2E_{p_{2}^{'}}V}}\frac{1}{\sqrt{2E_{p_{1}}V}}\frac{1}{\sqrt{2E_{p_{2}}V}}\int d^{4}x e^{i(p_{1}^{'}+p_{2}^{'}-p_{1}-p_{2})}\bra{00}\ket{00}
\label{24}
\end{equation}


El valor de la integral es $(2\pi)^{4}\delta^{4}(p_{1}+p_{2}-p_{1}^{'}-p_{2}^{'})$ y $\bra{00}\ket{00}=1$ por tanto el elemento de matriz obtenido es

\begin{equation}
S_{fi}^{(1)}=-i\lambda_{l}\frac{1}{\sqrt{2E_{p_{1}^{'}}V}}\frac{1}{\sqrt{2E_{p_{2}^{'}}V}}\frac{1}{\sqrt{2E_{p_{1}}V}}\frac{1}{\sqrt{2E_{p_{2}}V}}(2\pi)^{4}\delta^{4}(p_{1}+p_{2}-p_{1}^{'}-p_{2}^{'})
\label{25}
\end{equation}



\section{Process calculation}

 De la ecuación \ref{25} la función $\delta$  es la conservación de los $4$-momentos en el proceso general, el cual es multiplicado por $(2\pi)^{4}$. Hay un factor $(2EV)^{\frac{-1}{2} }$ para cada partícula del estado inicial y del estado final de energía $E$. El restos de termino de la expresión \ref{25} el cual depende sobre la naturaleza exacta de la interacción es llamada la amplitud de Feynman, es denotada por $iM_{fi}$.\\
 Para nuestro caso  
 \begin{equation}
iM_{fi}= -i\lambda_{l}
  \label{26}
 \end{equation}
              
   Para determinar la sección eficaz de nuestro modelo se determina a partir de expresión\cite{3}
   
   \begin{equation}
\frac{d\sigma}{d\Omega}=\frac{1}{64 \small{\pi} ^{2}s}\left[\frac{(s-(m_{1}^{'}+m_{2}^{'})^{2})(s-(m_{1}^{'}-m_{2}^{'})^{2}))}{(s-(m_{1}+m_{2})^{2})(s-(m_{1}-m_{2})^{2}))}\right]^{\frac{1}{2}}\overline{\vert M_{fi}\vert^{2}}
   \label{27}
   \end{equation}            
              
     Donde $m_{1}^{'}=m_{2}^{'}=m_{h}$ es la masa del Higgs, $m_{1}=m_{2}=m_{\eta}$ es la masa del eta y  $\overline{\vert M_{fi}\vert^{2}}=\lambda_{l}$ es la amplitud de Feynman, remplazando estos términos en la expresión \ref{27} se obtiene
     
     \begin{equation}
      \frac{d\sigma}{d\Omega}=\frac{1}{64\small{\pi} ^{2}s}\left[\frac{(s-4m_{h}^{2})}{(s-4m_{\eta}^{2})}\right]^{\frac{1}{2}}\left(\lambda_{l}\right)^{2}
          \label{28}
     \end{equation}
     
     
Con $s=\sqrt{E_{1}+E_{2}}$ y $E=\sqrt{p^{2}+m^{2}}$, la integral sobre $d\Omega$ es $4\pi$ por tanto la sección eficaz es 


\begin{equation}
  \sigma=\frac{(\lambda_{l})^{2}}{64\small{\pi} s}\left[\frac{(s-4m_{h}^{2})}{(s-4m_{\eta}^{2})}\right]^{\frac{1}{2}}
  \label{29}
\end{equation}  
          
     
     
Para determinar un valor númerico de la sección eficaz tomamos los siguientes valores: $ m_{\eta}=50GeV, m_{h}=120 GeV, \lambda_{l}=0.1$ y $\sqrt{s}=1016.68GeV $, Por tanto la expresión \ref{29} que así

\begin{equation*}
\sigma=\frac{(0.1)^{2}}{64\small{\pi} (1016.68GeV)^{2}}\left[\frac{(1016.68GeV)^{2}-4(120GeV)^{2}}{(1016.68GeV)^{2}-4(50GeV)^{2}}\right]^{\frac{1}{2}}
\label{30}GeV
\end{equation*}
     \begin{equation*}
    \frac{(0.1)^{2}}{64\small{\pi} (1016.68GeV)^{2}}\left[\frac{976055.3134 (GeV)^{2}}{1023638.232(GeV)^{2}}\right]^{\frac{1}{2}}
     \end{equation*}
              
\begin{equation*}
   = \frac{(0.1)^{2}}{64\small{\pi} (1016.68GeV)^{2}}(0,9764)
     \end{equation*}   
     \begin{equation}
    =4,69\times 10^{-11}(GeV)^{-2}
     \end{equation}              
              
\section{CalcHEP comparison}
Para ser la simulación se instala los programas de LanHep y CalHep.\\
Después se baja los archivos del proceso a simular de github, nuestro archivo pertinente es la carpeta idm para nuestro sistema. Se  instala la carpeta idm  en LanHep y CalHep y se ejecuta  CalHep desde el directorio idm para iniciar la simulación.\\
Los parámetros de la simulación son $~H0,~H0->H,H$ donde se descoge el diagrama de la figura \ref{fi} luego se ingresa los parámetros de la simulación que son $\lambda_{l}=0.1, m_{\eta}=50 GeV, m_{h}=120 GeV$, momento inicial $500 GeV$ y momento final es de $500GeV$ por tanto al sección eficaz de la simulación es de $0.145983[pb]$  

\section{Copyright}
\includegraphics[scale=0.5]{cc} Creative Commons Attribution-Share Alike 3.0 United States License.

\begin{thebibliography}{9}

\bibitem{1} D. Aristizabal Sierra, Jisuke Kubo and Daijiro Suematsu, Physical Review D 79, 013001 (2009)
\bibitem{2} Ernast Ma, Physical Review D 73, 0077301(2006)
\bibitem{3} Quantum Fiel Theory, Amitabha Lahiri and Palash B. Pal

\end{thebibliography}

%%% Local Variables: 
%%% mode: latex
%%% TeX-master: "qft_samples"
%%% End: 

% THDM II: Oscar Andrés hep-ph/0504050
%
\chapter{$\eta \eta \to h h$}

Author:  Wady Alexander Ríos Herrera

Proceso: \textbf{Radiative seesaw II}\\

Experimentos de reactores  de neutrinos han demostrado que los neutrinos tienen masa.


\section{Lagrangian}


El modelo considerado es una extensión del modelo estándar, el contiene SU($2$)$\times$ U($1$)$_{Y}$ de singletes $N_{i}$ y un segundo doblete de Higgs $\eta$. En adición, una simetría discreta exacta $Z_{2}$ es asumida tal que el nuevo campo son impares bajo $Z_{2}$, mientras que en el modelo estándar son pares. El lagrangiano de interacción de  Yukawa inducido por el nuevo doblete Higgs es dada por  \cite{1}
\begin{equation}
L_{Y}=\epsilon _{ab}h_{\alpha j}\bar{N}_{j}P_{L}L_{\alpha}^{a}\eta ^{b}+H.c,
\label{1}
\end{equation}
 
L son doblete de leptón izquierdo, $\alpha , j$ son indices de generación (Greek indices etiquetado sabor leptonico e,$\mu$,$\tau$) y $\epsilon _{ab}$ es el tensor antisimetrico completo.\\

A partir del lagrangiano de interacción de Yukawa se determina el langrangiano interacción escalar cuadrático  \cite{2} 

\begin{equation}
L_{I}=\lambda_{3}(\Phi^{+}\Phi)( \eta^{+}\eta)+\lambda_{2}(\Phi^{+}\eta)( \eta^{+}\Phi) \frac{\lambda _{5}}{2}(\Phi^{+}\eta + H.C)
\label{2}
\end{equation}


Donde $\Phi$ es el modelo estándar del doblete de Higgs y es solo relevante para generación de masa neutrina. Ya que $Z_{2}$ es asumida  ser  exactamente simétrica del modelo $\eta$ tiene valor esperado del vacío cero. La Física escalar de los bosones son por lo tanto $R_{e}\Phi ^{0},\eta^{\pm},\eta_{0}^{R}\equiv Re\eta_{0}, \eta_{0}^{I}\equiv Im\eta_{0} =0 $ y $Im\eta^{+} =0$.\\


para ecuación \ref{2} se tiene que 
\begin{equation}
\Phi=\begin{pmatrix}
0 & \frac{h+v}{\sqrt{2}}
\end{pmatrix}, \ \eta= \begin{pmatrix}
\eta^{+} & \\
\eta_{0} & 
\end{pmatrix} 
\label{2.1}
\end{equation}

donde $h$ y $v$ son parámetros reales por tanto $\Phi =\Phi ^{+}$


Remplazando \ref{2.1} y las partes reales de $\eta$ el lagrangiano \ref{2} se determina el termina el termino relevante

\begin{equation*}
L_{I}=\lambda_{l} \left[\begin{pmatrix}
0 & \frac{h+v}{\sqrt{2}}
\end{pmatrix}\begin{pmatrix}
\eta^{+} & \\
\eta_{0} & 
\end{pmatrix}\right]^{2}= \lambda_{l}\left[0\eta^{+}+\left( \frac{h+v}{\sqrt{2}}\right)\eta_{0}\right]^{2}
\end{equation*}
\begin{equation*}
=\lambda_{l}(h+v)^{2}(\eta_{0})^{2}=\lambda_{l}(h^{2}+2hv+v^{2})(\eta_{0})^{2}
\end{equation*}
\begin{equation}
=\underbrace{\lambda_{l}h^{2}\eta_{0}^{2}}_{\star}+2\lambda_{l}hv\eta_{0}^{2}+\lambda_{l}v^{2}\eta_{0}^{2}, 
\label{3}
\end{equation}

Con 
\begin{equation}
\lambda_{l}=\frac{\lambda _{3}+\lambda _{4}+\lambda _{5}}{2}
\end{equation}

donde el termino relevante de la expresión \ref{3} es denotado con $\star$, ya que este genera el siguiente proceso 

\begin{figure*}[htb]
    \centering
    \includegraphics[width=0.5\textwidth]{1}%(eps preferiblemente)
    \caption[electrones]{Proceso donde dos etas se destruyen y se crean dos Higgs}
    \label{fi}
\end{figure*}
 



Por tanto el lagrangiano de interés es
\begin{equation}
L_{i}=\lambda_{l}h^{2}\eta_{0}^{2}
\label{4}
\end{equation}










\section{S-matriz}
 El orden a calcular la matriz $S$ es de orden uno ya que solo tenemos un termino de interacción \ref{4} el cual  determina el proceso de figuara \ref{fi}.\\
 Consideremos el proceso de la figura \ref{fi} donde $\eta(p_{1})\eta(p_{2})$ decae a un par de $h(p_{1}^{'})h(p_{2}^{'})$.
 El elemento de matriz $S_{fi}^{1}$ entre el estado inicial  y el estado final es 
 
 \begin{equation}
S_{fi}^{(1)}=-\lambda_{l}\int d^{4}x\bra{h(p_{1}^{'})h(p_{2}^{'})}h^{2}\eta_{0}^{2}\ket{\eta(p_{1})\eta(p_{2})}, 
 \label{5}
 \end{equation}
 
 sabemos que
 
 \begin{equation}
\eta_{0}=\eta_{+}+\eta_{-}=\frac{1}{\sqrt{2E_{p}V}}\left[ae^{-ip.x}+a^{+}e^{ip.x}\right]
 \label{6}
 \end{equation}

\begin{equation}
h=h_{+}+h_{-}=\frac{1}{\sqrt{2E_{p}V}}\left[ae^{-ip.x}+a^{+}e^{ip.x}\right]
 \label{7}
 \end{equation}


Por lo tanto
\begin{equation*}
h^{2}=\left(h_{+}+h_{-}\right)^{2}=(h_{+}+h_{-})(h_{+}+h_{-})
\end{equation*}
 \begin{equation}
=h_{+}h_{+}+h_{+}h_{-}+h_{-}h_{+}+h_{-}h_{-}
 \label{8}
 \end{equation}

\begin{equation*}
\eta_{0}^{2}=\left(\eta_{+}+\eta_{-}\right)^{2}=(\eta_{+}+\eta_{-})(\eta_{+}+\eta_{-})
\end{equation*}
\begin{equation}
=\eta_{+}\eta_{+}+\eta_{+}\eta_{-}+\eta_{-}\eta_{+}+\eta_{-}\eta_{-}
\label{9}
\end{equation}

El termino $\frac{1}{\sqrt{2E_{p}V}}$ de la expresión \ref{6} y \ref{7} es de normalización.\\

Calculemos  el termino $h^{2}\eta_{0}^{2}$:
\begin{equation*}
h^{2}\eta_{0}^{2}=(h_{+}h_{+}+h_{+}h_{-}+h_{-}h_{+}+h_{-}h_{-})(\eta_{+}\eta_{+}+\eta_{+}\eta_{-}+\eta_{-}\eta_{+}+\eta_{-}\eta_{-})
\end{equation*}
\begin{equation*}
=h_{+}h_{+}\eta_{+}\eta_{+}+h_{+}h_{+}\eta_{+}\eta_{-}+h_{+}h_{+}\eta_{-}\eta_{+}+h_{+}h_{+}\eta_{-}\eta_{-}+
\end{equation*}
\begin{equation*}
h_{+}h_{-}\eta_{+}\eta_{+}+h_{+}h_{-}\eta_{+}\eta_{-}+h_{+}h_{-}\eta_{-}\eta_{+}+h_{+}h_{-}\eta_{-}\eta_{-}+
\end{equation*}
\begin{equation*}
h_{-}h_{+}\eta_{+}\eta_{+}+h_{-}h_{+}\eta_{+}\eta_{-}+h_{-}h_{+}\eta_{-}\eta_{+}+h_{-}h_{+}\eta_{-}\eta_{-}+
\end{equation*}
\begin{equation}
h_{-}h_{-}\eta_{+}\eta_{+}+h_{-}h_{-}\eta_{+}\eta_{-}+h_{-}h_{-}\eta_{-}\eta_{+}+h_{-}h_{-}\eta_{-}\eta_{-}
\label{10}
\end{equation}
 Por otro lado  sabemos que 
 
 \begin{equation}
 h_{+}\ket{h}=\frac{1}{\sqrt{2E_{p}V}}e^{-ip.x}\ket0 ,\ h_{-}\ket{h}=\frac{1}{\sqrt{2E_{p}V}}e^{ip.x}\ket{1_{h}}
 \label{11}
 \end{equation}
 \begin{equation}
 \bra h h_{-}=\frac{1}{\sqrt{2E_{p}V}}e^{ip.x}\bra{0} ,\ \bra h h_{+}=\frac{1}{\sqrt{2E_{p}V}}e^{-ip.x}\bra{1_{h}}
 \label{12}
 \end{equation}


\begin{equation}
 \eta_{+}\ket{\eta}=\frac{1}{\sqrt{2E_{p}V}}e^{-ip.x}\ket0 ,\ \eta_{-}\ket{\eta}=\frac{1}{\sqrt{2E_{p}V}}e^{ip.x}\ket{1_{\eta}}
 \label{13}
 \end{equation}
 \begin{equation}
 \bra \eta \eta_{-}=\frac{1}{\sqrt{2E_{p}V}}e^{ip.x}\bra{0} ,\ \bra \eta \eta_{+}=\frac{1}{\sqrt{2E_{p}V}}e^{-ip.x}\bra {1_{\eta}}
 \label{14}
 \end{equation}
 
 y
 
 \begin{equation}
\bra{h(p_{1}^{'})h(p_{2}^{'})}h_{+}h_{+}=\frac{1}{\sqrt{2E_{p_{1}^{'}}V}}\frac{1}{\sqrt{2E_{p_{2}^{'}}V}}e^{-ip_{1}^{'}.x}e^{-ip_{2}^{'}.x}\bra{1_{h}1_{h}}
 \label{15}
 \end{equation}
\begin{equation}
\bra{h(p_{1}^{'})h(p_{2}^{'})}h_{+}h_{-}=\frac{1}{\sqrt{2E_{p_{1}^{'}}V}}\frac{1}{\sqrt{2E_{p_{2}^{'}}V}}e^{-ip_{1}^{'}.x}e^{ip_{2}^{'}.x}\bra{1_{h}0}
 \label{16}
 \end{equation}
\begin{equation}
\bra{h(p_{1}^{'})h(p_{2}^{'})}h_{-}h_{+}=\frac{1}{\sqrt{2E_{p_{1}^{'}}V}}\frac{1}{\sqrt{2E_{p_{2}^{'}}V}}e^{ip_{1}^{'}.x}e^{-ip_{2}^{'}.x}\bra{01_{h}}
 \label{17}
 \end{equation}
\begin{equation}
\bra{h(p_{1}^{'})h(p_{2}^{'})}h_{-}h_{-}=\frac{1}{\sqrt{2E_{p_{1}^{'}}V}}\frac{1}{\sqrt{2E_{p_{2}^{'}}V}}e^{ip_{1}^{'}.x}e^{ip_{2}^{'}.x}\bra{00}
 \label{18}
 \end{equation}

\begin{equation}
\eta_{+}\eta_{+}\ket{\eta(p_{1})\eta(p_{2})}=\frac{1}{\sqrt{2E_{p_{1}}V}}\frac{1}{\sqrt{2E_{p_{2}}V}}e^{-ip_{1}.x}e^{-ip_{2}.x}\ket{00}
 \label{19}
 \end{equation}


\begin{equation}
\eta_{+}\eta_{-}\ket{\eta(p_{1})\eta(p_{2})}=\frac{1}{\sqrt{2E_{p_{1}}V}}\frac{1}{\sqrt{2E_{p_{2}}V}}e^{-ip_{1}.x}e^{ip_{2}.x}\ket{01_{\eta}}
 \label{20}
 \end{equation}
\begin{equation}
\eta_{-}\eta_{+}\ket{\eta(p_{1})\eta(p_{2})}=\frac{1}{\sqrt{2E_{p_{1}}V}}\frac{1}{\sqrt{2E_{p_{2}}V}}e^{ip_{1}.x}e^{-ip_{2}.x}\ket{1_{\eta}0}
 \label{21}
 \end{equation}
\begin{equation}
\eta_{-}\eta_{-}\ket{\eta(p_{1})\eta(p_{2})}=\frac{1}{\sqrt{2E_{p_{1}}V}}\frac{1}{\sqrt{2E_{p_{2}}V}}e^{ip_{1}.x}e^{ip_{2}.x}\ket{1_{\eta}1_{\eta}}
 \label{22}
 \end{equation}

Por tanto san duchando el termino $h^{2}\eta_{0}^{2}$ con las relaciones desde \ref{15} hasta \ref{22} el termino que sobrevive es


\begin{equation*}
\bra{h(p_{1}^{'})h(p_{2}^{'})}h^{2}\eta_{0}^{2}\ket{\eta(p_{1})\eta(p_{2})}=\bra{h(p_{1}^{'})h(p_{2}^{'})}h_{-}h_{-}\eta_{+}\eta_{+}\ket{\eta(p_{1})\eta(p_{2})}
\end{equation*}
\begin{equation}
=\frac{1}{\sqrt{2E_{p_{1}^{'}}V}}\frac{1}{\sqrt{2E_{p_{2}^{'}}V}}\frac{1}{\sqrt{2E_{p_{1}}V}}\frac{1}{\sqrt{2E_{p_{2}}V}}e^{i(p_{1}^{'}+p_{2}^{'}-p_{1}-p_{2})}\bra{00}\ket{00}
\label{23}
\end{equation}

remplazando \ref{23} en matriz $S_{fi}^{(1)}$ obtenemos

\begin{equation}
S_{fi}^{(1)}=-i\lambda_{l}\frac{1}{\sqrt{2E_{p_{1}^{'}}V}}\frac{1}{\sqrt{2E_{p_{2}^{'}}V}}\frac{1}{\sqrt{2E_{p_{1}}V}}\frac{1}{\sqrt{2E_{p_{2}}V}}\int d^{4}x e^{i(p_{1}^{'}+p_{2}^{'}-p_{1}-p_{2})}\bra{00}\ket{00}
\label{24}
\end{equation}


El valor de la integral es $(2\pi)^{4}\delta^{4}(p_{1}+p_{2}-p_{1}^{'}-p_{2}^{'})$ y $\bra{00}\ket{00}=1$ por tanto el elemento de matriz obtenido es

\begin{equation}
S_{fi}^{(1)}=-i\lambda_{l}\frac{1}{\sqrt{2E_{p_{1}^{'}}V}}\frac{1}{\sqrt{2E_{p_{2}^{'}}V}}\frac{1}{\sqrt{2E_{p_{1}}V}}\frac{1}{\sqrt{2E_{p_{2}}V}}(2\pi)^{4}\delta^{4}(p_{1}+p_{2}-p_{1}^{'}-p_{2}^{'})
\label{25}
\end{equation}



\section{Process calculation}

 De la ecuación \ref{25} la función $\delta$  es la conservación de los $4$-momentos en el proceso general, el cual es multiplicado por $(2\pi)^{4}$. Hay un factor $(2EV)^{\frac{-1}{2} }$ para cada partícula del estado inicial y del estado final de energía $E$. El restos de termino de la expresión \ref{25} el cual depende sobre la naturaleza exacta de la interacción es llamada la amplitud de Feynman, es denotada por $iM_{fi}$.\\
 Para nuestro caso  
 \begin{equation}
iM_{fi}= -i\lambda_{l}
  \label{26}
 \end{equation}
              
   Para determinar la sección eficaz de nuestro modelo se determina a partir de expresión\cite{3}
   
   \begin{equation}
\frac{d\sigma}{d\Omega}=\frac{1}{64 \small{\pi} ^{2}s}\left[\frac{(s-(m_{1}^{'}+m_{2}^{'})^{2})(s-(m_{1}^{'}-m_{2}^{'})^{2}))}{(s-(m_{1}+m_{2})^{2})(s-(m_{1}-m_{2})^{2}))}\right]^{\frac{1}{2}}\overline{\vert M_{fi}\vert^{2}}
   \label{27}
   \end{equation}            
              
     Donde $m_{1}^{'}=m_{2}^{'}=m_{h}$ es la masa del Higgs, $m_{1}=m_{2}=m_{\eta}$ es la masa del eta y  $\overline{\vert M_{fi}\vert^{2}}=\lambda_{l}$ es la amplitud de Feynman, remplazando estos términos en la expresión \ref{27} se obtiene
     
     \begin{equation}
      \frac{d\sigma}{d\Omega}=\frac{1}{64\small{\pi} ^{2}s}\left[\frac{(s-4m_{h}^{2})}{(s-4m_{\eta}^{2})}\right]^{\frac{1}{2}}\left(\lambda_{l}\right)^{2}
          \label{28}
     \end{equation}
     
     
Con $s=\sqrt{E_{1}+E_{2}}$ y $E=\sqrt{p^{2}+m^{2}}$, la integral sobre $d\Omega$ es $4\pi$ por tanto la sección eficaz es 


\begin{equation}
  \sigma=\frac{(\lambda_{l})^{2}}{64\small{\pi} s}\left[\frac{(s-4m_{h}^{2})}{(s-4m_{\eta}^{2})}\right]^{\frac{1}{2}}
  \label{29}
\end{equation}  
          
     
     
Para determinar un valor númerico de la sección eficaz tomamos los siguientes valores: $ m_{\eta}=50GeV, m_{h}=120 GeV, \lambda_{l}=0.1$ y $\sqrt{s}=1016.68GeV $, Por tanto la expresión \ref{29} que así

\begin{equation*}
\sigma=\frac{(0.1)^{2}}{64\small{\pi} (1016.68GeV)^{2}}\left[\frac{(1016.68GeV)^{2}-4(120GeV)^{2}}{(1016.68GeV)^{2}-4(50GeV)^{2}}\right]^{\frac{1}{2}}
\label{30}GeV
\end{equation*}
     \begin{equation*}
    \frac{(0.1)^{2}}{64\small{\pi} (1016.68GeV)^{2}}\left[\frac{976055.3134 (GeV)^{2}}{1023638.232(GeV)^{2}}\right]^{\frac{1}{2}}
     \end{equation*}
              
\begin{equation*}
   = \frac{(0.1)^{2}}{64\small{\pi} (1016.68GeV)^{2}}(0,9764)
     \end{equation*}   
     \begin{equation}
    =4,69\times 10^{-11}(GeV)^{-2}
     \end{equation}              
              
\section{CalcHEP comparison}
Para ser la simulación se instala los programas de LanHep y CalHep.\\
Después se baja los archivos del proceso a simular de github, nuestro archivo pertinente es la carpeta idm para nuestro sistema. Se  instala la carpeta idm  en LanHep y CalHep y se ejecuta  CalHep desde el directorio idm para iniciar la simulación.\\
Los parámetros de la simulación son $~H0,~H0->H,H$ donde se descoge el diagrama de la figura \ref{fi} luego se ingresa los parámetros de la simulación que son $\lambda_{l}=0.1, m_{\eta}=50 GeV, m_{h}=120 GeV$, momento inicial $500 GeV$ y momento final es de $500GeV$ por tanto al sección eficaz de la simulación es de $0.145983[pb]$  

\section{Copyright}
\includegraphics[scale=0.5]{cc} Creative Commons Attribution-Share Alike 3.0 United States License.

\begin{thebibliography}{9}

\bibitem{1} D. Aristizabal Sierra, Jisuke Kubo and Daijiro Suematsu, Physical Review D 79, 013001 (2009)
\bibitem{2} Ernast Ma, Physical Review D 73, 0077301(2006)
\bibitem{3} Quantum Fiel Theory, Amitabha Lahiri and Palash B. Pal

\end{thebibliography}

%%% Local Variables: 
%%% mode: latex
%%% TeX-master: "qft_samples"
%%% End: 



%\include{sdm}



\end{document}


%%% Local Variables: 
%%% mode: latex
%%% TeX-master: "qft_samples"
%%% End: 
