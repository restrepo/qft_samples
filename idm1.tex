
\chapter{$H_{0}H_{0}\rightarrow b\bar{b}$}

Author: Jehison Alexander Monsalve Zapata

\section*{Descripcion}

El modelo donde se agrega un higgs mas al modelo estandar con unas
simetrias particulares tal que $H_{0}$sea estable, el proceso que
voy a calcular es $H_{0}H_{0}\rightarrow b\bar{b}$


\section*{Langragiano}

$\mathcal{L}_{int}=\mathcal{L}_{sm}+\lambda_{3}H_{1}^{\dagger}H_{1}H_{2}^{\text{\ensuremath{\dagger}}}H_{2}+\lambda_{4}H_{1}^{\dagger}H_{2}H_{2}^{\dagger}H_{1}+\dfrac{\lambda_{5}}{2}\left[\left(H_{1}^{\dagger}H_{2}\right)^{2}+\left(H_{2}^{\dagger}H_{1}\right)^{2}\right]$

donde $H_{1}=\left[\begin{array}{c}
0\\
\dfrac{h+v}{\sqrt{2}}\end{array}\right]$ en el gauge unitario

y $H_{2}=\left[\begin{array}{c}
H^{+}\\
\dfrac{H^{0}+\imath A^{0}}{\sqrt{2}}\end{array}\right]$

ahora miremos el primer termino

\[
\lambda_{3}H_{1}^{\dagger}H_{1}H_{2}^{\text{\ensuremath{\dagger}}}H_{2}=\lambda_{3}\left[\begin{array}{cc}
0 & \dfrac{h+v}{\sqrt{2}}\end{array}\right]\left[\begin{array}{c}
0\\
\dfrac{h+v}{\sqrt{2}}\end{array}\right]\left[\begin{array}{cc}
H^{-} & \dfrac{H^{0}-\imath A^{0}}{\sqrt{2}}\end{array}\right]\left[\begin{array}{c}
H^{+}\\
\dfrac{H^{0}+\imath A^{0}}{\sqrt{2}}\end{array}\right]\]


						$=\dfrac{\lambda_{3}}{4}(h+v)^{2}(H^{-}H^{+}+(H^{0})^{2}+(A^{0})^{2})=\dfrac{\lambda_{3}}{4}(h^{2}+2vh+v^{2})(H^{-}H^{+}+H^{0}H^{0}+A^{0}A^{0})$

del cual solo nos interesa $\dfrac{\lambda_{3}}{4}2vhH^{0}H^{0}=\dfrac{\lambda_{3}}{2}vhH^{0}H^{0}$

miremos los terminos con $\lambda_{4}$

\[
\lambda_{4}H_{1}^{\dagger}H_{2}H_{2}^{\dagger}H_{1}=\lambda_{4}\left[\begin{array}{cc}
0 & \dfrac{h+v}{\sqrt{2}}\end{array}\right]\left[\begin{array}{c}
H^{+}\\
\dfrac{H^{0}+\imath A^{0}}{\sqrt{2}}\end{array}\right]\left[\begin{array}{cc}
H^{-} & \dfrac{H^{0}-\imath A^{0}}{\sqrt{2}}\end{array}\right]\left[\begin{array}{c}
0\\
\dfrac{h+v}{\sqrt{2}}\end{array}\right]\]


\[
=\dfrac{\lambda_{4}}{4}(h+v)(H^{0}+\imath A^{0})(H^{0}-\imath A^{0})(h+v)\]


ya que son bosones y cumple reglas de conmutacion

\[
=\dfrac{\lambda_{4}}{4}(h+v)^{2}\left[(H^{0})^{2}+(A^{0})^{2}\right]\]


de aqui sacamos el termico $\dfrac{\lambda_{4}}{4}2vhH^{0}H^{0}$

desarrollemos los terminos que acompañan a $\lambda_{5}$

\[
\left(H_{1}^{\dagger}H_{2}\right)^{2}+\left(H_{2}^{\dagger}H_{\text{1}}\right)^{2}=\left(\left[\begin{array}{cc}
0 & \dfrac{h+v}{\sqrt{2}}\end{array}\right]\left[\begin{array}{c}
H^{+}\\
\dfrac{H^{0}+\imath A^{0}}{\sqrt{2}}\end{array}\right]\right)^{2}+\left(\left[\begin{array}{cc}
H^{-} & \dfrac{H^{0}-\imath A^{0}}{\sqrt{2}}\end{array}\right]\left[\begin{array}{c}
0\\
\dfrac{h+v}{\sqrt{2}}\end{array}\right]\right)^{2}\]


\[
\propto\left(h(H^{0}+\imath A^{0})+v(H^{0}+iA^{0})\right)^{2}+\left((H^{0}-\imath A^{0})h+v(H^{0}-\imath A^{0})\right)^{2}\]


el factor de proporcionalidad es $\dfrac{1}{4}$

\[
=h^{2}(H^{0}+\imath A^{0})^{2}+2vh(H^{0}+\imath A^{0})^{2}+h^{2}(H^{0}-\imath A^{0})^{2}+2hv(H^{0}-\imath A^{0})^{2}+v^{2}(\ldots)\]


de los cuales solos nos interesas los terminos con $hH^{0}H^{0}$

es decir $2vhH^{0}H^{0}+2vhH^{0}H^{0}=4vhH^{0}H^{0}$

ahora retomando las constantes tenemos que el termino del lagragiano
que nos intereas es $\dfrac{\lambda_{5}}{2}vhH^{0}H^{0}$

ahora ponemos todo junto y el lagrangiano de interacion es 

\[
\mathcal{L_{\textrm{int}}}=v\dfrac{\lambda_{3}+\lambda_{4}+\lambda_{5}}{2}hH^{0}H^{0}=v\dfrac{\lambda_{L}}{2}hH^{0}H^{0}\]


junto con la interacion del higgs con los fermiones tenemos 

\[
\mathcal{L_{\textrm{int}\textrm{Total}}}=v\dfrac{\lambda_{L}}{2}hH^{0}H^{0}-k\overline{\psi}\psi h\]



\section*{Matriz S}

Solo nos interesa orden 2 en la matriz s ya que el proceso que calcularemos
lo requiere de ese modo entonces 

aplicanado la definicicon\[
S^{(2)}=\dfrac{(-\imath)^{2}}{2!}\int d^{4}x_{1}\int d^{4}x_{2}\mathcal{J}\left[\mathcal{H}(x_{1})\mathcal{H}(x_{2})\right]\]


de donde $\mathcal{H}(x)=\mathcal{H}_{1}(x)+\mathcal{H}_{2}(x)$ y
$\mathcal{H}_{1}(x)=-v\dfrac{\lambda_{L}}{2}hH^{0}H^{0}$, $\mathcal{H}_{2}(x)=k\overline{\psi}\psi h$

haciendo el producto 

$\mathcal{H}(x_{1})\mathcal{H}(x_{2})=\mathcal{H}_{1}(x_{1})\mathcal{H}_{1}(x_{2})+\mathcal{H}_{1}(x_{1})\mathcal{H}_{2}(x_{2})+\mathcal{H}_{2}(x_{1})\mathcal{H}_{1}(x_{2})+\mathcal{H}_{2}(x_{1})\mathcal{H}_{2}(x_{2})$

y para el proceso de que estamos calculando el unico termino distinto
de cero es $\mathcal{H}_{2}(x_{1})\mathcal{H}_{1}(x_{2})$

sustituyendo en la matriz s 

\[
S^{(2)}=\dfrac{(-\imath)^{2}}{2!}\int d^{4}x_{1}\int d^{4}x_{2}\mathcal{J}\left[\mathcal{H}_{2}(x_{1})\mathcal{H}_{1}(x_{2})\right]\]


\[
S^{(2)}=\dfrac{-kv\lambda_{L}}{4}\int d^{4}x_{1}\int d^{4}x_{2}\mathcal{J}\left[(:\overline{\psi}\psi h:)_{x_{1}}(:hH^{0}H^{0}:)_{x_{2}}\right]\]


ya que la unica contracion es entre el $h(x_{1})$ con el $h(x_{2})$ 

\[
S^{(2)}=\dfrac{-kv\lambda_{L}}{4}\int d^{4}x_{1}\int d^{4}x_{2}(:\bar{\psi}\psi H^{0}H^{0}:)\imath\Delta_{F}(x_{1}-x_{2})\]


descomponiendo los campo en operdores de creacion y destruccion tenemos

\[
S^{(2)}=\dfrac{-kv\lambda_{L}}{4}\int d^{4}x_{1}\int d^{4}x_{2}\imath\Delta_{F}(x_{1}-x_{2}):\bar{(\psi}_{+}+\bar{\psi}_{-})(\psi_{+}+\psi_{-})(H_{+}^{0}+H_{-}^{0})(H_{+}^{0}+H_{-}^{0}):\]


los unicos terminos que nos son cero son aquellos tales que $\langle b^{-}(p,s)b^{+}(p',s')\vert ABCD\vert H^{0}(k)H^{0}(k')\rangle=\langle0\vert0\rangle$

entonces los unicos que cumplen esto son $\bar{\psi}_{-},\psi_{-},H_{+}^{0},H_{+}^{0}$y
asi nos queda

\[
S^{(2)}=\dfrac{-kv\lambda_{L}}{4}\int d^{4}x_{1}\int d^{4}x_{2}\imath\Delta_{F}(x_{1}-x_{2}):\bar{\psi}_{-}\psi_{-}H_{+}^{0}H_{+}^{0}:\]


ahora vamos a tratar esto separadamente para hacer mas comoda la deduccion 

el estado inicial son dos higgs cero $\vert H^{0}(k)H^{0}(k')\rangle=\dfrac{1}{V}a^{\dagger}(k)a^{\dagger}(k')\vert0\rangle$

el estod final es un boton y un antiboton $\vert b^{-}(p,s)b^{+}(p',s')\rangle=\dfrac{1}{V}f^{\dagger}(p,s)g^{\dagger}(p',s')\vert0\rangle$

\begin{eqnarray*}
\bar{\psi}_{+}\psi_{+}\vert b^{-}(p,s)b^{+}(p',s')\rangle & = & \int d^{3}q_{1}\int d^{3}q_{2}\dfrac{1}{(2\pi)^{6}\sqrt{4E(q_{1})E(q_{2})}}u^{\alpha}(q_{1})\bar{v}^{\beta}(q_{2})f(q_{1})e^{-\imath q_{1}\cdot x_{2}}g(q_{2})e^{-\imath q_{2}\cdot x_{2}}\\
 &  & \dfrac{1}{V}f^{\dagger}(p,s)g^{\dagger}(p',s')\vert0\rangle\end{eqnarray*}


podemos sumar $f^{\dagger}(p,s)g^{\dagger}(p',s')f(q_{1})g(q_{2})\vert0\rangle=0$

con lo cual nos queda el termino

$f(q_{1})g(q_{2})f^{\dagger}(p,s)g^{\dagger}(p',s')+f^{\dagger}(p,s)g^{\dagger}(p',s')f(q_{1})g(q_{2})=\left\{ f(q_{1})g(q_{2}),f^{\dagger}(p,s)g^{\dagger}(p',s')\right\} $

y tenemos una propiedad de los anticomutadores 

$\left\{ AB,CD\right\} =A\left\{ B,C\right\} D-\left\{ A,C\right\} BD+CA\left\{ B,D\right\} -C\left\{ A,D\right\} B$

utilizandola

\begin{eqnarray*}
\left\{ fg,f^{\dagger}g^{\dagger}\right\}  & = & f\left\{ g,f^{\dagger}\right\} g^{\dagger}-\left\{ f,f^{\dagger}\right\} gg^{\dagger}+f^{\dagger}f\left\{ g,g^{\dagger}\right\} -f^{\dagger}\left\{ f,g^{\dagger}\right\} g\\
 &  & =-\left\{ f,f^{\dagger}\right\} gg^{\dagger}+f^{\dagger}f\left\{ g,g^{\dagger}\right\} \end{eqnarray*}


ya que los otros anticonmutadores son nulos 

\begin{multline*}
\left\{ f(q_{1})g(q_{2}),f^{\dagger}(p)g^{\dagger}(p')\right\} =-\delta^{3}(q_{1}-p)g(q_{1})g^{\dagger}(p')+f^{\dagger}(p)f(p)\delta^{3}(q_{2}-p')\end{multline*}


\begin{multline*}
f^{\dagger}f\left\{ g,g^{\dagger}\right\} \vert0\rangle=0\end{multline*}


luego aplicando las reglas de anticomutacion entonces tenemos que 

\begin{multline*}
\left\{ f(q_{1})g(q_{2}),f^{\dagger}(p)g^{\dagger}(p')\right\} =-\delta^{3}(q_{1}-p)\delta^{3}(q_{2}-p')\end{multline*}


luego 

$\bar{\psi}_{+}\psi_{+}\vert b^{-}(p,s)b^{+}(p',s')\rangle=-\int d^{3}q_{1}\int d^{3}q_{2}\dfrac{1}{(2\pi)^{6}V\sqrt{4E(q_{1})E(q_{2})}}u^{\alpha}(q_{1})\bar{v}^{\beta}(q_{2})e^{-\imath q_{1}\cdot x_{2}}e^{-\imath q_{2}\cdot x_{2}}\delta^{3}(q_{1}-p)\delta^{3}(q_{2}-p')\vert0\rangle$

$\bar{\psi}_{+}\psi_{+}\vert b^{-}(p,s)b^{+}(p',s')\rangle=-\dfrac{1}{(2\pi)^{6}V\sqrt{4E(q_{1})E(q_{2})}}u^{\alpha}(p)\bar{v}^{\beta}(p')e^{-\imath p\cdot x_{2}}e^{-\imath p'\cdot x_{2}}\vert0\rangle$

\begin{eqnarray*}
H_{+}^{0}(x_{1})H_{+}^{0}(x_{1})\vert H^{0}(k)H^{0}(k')\rangle & = & \int d^{3}q_{1}\int d^{3}q_{2}\dfrac{1}{(2\pi)^{6}\sqrt{4E(q_{1})E(q_{2})}}a(q_{1})e^{-\imath q_{1}\cdot x_{1}}a(q_{2})e^{-\imath q_{2}\cdot x_{2}}\\
 &  & \dfrac{1}{V}a^{\dagger}(k)a^{\dagger}(k')\vert0\rangle\end{eqnarray*}


desarrollemos $a(q_{1})a(q_{2})a^{\dagger}(k)a^{\dagger}(k')\vert0\rangle=a(q_{1})a(q_{2})a^{\dagger}(k)a^{\dagger}(k')\vert0\rangle-a^{\dagger}(k)a^{\dagger}(k')a(q_{1})a(q_{2})\vert0\rangle$

y nos queda $a(q_{1})a(q_{2})a^{\dagger}(k)a^{\dagger}(k')\vert0\rangle=\left[a(q_{1})a(q_{2}),a^{\dagger}(k)a^{\dagger}(k')\right]\vert0\rangle$

por identidad de conmutadores $\left[AB,CD\right]=\left[A,C\right]BD+C\left[A,D\right]B+A\left[B,C\right]D+CA\left[B,D\right]$

\[
\left[a_{q_{1}}a_{q_{2}},a_{k}^{\dagger}a_{k'}^{\dagger}\right]=\left[a_{q_{1}},a_{k}^{\dagger}\right]a_{q_{2}}a_{k'}^{\dagger}+a_{k}^{\dagger}\left[a_{q_{1}},a_{k'}^{\dagger}\right]a_{q_{2}}+a_{q_{1}}\left[a_{q_{2}},a_{k}^{\dagger}\right]a_{k'}^{\dagger}+a_{k}^{\dagger}a_{q_{1}}\left[a_{q_{2}},a_{k'}^{\dagger}\right]\]


\[
\left[a_{q_{1}}a_{q_{2}},a_{k}^{\dagger}a_{k'}^{\dagger}\right]=\delta^{3}(q_{1}-k)a_{q_{2}}a_{k'}^{\dagger}+a_{k}^{\dagger}\delta^{3}(q_{1}-k')a_{q_{2}}+a_{q_{1}}\delta^{3}(q_{2}-k)a_{k'}^{\dagger}+a_{k}^{\dagger}a_{q_{1}}\delta^{3}(q_{2}-k')\]


y como dos son cero aplicandolos al vacio

\[
\left[a_{q_{1}}a_{q_{2}},a_{k}^{\dagger}a_{k'}^{\dagger}\right]=\delta^{3}(q_{1}-k)a_{q_{2}}a_{k'}^{\dagger}+\delta^{3}(q_{2}-k)a_{q_{1}}a_{k'}^{\dagger}\]


aplicando el mismo metodo tenemos 

\[
\left[a_{q_{1}}a_{q_{2}},a_{k}^{\dagger}a_{k'}^{\dagger}\right]=\delta^{3}(q_{1}-k)\delta^{3}(q_{2}-k')+\delta^{3}(q_{2}-k)\delta^{3}(q_{1}-k')\]


lugo 

\[
H_{+}^{0}(x_{1})H_{+}^{0}(x_{1})\vert H^{0}(k)H^{0}(k')\rangle=\dfrac{2}{(2\pi)^{6}V\sqrt{4E(k)E(k')}}e^{-\imath k\cdot x_{1}}e^{-\imath k'\cdot x_{1}}\vert0\rangle\]


\[
S_{fi}^{(2)}=\dfrac{-kv\lambda_{L}}{4}\int d^{4}x_{1}\int d^{4}x_{2}\imath\Delta_{F}(x_{1}-x_{2})\langle\bar{b^{-}(p,s)b^{+}(p',s')\vert\psi}_{-}\psi_{-}H_{+}^{0}H_{+}^{0}\vert H^{0}(k)H^{0}(k')\rangle\]


reemplazando lo anterior en esta ecuancion nos queda

\[
S_{fi}^{(2)}=\dfrac{-kv\lambda_{L}}{4}\int d^{4}x_{1}\int d^{4}x_{2}\imath\Delta_{F}(x_{1}-x_{2})\langle0\vert\dfrac{\delta_{ss'}}{V\sqrt{4E(p)E(p')}}\bar{u}_{s}(p)e^{\imath p\cdot x_{2}}v_{s'}(p')e^{\imath p'\cdot x_{2}}\dfrac{2}{V\sqrt{4E(k)E(k')}}e^{-\imath k\cdot x_{1}}e^{-\imath k'\cdot x_{1}}\vert0\rangle\]


ya que $\langle0\vert0\rangle=\text{1}$ nos queda 

\[
S_{fi}^{(2)}=\dfrac{-kv\lambda_{L}}{4}\int d^{4}x_{1}\int d^{4}x_{2}\imath\Delta_{F}(x_{1}-x_{2})\dfrac{\delta_{ss'}}{V\sqrt{4E(p)E(p')}}\bar{u}_{s}(p)e^{\imath p\cdot x_{2}}v_{s'}(p')e^{\imath p'\cdot x_{2}}\dfrac{1}{V\sqrt{4E(k)E(k')}}e^{-\imath k\cdot x_{1}}e^{-\imath k'\cdot x_{1}}\]


reagrupando ternimos

\[
S_{fi}^{(2)}=\dfrac{-kv\lambda_{L}}{2}\int d^{4}x_{1}\int d^{4}x_{2}\dfrac{\delta_{ss'}}{V^{2}\sqrt{16E(p)E(p')E(k)E(k')}}\bar{u}_{s}(p)v_{s'}(p')\imath\Delta_{F}(x_{1}-x_{2})e^{\imath(p+p')\cdot x_{2}}e^{-\imath(k+k')\cdot x_{1}}\]


tenemos que el propagador es $\imath\Delta_{F}(x_{1}-x_{2})=\int\dfrac{d^{4}p}{(2\pi)^{4}}\imath\Delta_{F}(q)e^{\imath q\cdot(x_{1}-x_{2})}$

\[
S_{fi}^{(2)}=\dfrac{-kv\lambda_{L}}{2}\int d^{4}x_{1}\int d^{4}x_{2}\dfrac{\delta_{ss'}}{4V^{2}\sqrt{E(p)E(p')E(k)E(k')}}\bar{u}_{s}(p)v_{s'}(p')\int\dfrac{d^{4}q}{(2\pi)^{4}}\imath\Delta_{F}(q)e^{\imath q\cdot(x_{1}-x_{2})}e^{\imath(p+p')\cdot x_{2}}e^{-\imath(k+k')\cdot x_{1}}\]


\[
S_{fi}^{(2)}=\dfrac{-kv\lambda_{L}}{2}\int\dfrac{d^{4}q}{(2\pi)^{4}}\int d^{4}x_{1}\int d^{4}x_{2}\dfrac{\delta_{ss'}}{4V^{2}\sqrt{E(p)E(p')E(k)E(k')}}\bar{u}_{s}(p)v_{s'}(p')\imath\Delta_{F}(q)e^{\imath(p+p'-q)\cdot x_{2}}e^{-\imath(k+k'-q)\cdot x_{1}}\]


integrando en $x_{2}$nos queda

\[
S_{fi}^{(2)}=\dfrac{-kv\lambda_{L}}{2}\int d^{4}q\int d^{4}x_{2}\dfrac{\delta_{ss'}}{4V^{2}\sqrt{E(p)E(p')E(k)E(k')}}\bar{u}_{s}(p)v_{s'}(p')\imath\Delta_{F}(q)\delta^{4}(p+p'-q)e^{\imath(q-k-k')\cdot x_{1}}\]


integrando en $x_{1}$

\[
S_{fi}^{(2)}=\dfrac{-kv\lambda_{L}}{2}\int d^{4}q\dfrac{\delta_{ss'}}{4V^{2}\sqrt{E(p)E(p')E(k)E(k')}}\bar{u}_{s}(p)v_{s'}(p')\imath\Delta_{F}(q)\delta^{4}(p+p'-q)(2\pi)^{4}\delta^{4}(q-k-k')\]


integrando en $q$

\[
S_{fi}^{(2)}=(2\pi)^{4}\dfrac{\delta_{ss'}\delta^{4}(p+p'-k-k')}{4V^{2}\sqrt{E(p)E(p')E(k)E(k')}}\bar{u}_{s}(p)v_{s'}(p')\imath\Delta_{F}(k+k')\]


donde la amplitud

\[
\mathcal{M}_{fi}=\dfrac{-\imath kv\lambda_{L}}{2}\bar{u}_{s}(p)v_{s'}(p')\Delta_{F}(k+k')\]


pero lo que nesecitamos es la norma de la amplitud, y sumemos sobre
todo s, y s'

\[
\vert\mathcal{M}_{fi}\vert^{2}=\mathcal{M}_{fi}^{\dagger}\mathcal{M}_{fi}=\left(\dfrac{kv\lambda_{L}}{2}\right)^{2}\sum_{s,s'}\bar{u}_{s}(p)v_{s'}(p')\bar{v}_{s'}(p')u_{s}(p)\Delta_{F}^{2}(k+k')\]


miremos que es 

\[
\sum_{s,s'}\bar{u}_{s}(p)v_{s'}(p')\bar{v}_{s'}(p')u_{s}(p)=\sum_{s}\bar{u}_{s}(p)u_{s}(p)\sum_{s'}v_{s'}(p')\bar{v}_{s'}(p')_{s'}\]


\[
\sum_{s}\bar{u}_{\beta}(p,s)u_{\alpha}(p,s)\sum_{s'}v_{\alpha}(p',s')\bar{v}_{\beta}(p',s')_{s'}=(\gamma_{\mu}p^{\mu}+m)_{\beta\alpha}(\gamma_{\nu}p'^{\nu}-m)_{\text{\ensuremath{\alpha\beta}\ }}\]
\[
Tr\left((\gamma_{\mu}p^{\mu}+m)(\gamma_{\nu}p'^{\nu}-m)\right)=Tr(\gamma_{\mu}\gamma_{\nu}p^{\mu}p'^{\nu}+\gamma_{\nu}mp'^{\nu}-\gamma_{\mu}mp^{\text{\ensuremath{\mu}}}-m^{2})\]


por propiedades del traza y teniendo en cuenta $Tr(\gamma_{\nu})=0$
; $Tr(\gamma_{\mu}\gamma_{\nu})=4g_{\mu\nu}$

\[
Tr(\gamma_{\mu}\gamma_{\nu}p^{\mu}p'^{\nu})+Tr(\gamma_{\nu}mp'^{\nu})-Tr(\gamma_{\mu}mp^{\text{\ensuremath{\mu}}})-Tr(m^{2})=4(p^{\mu}p'_{\nu}-m^{2})\]


entoces

\[
\vert\mathcal{M}_{fi}\vert^{2}=\left(kv\lambda_{L}\right)^{2}(p^{\mu}p'_{\nu}-m^{2})\Delta_{F}^{2}(k+k')\]



\section*{Seccion Eficaz}

\[
\dfrac{d\sigma}{d\Omega}=\dfrac{3}{64\pi^{2}s}\left\{ \dfrac{\left[s-(m'_{1}+m'_{2})^{2}\right]\left[s-(m'_{1}-m'_{2})^{2}\right]}{\left[s-(m_{1}+m_{2})^{2}\right]\left[s-(m_{1}-m_{2})^{2}\right]}\right\} ^{\dfrac{1}{2}}\vert\mathcal{M}_{fi}\vert^{2}\]


con $s=4E_{p}^{2}$, donde $m'_{1}=m'_{2}$y $m_{1}=m_{2}$

\[
\dfrac{d\sigma}{d\Omega}=3\dfrac{\left(kv\lambda_{L}\right)^{2}}{64\pi^{2}s}\left\{ \dfrac{\left[s-(2m'_{1})^{2}\right]}{\left[s-(2m_{1})^{2}\right]}\right\} ^{\dfrac{1}{2}}(p^{\mu}p'_{\mu}-m^{2})\Delta_{F}^{2}(k+k')\]


por conservacion del cuadrimomentun $p+p'=k+k'$ y las masas igules
y para simplificar las particulas iniciales tienen el misom momentun
tenemos que $2E_{p}=2E_{k}$ y $\vert\vec{p}\vert=\vert\text{\ensuremath{\vec{p}}}'\vert$

entonces \[
p^{\mu}p'_{\mu}=E_{p}^{2}-\vert p\vert^{2}=m_{H}^{2}\]


\[
\Delta_{F}(k+k')=\dfrac{1}{((k+k')^{2}-4m_{h}^{2})}=\dfrac{1}{2\left(kk'-m_{h}^{2}\right)}=\dfrac{1}{2(E_{k}^{2}-\vert k\vert^{2}-m_{h}^{2})}=\dfrac{1}{2(E_{p}^{2}-\vert p\vert^{2}-m_{h}^{2})}\]


y nos deja 

\[
\dfrac{d\sigma}{d\Omega}=3\dfrac{\left(kv\lambda_{L}\right)^{2}}{64\pi^{2}s}\left\{ \dfrac{\left[s-(2m'_{1})^{2}\right]}{\left[s-(2m_{1})^{2}\right]}\right\} ^{\dfrac{1}{2}}\left[\dfrac{2E_{p}^{2}-m_{H}^{2}}{4(E_{p}^{2}-m_{h}^{2})}\right]^{2}\]


y la seccion eficaz es \[
\sigma=3\dfrac{\left(kv\lambda_{L}\right)^{2}}{64\pi s}\left\{ \dfrac{\left[s-(2m)^{2}\right]}{\left[s-(2m_{H})^{2}\right]}\right\} ^{\dfrac{1}{2}}\left[\dfrac{2E_{p}^{2}-m_{H}^{2}}{4(E_{p}^{2}-m_{h}^{2})}\right]^{2}=3\dfrac{\left(kv\lambda_{L}\right)^{2}}{64\pi s}\left\{ \dfrac{\left[s-(2m)^{2}\right]}{\left[s-(2m_{H})^{2}\right]}\right\} ^{\dfrac{1}{2}}\left[\dfrac{s/2-m_{H}^{2}}{(s-4m_{h}^{2})}\right]^{2}\]


datos 

$\lambda_{L}=0.1\,\,\,\,\,\,\,\,\, k=-\dfrac{m}{v}$

$m_{h}=120\, GeV\,\,\,\,\,\,\,\,\,\,\, m_{H}=50GeV\,\,\,\,\,\,\,\,\, s=(1000GeV)^{2}\,\,\,\,\,\,\,\,\: m=4.3\text{GeV}\,\,\,\,\,\,\, v=246GeV$

luego de poner todos eso numeros en su lugar

$s=7.726888\times10^{-10}GeV^{-2}=0.030086184pb$


\section*{CalcHep }

$\sigma_{c}=3.48145\times10^{-5}pb$

inexplicable!!!!!!!!

\section{Copyright}
\includegraphics[scale=0.5]{cc} Creative Commons Attribution-Share Alike 3.0 United States License.


%%% Local Variables: 
%%% mode: latex
%%% TeX-master: "qft_samples"
%%% End: 
